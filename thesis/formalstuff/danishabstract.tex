\chapter*{Dansk sammenfatning}
Formelle metoder er ikke altid lige anvendelige i software designprocesser. Vi {\o}nsker at forbedre dette ved at udvide eksisterende modelleringsformalismer. Hovedproblemet er, at de eksisterende formalismer til systemspecifikation ikke altid er i stand til at modellere de reelle systemkoncepter i det rigtige abstraktionsniveau. Derudover skalerer de eksisterende frameworks ikke godt nok til at de kan anvendes p{\aa} store og mere komplekse systemer.

Vi studerer udvidelser af \emph{Modale Transitionssystemer} (MTS), et framework til systemmodellering, der underst{\o}tter system-raffinering. Det er fordelagtigt i designprocesser, hvor et system er designet gradvist fra en overordnet specifikation ned til en komplet systemspecifikation med en fastlagt implementering.

Til at starte med udvider vi MTS-frameworket med en r{\ae}kke konfigurationsparametre. Disse parametre g{\o}r det muligt at modellere persistente valg i systemmodellen.
Dern{\ae}st udvider vi modelleringssproget med tid og priser, som kan bruges til at analysere omkostningerne ved at bygge et system og k{\o}re systemet. Det kan hj{\ae}lpe udviklere, allerede tidligt i designfasen, med at sikre, at systemet overholder budgetterne. Til sidst designer vi en logik til MTS-frameworket. I denne logik kan man udtrykke formelle systemkrav og verificere, hvorvidt systemmodellen opfylder disse. Hvis systemmodellen har konfigurationsparametre, s{\aa} viser vi ogs{\aa}, hvordan man kan bruge logikken til at afg{\o}re for hvilke konfigurationer af systemet kravene er opfyldt.

Vi studerer ogs{\aa} kommunikationsegenskaber i komponent-baserede designmetoder. Komponent-baserede designs er favorable, n{\aa}r man udvikler st{\o}rre og komplekse systemer, da systemet bliver delt op i mindre moduler, der kommunikerer. Det kan v{\ae}re med til at give et bedre overblik, men det medf{\o}rer ogs{\aa} et nyt problem, nemlig det at modulerne skal kunne kommunikere korrekt. Derfor studerer vi en r{\ae}kke kommunikationsegenskaber p{\aa} asynkront sammensatte systemer. Vi betragter systemer modelleret som asynkrone I/O Petri nets, og vi beviser, at alle vores kommunikationsegenskaber er afg{\o}rbare i dette setup. Derudover beviser vi, at de egenskaber et modul har, ikke {\ae}ndres, hvis det bliver sat sammen med et andet modul. Det betyder, at vores framework underst{\o}tter trinvis design.

Til sidst udvider vi vores asynkrone I/O Petri nets med modaliteter. Det giver os et formelt modelleringsframework med fordelene fra MTS, hvor vi kan unders{\o}ge kommunikationsegenskaber selv under raffinering af modellen. I alt giver det os en metode til at designe distribuerede systemer, der underst{\o}tter, trinvis design, indkapsling af moduler og trinvis raffinering.
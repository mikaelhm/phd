%%% Modal transition systems arcs 
%%% by Mikael H. Møller - mikael@i-dyllen.dk


%%% required packages:
\RequirePackage[T1]{fontenc}
\RequirePackage{tikz} 

%%% Arrow Commands
\newcommand{\may}[1]{\stackrel{#1}{\mayarrow}}
\newcommand{\must}[1]{\stackrel{#1}{\mustarrow}}
\newcommand{\mayw}[1]{\stackrel{#1}{\weakmayarrow}}
\newcommand{\mustw}[1]{\stackrel{#1}{\weakmustarrow}}

\newcommand{\lmay}[1]{\stackmay{#1}{\mayfill}}
\newcommand{\lmust}[1]{\stackmust{#1}{\mustfill}}
\newcommand{\lmayw}[1]{\stackmay{#1}{\maywfill}}
\newcommand{\lmustw}[1]{\stackmust{#1}{\mustwfill}}


%%% Arrow drawings (normal arrows)
\newcommand{\mayarrow}{\mathrel{
\begin{tikzpicture}[line width=.4, scale=0.13,line cap=round]
%set symbolarea:
\clip (-.06,-.7) rectangle (4.0,.7);
%draw dashed line:
\draw (0,0) -- (.9,0);
\draw (1.4,0) -- (2.2,0);
\draw (2.7,0) -- (3.6,0);
%draw arrow tip (symbol requires \usepackage[T1]{fontenc}):
\node at (3.5,0) {\upshape{\guilsinglright}};
\end{tikzpicture}
}}

\newcommand{\mustarrow}{\mathrel{
\begin{tikzpicture}[line width=.4, scale=0.13,line cap=round]
%set symbolarea:
\clip (-.06,-.7) rectangle (4.0,.7);
%draw full line:
\draw (0,0) -- (3.6,0);
%draw arrow tip (symbol requires \usepackage[T1]{fontenc}):
\node at (3.5,0) {\upshape{\guilsinglright}};
\end{tikzpicture}
}}

\newcommand{\weakmayarrow}{\mathrel{
\begin{tikzpicture}[line width=.4, scale=0.13,line cap=round]
%set symbolarea:
\clip (-.06,-.7) rectangle (4.0,.7);
%draw upper dashed line:
\draw (0,.2) -- (.9,.2);
\draw (1.4,.2) -- (2.2,.2);
\draw (2.7,.2) -- (3.45,.2);
%draw lower dashed line:
\draw (0,-.2) -- (.9,-.2);
\draw (1.4,-.2) -- (2.2,-.2);
\draw (2.7,-.2) -- (3.45,-.2);
%draw arrow tip (symbol requires \usepackage[T1]{fontenc}):
\node at (3.5,0) {\upshape{\guilsinglright}};
\end{tikzpicture}
}}

\newcommand{\weakmustarrow}{\mathrel{
\begin{tikzpicture}[line width=.4, scale=0.13,line cap=round]
%set symbolarea:
\clip (-.06,-.7) rectangle (4.0,.7);
%draw upper full line:
\draw (0,.2) -- (3.45,0.2);
%draw lower full line:
\draw (0,-.2) -- (3.45,-.2);
%draw arrow tip (symbol requires \usepackage[T1]{fontenc}):
\node at (3.5,0) {\upshape{\guilsinglright}};
\end{tikzpicture}
}}

%%% Arrow drawings (long arrows)
%
% Here only the line of the arrow is drawn, and then repeated using \xleaders
\newcommand{\arrowend}{\mathrel{
\begin{tikzpicture}[baseline=-0.559ex, line width=.4, scale=0.13,line cap=round]
%set symbolarea:
\clip (-.06,-.7) rectangle (1.06,.7);
%draw dashed line:
\draw (0,0) -- (.9,0);
\end{tikzpicture}
}}

\newcommand{\weakarrowend}{\mathrel{
\begin{tikzpicture}[baseline=-0.559ex, line width=.4, scale=0.13,line cap=round]
%set symbolarea:
\clip (-.06,-.7) rectangle (1.06,.7);
%draw upper dashed line:
\draw (0,.2) -- (1,.2);
%draw lower dashed line:
\draw (0,-.2) -- (1,-.2);
\end{tikzpicture}
}}

%% Long arrows (repeating the line segnemt)
\newcommand{\mayfill}{$
\xleaders\hbox{\hspace{0.2ex}$\arrowend$}
\hfill\text{\hspace{-.74ex}\upshape{\guilsinglright}}$} 

\newcommand{\mustfill}{$
\xleaders\hbox{\hspace{-0.4ex}$\arrowend$}
\hfill\text{\hspace{-.74ex}\upshape{\guilsinglright}}$} 

\newcommand{\maywfill}{$
\xleaders\hbox{\hspace{0.2ex}$\weakarrowend$}
\hfill\text{\hspace{-.54ex}\upshape{\guilsinglright}}$} 

\newcommand{\mustwfill}{$
\xleaders\hbox{\hspace{-0.4ex}$\weakarrowend$}
\hfill\text{\hspace{-.54ex}\upshape{\guilsinglright}}$} 

%% adjusting text on long arrows
\newcommand{\stackmay}[2]{\stackrel{\hspace{0.8ex}#1\hspace{1ex}}{\hbox{#2}}}
\newcommand{\stackmust}[2]{\stackrel{\hspace{0ex}#1\hspace{1ex}}{\hbox{#2}}}



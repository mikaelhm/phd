%!TEX root = /Users/mikaelhm/projects/biPriced-MTS/bipriced/main.tex

We first focus on the implementation problem without considering 
the hardware ($H=\emptyset$).
We show how the implementation problem can be solved by reduction to the
weighted MPGs. The first translation we present is exponential, 
however, we provide methods for making it smaller 
%afterwards.
in the subsequent section.

We are given an~MTSD $\Spec = (S,T,D,\Phi,s_0)$ and a~dual-price scheme
$(r,H,\Psi,\hc)$ and assume that there is no state $s$ with $\emptyset \in
\val(s)$. 
%, i.e.~no deadlocks are allowed.  
Let us define the following auxiliary
vertices that will be used to simulate the more complicated transitions of MTSD
in the simpler setting of weighted MPG (by convention all singleton
intervals are treated as uncontrollable).
\begin{align*}
T_\un &= \{(s,a,t) \mid (s,a,t) \in T;\,D(s,a,t) \in \mathcal I_\un\}\\
T_\cn &= \{(s,a,t) \mid (s,a,t) \in T;\,D(s,a,t) \in \mathcal I_\cn\}\\
\modT  &= \{(s,a,j,t) \mid (s,a,t) \in T;\,j \in D(s,a,t) \}
\end{align*}
We construct the weighted mean-payoff game with
%\begin{itemize}
%\item $V = S  \cup 2^T \cup T \cup \modT$
%\item 
$V_{min} = S \cup T_\cn \cup \modT$, 
%\item 
$V_{max} = 2^T \cup T_\un$ and
%\item 
$E$ defined as follows:
\begin{align*}
	(s, X) &\in E &&\iff \exists V \in \val(s) : 
	X = \{ (s,a,t) \mid (a,t) \in V\}\\
	(X,(s,a,t)) &\in E &&\iff (s,a,t) \in X\\
	((s,a,t),(s,a,j,t)) &\in E &&\iff j \in D(s,a,t)\\
	((s,a,j,t),t) &\in E && \text{(always)}
\end{align*}
Further, 
%\item 
$r((s,a,j,t),t) = r(a)$, $d((s,a,j,t),t) = j$
and $r(-,-) = d(-,-) = 0$ otherwise.
%\end{itemize}

% A~strategy for \emph{min} can now be translated into an implementation of the
% original MTSD in a straightforward way. The implemented transitions in $s$ are
% given by $\sigma(s)$, similarly the durations of a transition $(s,a,t)$ with a controllable
% interval is given by the third component of $\sigma((s,a,t))$. Details are given in Appendix.


\begin{example}
In Figure~\ref{fig:example:exp} we show an example of how this
translation to weighted MPG works. For simplicity we only translate a part of
the MTSD $\Spec$ shown in Figure~\ref{fig:trans1}. 
%Note that $S$ is not a valid specification, since it contain deadlocks. 
The resulting weighted MPG is shown in Figure~\ref{fig:trans2}. The
diamond shaped states belong to $min$ and the squared states belong to $max$.
In the vertex $s$, $min$ chooses which outgoing transition are implemented.
Only the choices satisfying $\Phi(s)$ are present in the game. Afterwards, $max$
decides which transition to take. The chosen transition is then assigned 
by one of the players a time that it is going to take.

Notice that $(s,a,t_1)$ is the only
transition controlled by $min$, because it has a controllable interval
$\ci{2,3}$. The remaining transitions with uncontrollable intervals are operated
by $max$ who chooses the time from these intervals. All the ``auxiliary''
transitions are displayed without any labels meaning their duration (and rate)
is zero. Thus, only the transitions corresponding to ``real'' transitions in
MTSDs are taken into account in the value of every play.
\end{example}
%!TEX root = ../../thesis.tex

\begin{figure}[ht]
	\subfloat[A part of an MTSD $S$]{\label{fig:trans1}
		\begin{tikzpicture}[node distance=1.2]
			\node[state,initial] (s) {$s$};
			\node[state,right=of s] (t2) {$t_2$};
			\node[state,above=of t2] (t1) {$t_1$};
			\node[state,below=of t2] (t3) {$t_3$};
			
			\node[below=2 of s]{$\Phi(s)=(a,t_1)\wedge ((b,t_2) \vee (c,t_3))$};
			
			\draw [arc]
				(s) edge[bend left] node[sloped,above] {$a$} node[sloped,below] {$\ci{2,3}$} (t1) 
				(s) edge[dashed] node[sloped,above] {$b$} node[sloped,below] {$\ui{1,3}$} (t2) 
				(s) edge[dashed, bend right] node[sloped,above] {$c$} node[sloped,below] {$3$} (t3) 
			{};
		\end{tikzpicture} }
	\subfloat[Resulting weighted MPG from translation of $S$]{\label{fig:trans2}
	\scalebox{\myscale}{
	\begin{tikzpicture}[node distance=1.2]
		\node[min] (s) {$s$};
		
		\node[max,right=0.2 of s] (s12) {$\{(s,a,t_1),(s,c,t_3) \}$};
		\node[max,above=1.3 of s12] (s11) {$\{(s,a,t_1),(s,b,t_2)\}$};
		\node[max,below=1.3 of s12] (s13) {
		\begin{minipage}{96pt}
		  $\{(s,a,t_1),(s,b,t_2),\\ ~\;\;(s,c,t_3)\}$
		 \end{minipage}
		};
		
		\node[max,right=0.5 of s12] (s22) {$(s,b,t_2)$};
		
		\node[min,above=of s22,yshift=0] (s21) {$(s,a,t_1)$};
		\node[max,below=of s22, yshift=-22] (s23) {$(s,c,t_3)$};


		\node[min,node distance=2,right=0.5 of s21] (s32) {$(s,a,3,t_1)$};
		\node[min,node distance=\diadist,above left=of s32] (s31) {$(s,a,2,t_1)$};
		\node[min,node distance=\diadist,below left=of s32] (s33) {$(s,b,1,t_2)$};		
		\node[min,node distance=\diadist,below right=of s33] (s34) {$(s,b,2,t_2)$};

		\node[min,node distance=\diadist,below left=of s34] (s35) {$(s,b,3,t_2)$};
		\node[min,node distance=\diadist,below right=of s35] (s36) {$(s,c,3,t_3)$};
		
		\node[min,right=0.6 of s32] (t1) {$t_1$};
		\node[min,right=0.6 of s34] (t2) {$t_2$};
		\node[min,right=0.6 of s36] (t3) {$t_3$};
		
		\draw [arc]
			(s) edge (s11) (s) edge (s12) (s) edge (s13)
			
			(s11) edge (s21) (s11) edge (s22) 
			(s12) edge (s21) (s12) edge (s23)
			(s13.20) edge (s21) (s13.20) edge (s22) (s13) edge (s23)
			
			(s21) edge (s31) (s21) edge (s32)
			(s22) edge (s33) (s22) edge (s34.180) (s22) edge (s35)
			(s23) edge (s36.180)
			
			(s31.0) to[bend left] node[sloped,above] {2} node[sloped,below] {$r(a)$}(t1)
			(s32.0) to node[sloped,above] {3} node[sloped,below] {$r(a)$}(t1)
			(s33.0) to[bend left] node[sloped,above] {1} node[sloped,below] {$r(b)$} (t2)
			(s34.0) to node[sloped,above] {2} node[sloped,below] {$r(b)$}(t2)
			(s35.0) to[bend right] node[sloped,above] {3} node[sloped,below] {$r(b)$}(t2)
			
			(s36.0) to node[sloped,above] {3} node[sloped,below] {$r(c)$}(t3)
		{};
	\end{tikzpicture} }}
	
\caption{Translating MTSD to weighted MPG.}\label{fig:example:exp}
\end{figure}



% A~strategy for \emph{min} can now be translated into an implementation of the
% original MTSD in a straightforward way. The implemented transitions in $s$ are
% given by $\sigma(s)$, similarly the durations of a transition $(s,a,t)$ with a controllable
% interval are given by the third component of $\sigma((s,a,t))$. 
%Details are given in Appendix.

We now show how a~strategy for player \emph{min} in the constructed weighted MPG may
be translated into an implementation of the original MTSD. In Section~\ref{mpgd}, we have shown that there are optimal \emph{positional} strategies. Hence, we may safely restrict this translation to positional strategies.
%, and for sake of simplicity, we do so. 
Let $\sigma$ be such a positional
strategy. We build the implementation as $\Impl=(S',T',D',\Phi',s_{0\ii})$ where
\begin{itemize}
\item $S' = \{  s_\ii \mid s \in S \}$
\item $(s_\ii,a,t_\ii) \in T'$ if $(s,a,t) \in X$ where $\sigma(s) = X$
\item $D'(s_\ii,a,t_\ii) = D(s,a,t)$ if $D(s,a,t) \in \mathcal I_\mathsf u$
\item $D'(s_\ii,a,t_\ii) = j$ if $D(s,a,t) \in \mathcal I_\mathsf c$ and
	$\sigma((s,a,t)) = (s,a,j,t)$
\item $\Phi'(s_\ii) = \bigwedge_{(s_\ii,a,t_\ii)\in T'} (a,t_\ii)$
\end{itemize}
The set of states remains the same as in the original MTSD; we change every
state $s$ into $s_\ii$ to be able to reason about the states of the original
MTSD and of the implementation separately.
%
The following two lemmas state that the construction is sound.


\begin{lemma}\label{lem:strat2impl}
The constructed implementation $\Impl$ is %indeed 
an implementation of the original
MTSD. 
\end{lemma}

\begin{proof}
We show that $R = \{ (s_\ii,s) \mid s \in S \}$ is a~modal refinement relation.
Let $(s_\ii,s) \in R$ and let $M \in \val(s_\ii)$. Clearly 
$M = \{ (a,t_\ii) \mid (s_\ii,a,t_\ii) \in T'\}$.
We take $N = \{ (a,t) \mid (a,t_\ii) \in M \}$. The fact that $N \in \val(s)$ 
is clear from the construction. We now need to show that
$D'(s_\ii,a,t_\ii) \le D(s,a,t)$. 
If $D(s,a,t) \in \mathcal I_\un$ then $D'(s_\ii,a,t_\ii) = D(s,a,t)$ and
the statement holds.
If $D(s,a,t) \in \mathcal I_\cn$ then $D'(s_\ii,a,t_\ii) = j$ where
$j \in D(s,a,t)$ due to the construction.
Thus $R$ is a~modal refinement relation where 
$s_\ii \mr s$ for all $s \in S$ %also $s_{0\ii} \mr s_0$ 
and therefore $\Impl \mr \Spec$.%\qed
\end{proof}



\begin{lemma}\label{lem:keepsvalue}
We have $\rc(\Impl)\leq\lambda$ if and only if the strategy $\sigma$ 
%guarantees $v(s_0) \leq \lambda$.
ensures a value of at most $\lambda$.
\end{lemma}

\begin{proof}
Every run of the implementation corresponds to a~play (where player \emph{min}
plays according to strategy $\sigma$) and vice versa. Hence, the worst case of
the long run average over all runs in $\Impl$ is the same as over all plays
according to $\sigma$.%\qed
\end{proof}

We conclude the proof of correctness of the reduction by showing its completeness.

\begin{lemma}\label{impl2strat}
For every implementation $\Impl$ of $\Spec$, there exists
a~strategy $\sigma$ for player \emph{min} such that $\sigma$ ensures value of
at most $\rc(\Impl)$.
\end{lemma}
\begin{proof}
 We show how to transform an arbitrary implementation $\Impl$ into a strategy $\sigma:V^*\to V$ (depending on the whole prefix of a play where we currently are) for player
\emph{min} that guarantees the same or smaller value. (Although this is not a positional
strategy, we know there is also a positional strategy ensuring the same or
smaller value.) The idea of the construction is that for each history $\sigma$ mimics the behaviour of the implementation at its respective state. In other words, the decision of $\sigma$
in some history is based on mapping the history to the implementation and doing
what the implementation does. However, there is a small catch: the implementation may implement more possible decisions (its branching may be even uncountable). Nonetheless, we may take any of the decisions, as the resulting strategy then corresponds to a pruning of the original implementation and the strategy's worst case long run average is thus either the same or even smaller. 

Let $i_0$ be the initial state of $\Impl$ and $s_0$ the initial state of
$\Spec$; we have~$i_0\mr s_0$. The corresponding vertex of the constructed
weighted MPG also bears the name $s_0$.  We first define a mapping $\mu:V^*\to
I^*$ from paths in the weighted MPG of the form $s_0\, (s_0,N_0)\,
(s_0,a_0,s_1)\, (s_0,a_0,j_0,s_1)\, s_1\, \cdots s_{n}$ to sequences of states
of $\Impl$ inductively as follows.
\begin{align*}
 \mu(s_0)&=i_0\\
 \mu(s_0\cdots s_n\, (s_n,N_n)\, (s_n,a_n,s_{n+1})\, (s_n,a_n,j_n,s_{n+1})\, s_{n+1})&=\mu(s_0\cdots s_n)i_{n+1} 
\end{align*}
where $i_{n+1}$ is an arbitrary state satisfying $(i_n,a_n,i_{n+1})\in T$ and
$i_{n+1}\mr s_{n+1}$.  The image of $\mu$ now defines the desired pruning of
$\Impl$ that is still an implementation of $\Spec$, has at most the same worst
case long run average, and can now be canonically mapped to a deterministic
strategy $\sigma$ as follows.

Let $\pi=s_0\, (s_0,N_0)\, (s_0,a_0,s_1)\, (s_0,a_0,j_0,s_1)\, s_1\, \cdots x\in V^*$ be a prefix of a play with $x\in V_{min}$. Denote $s_n$ the last element of $\Spec$ in $\pi$ and $i_0\cdots i_n=\mu(s_0\cdots s_n)$.
We define $\sigma(\pi)$ as follows.
\begin{itemize}
\item If $x\in S$, then $\pi=s_0\cdots s_n$ and due to $i_n\mr s_n$ there exists $N$ from the definition of modal refinement. We set $\sigma_s(\pi) = (s_n,N)$.
\item If $x\in T_\cn$, then $\pi=s_0\cdots s_n\, (s_n,N_n)\,(s_n,a_n,s_{n+1})$ and there exists a unique $i_{n+1}$ such that $i_0\cdots i_ni_{n+1}=\mu(s_0\cdots s_n s_{n+1})$. Here we use the fact that $\mu$ does not depend on which vertices of the form $(s,a,j,t)$ were visited. We set $\sigma(\pi) = (s_n,a_n,D(i_n,a_n,i_{n+1}),s_{n+1})$. 
\item If $x\in \modT$, then $x=(s_n,a_n,j_n,s_{n+1})$ and there exists only one outgoing edge. We set $\sigma(\pi)=s_{n+1}$.
\end{itemize}
Now for each play that player \emph{min} plays according
to $\sigma$, there is a~run in the original implementation that has the same long-run average.
Hence the supremum over all plays is at most the supremum over all runs. %\qed
\end{proof}


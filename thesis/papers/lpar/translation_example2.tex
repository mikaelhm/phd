%!TEX root = ../../thesis.tex

\begin{figure}[!ht]
	\centering
	% \scalebox{\myscale}{
	\begin{tikzpicture}[node distance=1.3]
		\node[max] (s) {$(s,\Phi(s))$};
		
		\node[min,below=.5 of s] (s11) {$(s,(b,t_2)\vee(c,t_3))$};
		
		\node[min,right=of s] (s21) {$(s,(a,t_1))$};
		\node[max,below=of s21] (s22) {$(s,(b,t_2))$};
		\node[max,below=of s22,yshift=-20.5] (s23) {$(s,(c,t_3))$};


		\node[min,node distance=2,right=of s21] (s32) {$(s,a,3,t_1)$};
		\node[min,node distance=\diadist,above left=of s32] (s31) {$(s,a,2,t_1)$};
		\node[min,node distance=\diadist,below left=of s32] (s33) {$(s,b,1,t_2)$};		
		\node[min,draw=white,node distance=\diadist,below right=of s33,white] (s34) {$(s,b,2,t_2)$};

		\node[min,node distance=\diadist,below left=of s34] (s35) {$(s,b,3,t_2)$};
		\node[min,node distance=\diadist,below right=of s35] (s36) {$(s,c,3,t_3)$};
		
		\node[circle,draw=darkblue,fill=lightlightblue,right=of s32] (t1) {$(t_1,\Phi(t_1))$};
		\node[circle,draw=darkblue,fill=lightlightblue,right=of s34] (t2) {$(t_2,\Phi(t_2))$};
		\node[circle,draw=darkblue,fill=lightlightblue,right=of s36] (t3) {$(t_3,\Phi(t_3))$};
		
		\draw [arc] (s) edge (s11) (s) edge (s21){};
		\draw [arc] (s11) edge (s22) (s11) edge (s23) {};
		\draw [arc] (s21) edge (s31) (s21) edge (s32){};
		\draw [arc] (s22) edge (s33){};
		\draw [arc] (s22) edge (s35){};
		\draw [arc] (s23) edge (s36.180){};

		\draw [arc] (s31.0) to[bend left] node[sloped,above] {2} node[sloped,below] {$r(a)$}(t1) {};
		\draw [arc] (s32.0) to node[sloped,above] {3} node[sloped,below] {$r(a)$} (t1){};
		\draw [arc] (s33.0) to[bend left] node[sloped,above] {1} node[sloped,below] {$r(b)$} (t2){};
		\draw [arc] (s35.0) to[bend right] node[sloped,above] {3} node[sloped,below] {$r(b)$} (t2){};
		\draw [arc] (s36.0) to node[sloped,above] {3} node[sloped,below] {$r(c)$} (t3) 	{};
	\end{tikzpicture} %}
	
\caption{Result of the improved translation of $S$ in Figure~\ref{fig:trans1}.}\label{fig:example:poly}
\end{figure}	


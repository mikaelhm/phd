%!TEX root = ../../thesis.tex


In order to define MTS with durations, we first introduce the notion of
\emph{controllable} and \emph{uncontrollable} duration intervals.
A~controllable interval is a~pair $\ci{m,n}$ where $m,n \in \mathbb N_0$
and $m \le n$. Similarly, an~uncontrollable interval is a~pair $\ui{m,n}$
where $m,n \in \mathbb N_0$ and $m \le n$.
We denote the set of all controllable intervals by $\Intl_\cn$, the set of all
uncontrollable intervals by $\Intl_\un$, and the set of all intervals by 
$\Intl = \Intl_\cn \cup \Intl_\un$.
We also write only $m$ to denote the singleton interval $\ui{m,m}$. 
Singleton controllable intervals need not be handled separately as there is no semantic difference to the uncontrollable counterpart.


We can now formally define modal transition systems with durations. 
In what follows, $\mathcal B(X)$ denotes the set of propositional logic 
formulae over the set $X$ of atomic propositions, where we assume the standard connectives $\wedge,\vee,\neg$.

\begin{definition}[MTSD] 
%\todo{Jan: don't we want the initial state to be in the tuple? we have to handle it separately everywhere:-(}
A~\emph{Modal Transition System with Durations} \allowbreak (MTSD) is a~tuple
$\Spec = (S,T,D,\Phi,s_0)$ such that
\begin{itemize}
 	\item $S$ is a~set of \emph{states} with the
\emph{initial state} $s_0$,
 	\item $T \subseteq S \times \Sigma \times S$ is a set of
\emph{transitions},
 	\item $D : T \to \Intl$ is a~\emph{duration interval
function}, and
 	\item $\Phi : S \to \mathcal B(\Sigma\times S)$ is
an~\emph{obligation function}.
 \end{itemize}     
 We assume that whenever the atomic proposition
$(a,t)$ occurs in the \allowbreak Boolean formula $\Phi(s)$ then also $(s,a,t) \in T$. 

Moreover, we require that there is no cycle of transitions that allows 
for zero accumulated duration, i.e.~there is no path
$s_1a_1s_2a_2\cdots s_n$ where \linebreak $(s_i,a_i,s_{i+1}) \in T$ and $s_n=s_1$
such that for all $i$, the interval $D((s_i,a_i,s_{i+1}))$ 
is of the form either $\ci{0,m}$ or $\ui{0,m}$ for some $m$.
\end{definition}

Note that instead of the basic may and must modalities known
from the classical modal transition  %\todo{could be replaced by \cite{DBLP:conf/lics/LarsenT88}}
systems~(see e.g.~\cite{AHLNW:EATCS:08}), we use
arbitrary boolean formulae over the outgoing transitions of each
state in the system as introduced in~\cite{BKLMS:ATVA:11}. This provides
a higher generality as the formalism is capable to describe, 
apart from standard modal transition systems, also more expressive formalisms
like disjunctive modal transition 
systems~\cite{DBLP:conf/lics/LarsenX90} and transition systems with 
obligations~\cite{benes_et_al:OASIcs:2011:3070}. 
See~\cite{BKLMS:ATVA:11} for a more thorough discussion of this formalism.

In the rest of the paper, 
%we shall often refer to an MTSD 
%$(S,T,D,\Phi)$ simply as $S$.  Moreover, we
%will sometimes assume an implicitly given initial state $s_0 \in S$
%of the MTSD $S$. We 
we adapt the following convention when drawing MTSDs.
Whenever a state $s$ is connected with a solid arrow labelled by $a$
to a state $s'$, this means that in any satisfying assignment of the Boolean
formula $\Phi(s)$, the atomic proposition $(a,s')$ is always set
to true (the transition \emph{must} be present in any refinement of the system).
Should this not be the case, we use a dashed arrow instead (meaning
that the corresponding transition \emph{may} be present in a refinement
of the system but it can be also left out). 
For example the solid edges in Figure~\ref{fig:spec1} correspond
to an implicitly assumed $\Phi(s) = (a,s')$ where $(s,a,s')$ is the
(only) outgoing edge from $s$; in this case we do not explicitly 
write the obligation
function. The three dashed transitions in the figure
are optional, though at least one of them has to
be preserved during any refinement (a feature that can be
modelled for example in disjunctive MTS~\cite{DBLP:conf/lics/LarsenX90}).

\begin{remark} \label{rem:mts}
%The convention introduced above gives us 
The standard notion of modal transition systems 
(see e.g.~\cite{AHLNW:EATCS:08}) is obtained
under the restriction that the formulae $\Phi(s)$ in any state $s \in S$
have the form $(a_1,s_1) \wedge \ldots \wedge (a_n,s_n)$
where $(s,a_1,s_1), \ldots, (s,a_n,s_n) \in T$. The edges mentioned in
such formulae are exactly all must transitions; may transitions are not 
listed in the formula and hence can be arbitrarily set to true or false.  
\end{remark}
Let by $T(s) = \{(a,t)\mid (s,a,t) \in T\}$ denote the set of all 
outgoing transitions from the state $s \in S$.
A~modal transition system with durations is called an~\emph{implementation}
if $\Phi(s) = \bigwedge T(s)$ for all $s \in S$ (every allowed transition
is also required), and 
$D(s,a,s') \in \mathcal I_\mathsf u$ for all $(s,a,s') \in T$, i.e.~all
intervals are uncontrollable, often singletons.  Figure~\ref{fig:impl1} shows
an example of an implementation, while
Figure~\ref{fig:spec2} is not yet an implementation as it still contains
the controllable intervals $\ci{3,5}$ and $\ci{4,6}$.

We now define a~notion of modal refinement. In order to do that, we
first need to define refinement of intervals as a binary relation
$\mathord{\leq} \subseteq \Intl \times \Intl$ such that
\begin{itemize}
\item $\ci{m',n'}\leq \ci{m,n}$ whenever $m' \ge m$ and $n' \le n$, and
\item $\ui{m',n'} \leq \ci{m,n}$ whenever $m' \ge m$ and $n' \le n$.
\end{itemize}
%. An interval $\iota \in \mathcal I$
%refines interval $\iota' \in \mathcal I$, written $\iota \leq \iota'$ if
%\begin{itemize}
%\item $\iota = \iota' = \ui{m,n}$, or
%\item $\iota = \ci{m,n}$, $\iota' = \ci{m',n'}$, $m' \le m$, $n \le n'$, or
%\item $\iota = \ui{m,n}$, $\iota' = \ci{m',n'}$, $m' \le m$, $n \le n'$.
%\end{itemize}
Thus controllable intervals can be refined by narrowing them, 
at most until they become singleton intervals, or  until they are 
changed to uncontrollable intervals. 
Let us denote the collection of all possible sets of outgoing transitions from a state $s$ by
$\val(s) := \{ E \subseteq T(s) \mid E \models \Phi(s) \}$ where
$\models$ is the classical satisfaction relation on propositional
formulae assuming that $E$ lists all true propositions.
%\todo{we haven't defined $\models$ -- Nikola}

\begin{definition}[Modal Refinement]
Let $\Spec_1 = (S_1,T_1,D_1,\Phi_1,s_1)$ and $\Spec_2 = (S_2,T_2,D_2,\Phi_2,s_2)$ be two MTSDs.
A binary relation
$R \subseteq S_1 \times S_2$ is a~\emph{modal refinement} if
for every $(s,t) \in R$ the following holds:
\begin{align*} 
 \forall M \in & \ \val(s) : \exists N \in \val(t) : \\
& \ \forall (a,s') \in M : \exists (a,t') \in N : 
%%\left(
%%	(s',t') \in R
%	((s,a,s'),(t,a,t')) \in \mathrm{Match}_R
%%	\right)
D_1(s,a,s') \leq D_2(t,a,t')
\ \wedge\ 
(s',t') \in R  \text{ and }
%	\ \ \wedge 
\\
& \ \forall (a,t') \in N : \exists (a,s') \in M : 
%%	(s',t') \in R\ 
	%((s,a,s'),(t,a,t')) \in \mathrm{Match}_R
D_1(s,a,s') \leq D_2(t,a,t')
\ \wedge\ 
(s',t') \in R \ .
\end{align*}
%where $\mathrm{Match}_R = \{ ((s,a,s'),(t,a,t')) \mid (s',t') \in R
%\wedge 
%D_1(s,a,s') \text{ refines } D_2(t,a,t')\}$. 
We say that $s \in S_1$ modally refines $s' \in S_2$, 
denoted by $s \mr s'$, if there
exists a~modal refinement $R$ such that $(s,s') \in R$.
We also write $\Spec_1 \mr \Spec_2$ if $s_1 \mr s_2$.
%$s_1$ and $s_2$ are the implicitly given initial states of $S_1$ and $S_2$.
\end{definition}

Intuitively, the pair $(s,t)$ can be in the relation $R$ if for any
satisfiable instantiation of outgoing edges from $s$ there is
a satisfiable instantiation of outgoing edges from $t$ so that they
can be mutually matched, possibly with $s$ having more refined
intervals, and the resulting states are again in the relation $R$.

Observe that in our running example 
the following systems are in modal 
refinement: $\Impl_1\mr \Spec_1 \mr \Spec$ and thus also
$\Impl_1 \mr \Spec$, and similarly $\Impl_2\mr \Spec_2 \mr \Spec$ and thus also  $\Impl_2 \mr \Spec$.

The reader can verify that on the standard modal transition systems (see 
Remark~\ref{rem:mts}) the modal refinement relation corresponds
to the classical modal refinement as 
introduced in~\cite{DBLP:conf/lics/LarsenT88}.  


The algorithm for our problem, given a specification $\Spec$, works as follows.
\begin{enumerate}
\item Nondeterministically choose hardware with the total price at most $max_\ic$.
\item Create the weighted MPG out of $\Spec$.
\item Solve the weighted MPG using the reduction to MPG and any standard
algorithm for MPG that finds an optimal strategy for player \emph{min} and
computes the value $v(s_0)$.
\item Transform the strategy to an implementation $\Impl$.
\item In the case of the cheapest-implementation problem return $\Impl$;\\ in
the case of the implementation (decision) problem return $v(s_0)\leq max_\rc$. 
\end{enumerate}
We can now prove the following result, finishing the proof
of Theorem~\ref{thm:NP}.
%Note that the restricted case has a better complexity.

\begin{proposition}
The implementation problem is in NP.
\end{proposition}
\begin{proof}
We first nondeterministically guess the hardware assignment.
Due to \linebreak Section~\ref{translation}, we know that 
the desired implementation has the same
states as the original MTSD and its transitions are a~subset of the transitions
of the original MTSD as the corresponding optimal strategies are positional.
% can be chosen as positional ones.
The first optimization (Section~\ref{opt}) guarantees that 
durations can be chosen as the extremal points of the intervals. 
Thus we can nondeterministically guess an optimal implementation 
and its durations,
and verify that it satisfies the price inequality.
% \qed
\end{proof}

\begin{proposition}
The implementation problem for MTSD with %$\Phi$ 
positive obligation function and 
a constant number of hardware components
%hardware requirment function in %$\Psi$ in DNF 
is in $\text{NP} \cap \text{coNP}$ 
and solvable in pseudo-polynomial time.
\end{proposition}
\begin{proof}
With the constant number of hardware components,
we get a constant number of possible hardware configurations and 
we can check each configuration separately one by one. 
Further, by the first and the second optimization in Section~\ref{opt}, 
the MPG graph is of size $\bigO(|T|+|\Phi|)$. Therefore, we polynomially reduce the implementation problem to the problem of solving constantly many 
mean payoff games. The result follows by the existence of pseudo-polynomial 
algorithms for MPGs~\cite{Zwick96thecomplexity}.
% \qed 
\end{proof}

Further, our problem is at least as hard as solving MPGs that are clearly 
a special case of our problem. Hence, Theorem~\ref{thm:MPG} follows.


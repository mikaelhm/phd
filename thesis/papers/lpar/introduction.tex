Modal  Transition  Systems  (MTS) is a specification formalism
\cite{DBLP:conf/lics/LarsenT88,AHLNW:EATCS:08} that aims at providing
a flexible  and   easy-to-use  compositional
development methodology  for reactive systems.  The formalism
can be viewed as  a   fragment  of   a  temporal   
%\todo{also a recent follow-up work http://dblp.uni-trier.de/rec/bibtex/journals/corr/abs-1108-4464}
logic~\cite{Aceto:graphical,DBLP:journals/tcs/BoudolL92} that at the same time
offers  a   behavioural compositional semantics with an intuitive
notion of process refinement.
%Recent attention to this logical view of MTS has been given in~\cite{Aceto:graphical}.
The formalism of MTS is essentially a labelled  transition system
that distinguishes two types of labelled transitions:  
\emph{must}  transitions  which  are required in any  refinement
of the system, and  \emph{may}  transitions  that are allowed
to appear in a refined system but are not required.
The refinement  of an  MTS  now  essentially consists  of
iteratively resolving the presence or absence of may transitions in 
the refined process.

In a recent line of work \cite{JLS:weighted,MFCS11}, the MTS framework
has  been  extended  to  allow  for the  specification  of  additional
constraints on  quantitative aspects  (e.g.  time, power  or memory),
which are  highly relevant in the  area of embedded  systems.  In this
paper we  continue the  pursuit of quantitative  extensions of  MTS by
presenting a novel extension of MTS with time durations being modelled
as  controllable or  uncontrollable  intervals. We  further equip  the
model with two kinds of  quantitative aspects: each action has its own
running cost per  time unit, and actions may  require several hardware
components of  different costs.   Thus, we ask  the question,  given a
fixed budget  for the investment into the hardware components, what  is the implementation
with the cheapest long-run average reward.

\vspace{\smallskipamount}
Before we  give a formal  definition of modal transition  systems with
durations (MTSD) and the dual-price scheme, and provide algorithms for
computing  optimal  implementations,  we  present  a  small  motivating
example.

%!TEX root = ../../thesis.tex

\begin{figure}[ht]
\centering
	\begin{tikzpicture}[node distance=100pt]
		\node[state,minimum size=20pt,initial] (s0) {$s$};
		\node[state,minimum size=20pt,right=of s0] (s1) {};
		\node[state,minimum size=20pt,right=of s1] (s2) {};
		\node[state,minimum size=20pt,below=of s2] (s3) {};
		\node[state,minimum size=20pt,below=of s1] (s4) {};
		\node[state,minimum size=20pt,below=of s0] (s5) {$t$};
		\node[below=0.4 of s4] {
			$ \Phi(t)= (\BigClean,s)\ \vee (\SkipClean,s)\ \vee (\SmallClean,s)$};
		
		\draw[arc] 
			(s0) edge node[above] {\Wait} node[below] {$\ci{1,5}$} (s1) 
			(s1) edge node[above] {\DriveBus} node[below] {$\ui{6,10}$} (s2) 
			(s2) edge node[sloped,above] {\SmallClean} node[sloped,below] {$\ci{4,6}$} (s3) 
			(s3) edge node[above] {\Wait} node[below] {$\ci{1,5}$} (s4) 
			(s4) edge node[above] {\DriveBus} node[below] {$\ui{6,10}$} (s5) 
			(s5) edge[dashed,bend left=80] node[sloped,above,yshift=-3] {\SmallClean} node[sloped,below,yshift=2] {$\ci{4,6}$} (s0) 
			(s5) edge[dashed,bend right=70] node[sloped,above,yshift=-1] {\BigClean} node[sloped,below,yshift=2] {$\ci{20,30}$} (s0) 
			(s5) edge[dashed] node[sloped,above,yshift=-3] {\SkipClean} node[sloped,below,yshift=2] {0} (s0) 
		{};
	\end{tikzpicture}
	\caption{Example of Dual-Priced Modal Transition Systems with Time Durations, specification $\Spec$.\label{fig:spec1}}
\end{figure}


Consider the specification $\Spec$ in Figure~\ref{fig:spec1} 
describing the work of a shuttle bus driver.
He drives a bus between a hotel and the
airport. First, the driver 
has to \Wait for the passengers at the hotel. This can take one
to five minutes. Since this behaviour is required to be present in all the
implementations of this specification, it is drawn as a solid arrow and 
called a \emph{must} transition. Then the driver has to \DriveBus the bus to
the airport (this takes six to ten minutes)
where he has to do a \SmallClean, then
\Wait before he can \DriveBus the bus back to the hotel. When he returns he can
do either a \SmallClean, \BigClean or \SkipClean of the bus before he
continues. Here we do not require a particular option to be realised in the
implementations, hence we only draw the transitions as dashed arrows. As these
transitions may or may not be present in the implementations, they are called
\emph{may} transitions. However, here the intention is to require at least one
option be realised. Hence, we specify this using a propositional formula $\Phi$
assigned to the state $t$ over its outgoing transitions as described
in~\cite{benes_et_al:OASIcs:2011:3070,BKLMS:ATVA:11}. After performing one of the
actions, the driver starts over again. Note that next time 
the choice in $t$ may differ.

Observe that there are three types of durations on the transitions. First,
there are \emph{controllable} intervals, written in angle brackets. The meaning
of e.g.~$\langle1,5\rangle$ is that in the implementation we can instruct the
driver to wait for a fixed number of minutes in the range. Second, there are
\emph{uncontrollable} intervals, written in square brackets. 
The interval~$[6,10]$ on
the \DriveBus transition means that in the implementation we cannot fix any
particular time and the time can vary, say, depending on the traffic 
and it is chosen
nondeterministically by the environment. Third, the degenerated case of a
single number, e.g.~$0$, denotes that the time taken is always constant and
given by this number. In particular, a~zero duration means that the transition
happens instantaneously.
\begin{figure}[ht]
	\centering
	\subfloat[Specification $\Spec_1$.\label{fig:spec2}]{
	\begin{tikzpicture}[node distance=80pt]
			\node[state,minimum size=20pt,initial] (s0) {$s$};
			\node[state,minimum size=20pt,right=of s0] (s1) {};
			\node[state,minimum size=20pt,below=of s1,xshift=-50pt] (s2) {};
			
			\draw[arc] 
				(s0) edge node[above] {\Wait} node[below] {$\ci{3,5}$} (s1) 
				(s1) edge node[sloped,below] {\DriveBus} node[sloped,above] {$\ui{6,10}$} (s2) 
				(s2) edge node[sloped,below] {\SmallClean} node[sloped,above] {$\ci{4,6}$}(s0)		{};
		\end{tikzpicture}
	}
	\qquad
	\subfloat[Implementation $\Impl_1$.\label{fig:impl1}]{
		\begin{tikzpicture}[node distance=80pt]
			\node[state,minimum size=20pt,initial] (s0) {$s$};
			\node[state,minimum size=20pt,right=of s0] (s1) {};
			\node[state,minimum size=20pt,below=of s1,xshift=-50pt] (s2) {};
			
			\draw[arc] 
				(s0) edge node[above] {\Wait} node[below] {$5$} (s1) 
				(s1) edge node[sloped,below] {\DriveBus} node[sloped,above] {$\ui{6,10}$} (s2) 
				(s2) edge node[sloped,below] {\SmallClean} node[sloped,above] {$6$}(s0)		{};
		\end{tikzpicture}
	}
	\caption{Examples of Dual-Priced Modal Transition Systems with Time Durations.}
\end{figure}

The system $\Spec_1$ is another specification, 
a \emph{refinement} of $\Spec$, where 
we additionally specify that the driver 
must do a \SmallClean after each \DriveBus.
Note that the \Wait interval has been narrowed. 
The system $\Impl_1$ is an implementation of
$\Spec_1$ (and actually also of $\Spec$) 
where all controllable time intervals have
already been fully resolved to their final single values: 
the driver must \Wait for 5
minutes and do the \SmallClean for 6 minutes. Note that uncontrollable
intervals remain unresolved in the implementations and the time is
chosen by the environment each time the action is
performed. This reflects  the inherent uncontrollable uncertainty of the
timing, e.g.~of a traffic.
\begin{figure}[ht]
\centering
	\begin{tikzpicture}[node distance=100pt]
		\node[state,minimum size=20pt,initial] (s0) {$s$};
		\node[state,minimum size=20pt,right=of s0] (s1) {};
		\node[state,minimum size=20pt,right=of s1] (s2) {};
		\node[state,minimum size=20pt,below=of s2] (s3) {};
		\node[state,minimum size=20pt,below=of s1] (s4) {};
		\node[state,minimum size=20pt,below=of s0] (s5) {$t$};
		% \node[below=0.4 of s4] {$ \Phi(t)= (\BigClean,s)\ \vee (\SkipClean,s)\ \vee (\SmallClean,s)$};
		
		\draw[arc] 
			(s0) edge node[above] {\Wait} node[below] {$\ci{1,5}$} (s1) 
			(s1) edge node[above] {\DriveBus} node[below] {$\ui{6,10}$} (s2) 
			(s2) edge node[sloped,above] {\SmallClean} node[sloped,below] {$6$} (s3) 
			(s3) edge node[above] {\Wait} node[below] {$\ci{1,5}$} (s4) 
			(s4) edge node[above] {\DriveBus} node[below] {$\ui{6,10}$} (s5) 
			(s5) edge[dashed,bend left=80] node[sloped,above,yshift=-3] {\SmallClean} node[sloped,below,yshift=2] {$\ui{4,5}$} (s0) 
			(s5) edge[bend right=70] node[sloped,above,yshift=-1] {\BigClean} node[sloped,below,yshift=2] {$\ci{20,30}$} (s0) 
			% (s5) edge[dashed] node[sloped,above,yshift=-3] {\SkipClean} node[sloped,below,yshift=2] {0} (s0) 
			{};
	\end{tikzpicture}
	\caption{Specification $\Spec_2$.\label{fig:spec3}}
\end{figure}
\begin{figure}[ht]
\centering
	\begin{tikzpicture}[node distance=100pt]
		\node[state,minimum size=20pt,initial] (s0) {$s$};
		\node[state,minimum size=20pt,right=of s0] (s1) {};
		\node[state,minimum size=20pt,right=of s1] (s2) {};
		\node[state,minimum size=20pt,below=of s2] (s3) {};
		\node[state,minimum size=20pt,below=of s1] (s4) {};
		\node[state,minimum size=20pt,below=of s0] (s5) {$t$};
		% \node[below=0.4 of s4] {$ \Phi(t)= (\BigClean,s)\ \vee (\SkipClean,s)\ \vee (\SmallClean,s)$};
		
		\draw[arc] 
			(s0) edge node[above] {\Wait} node[below] {$1$} (s1) 
			(s1) edge node[above] {\DriveBus} node[below] {$\ui{6,10}$} (s2) 
			(s2) edge node[sloped,above] {\SmallClean} node[sloped,below] {$6$} (s3) 
			(s3) edge node[above] {\Wait} node[below] {$1$} (s4) 
			(s4) edge node[above] {\DriveBus} node[below] {$\ui{6,10}$} (s5) 
			% (s5) edge[dashed,bend left=80] node[sloped,above,yshift=-3] {\SmallClean} node[sloped,below,yshift=2] {$\ui{4,5}$} (s0) 
			(s5) edge node[sloped,above,yshift=-1] {\BigClean} node[sloped,below,yshift=2] {$30$} (s0) 
			% (s5) edge[dashed] node[sloped,above,yshift=-3] {\SkipClean} node[sloped,below,yshift=2] {0} (s0) 
			{};
	\end{tikzpicture}
	\caption{Implementation $\Impl_2$.\label{fig:impl2}}
\end{figure}

The system $\Spec_2$ is yet another specification and again a 
refinement of $\Spec$, where the
driver can always do a \BigClean in $t$ and 
possibly there is also an~alternative
allowed here of a \SmallClean. Notice that both \SmallClean intervals have been
restricted and changed to uncontrollable. This means that we give up the
control over the duration of this action and if this transition is implemented,
its duration will be every time chosen nondeterministically in that range.
Finally, $\Impl_2$ is then an implementation of $\Spec_2$ and $\Spec$.





\begin{figure}[!t]
	\centering
	\begin{tikzpicture}[node distance=.2]
		\node[anchor=north west,fill=lightlightblue] at (0,0) (run-cost){ 
			\begin{tabular}{l|r}
				$a\in\Sigma$  	&\; $r(a)$	\\ \hline
				\Wait 			& 8\\
				\DriveBus 		& 10\\
				\SmallClean 	& 6\\
				\BigClean 		& 7\\
				\SkipClean 		& 0
			\end{tabular} 
		};
		\node[right=of run-cost.north east,anchor=north west, fill=lightlightblue] (invest-cost){
			\begin{tabular}{l|r}
				$h\in H$  		&\; $\hc(h)$\\ \hline
				\VacuumCleaner  & 100\\
				\Sponge 		& 5
			\end{tabular}
		};

		\node[above=of run-cost.north west,anchor=south west, fill=lightlightblue] (H-set) {$H=\{\VacuumCleaner,\Sponge\}$};
		\node[below=of run-cost.south west,anchor=north west, fill=lightlightblue] (H-req) {
				$ 
					\Psi(a) = \begin{cases}
									\VacuumCleaner & \text{if } a = \BigClean \\
									\Sponge \vee \VacuumCleaner & \text{if } a= \SmallClean \\
									true & \text{otherwise}
								\end{cases}
				$
		};
	\end{tikzpicture}
\caption{Price Scheme.\label{fig:cost}}
\end{figure}	

Furthermore, we develop a way to model cost of resources. Each action is
assigned a \emph{running price} it costs per time unit, e.g.~\DriveBus costs 10
each time unit it is being performed as it can be  seen in the left table of
Figure~\ref{fig:cost}. In addition, in order to perform an action, some
hardware may be needed, e.g.~a~\VacuumCleaner for the \BigClean and its price
is 100 as can be seen on the right. This \emph{investment price} is paid once
only.

Let us now consider the problem of finding an optimal implementation, so that we
spend the least possible amount of money (e.g.~the pay to the driver) per time
unit while conforming to the specification $\Spec$. 
We call this problem \emph{the
cheapest implementation problem}.
The optimal implementation is to buy a vacuum cleaner if one can afford an
investment of 100 and do the \BigClean every time as long as possible and \Wait
as shortly as possible. (Note that \BigClean is more costly per time unit than
\SmallClean but lasts longer.) This is precisely implemented in $\Impl_2$ and
the (worst-case) average cost per time unit is $\approx7.97$. 
If one cannot afford the
vacuum cleaner but only a sponge, the optimal worst case long run average is
then a bit worse and is implemented by doing the \SmallClean as long as
possible and \Wait now as \emph{long} as possible. This is depicted in $\Impl_1$
and the respective average cost per time unit is $\approx8.10$. 
%For further details, see Example~\ref{ex:prices}.


\vspace{\smallskipamount}

The most related work is~\cite{DBLP:conf/emsoft/ChakrabartiAHS03} where prices are introduced into a class of interface theories and long-run average objectives are discussed. Our work omits the issue of distinguishing input and output actions. Nevertheless, compared to~\cite{DBLP:conf/emsoft/ChakrabartiAHS03}, this paper deals with the time durations, the one-shot hardware investment and, most importantly, refinement of specifications. Further, timed automata have also been extended with prices~\cite{DBLP:conf/fmco/BehrmannLR04} and the long-run average reward has been computed in~\cite{DBLP:journals/fmsd/BouyerBL08}. However, priced timed automata lack the hardware and any notion of refinement, too.

The paper is organized as follows. We introduce the MTS with the 
time durations in Section~\ref{bmts} and 
the dual-price scheme 
%together with the problem of the cheapest implementation 
in Section~\ref{prices}. Section~\ref{complexity} presents the main results on the complexity of the cheapest implementation problem. 
First, we state the complexity of this problem in general and in an important special case and prove the hardness part. The algorithms proving the complexity upper bounds are presented only after introducing an extension of {mean payoff games with time durations}. These are needed to establish the results but are also interesting on their own as discussed in~Section~\ref{mpgd}.
 We conclude and give some more account on related and future 
 work in Section~\ref{concl}.


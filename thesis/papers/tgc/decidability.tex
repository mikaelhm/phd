%!TEX root = main.tex
%\section{Decidability Issues}

As stated in the introduction, Petri nets are a useful model since (1) they 
model infinite state systems and (2) several relevant properties of transition
systems are decidable. The following proposition whose proof is an % slight
adaption of the one in~\cite{haddad-et-al-2013} establishes that 
one can decide both consuming properties. 

Before establishing the proof of Proposition~\ref{prop:decide:consuming},
we recall some results about Petri nets that we apply here.
Let $E \subseteq \N^k$, $E$ is a \emph{linear set}
if there exists a finite set of vectors of $\N^k$ $\{v_0,\ldots,v_n\}$
such that $E=\{v_0 + \sum_{1 \leq i \leq n} \lambda_i v_i\mid \forall i\ \lambda_i \in \N\}$.
A \emph{semi-linear set}~\cite{gs66} is a finite union of linear sets; a representation of it is given
by the family of finite sets of vectors defining the corresponding linear sets.
Semi-linear sets are \emph{effectively} closed w.r.t. union, intersection and complementation.
This means that one can compute a representation of the union, intersection and complementation
starting from a representation of the original semi-linear sets. 
$E$ is an \emph{upward closed set} if $\forall v \in E.\ v'\geq v \Rightarrow v' \in E$.
An upward closed set has a finite set of minimal vectors denoted $\min(E)$. 
An upward closed set is a semi-linear set which has a representation that can be derived
from the equation $E=\min(E)+\N^k$ if $\min(E)$ is computable.

Given a Petri net $\mathcal N$  and a marking $m$, the \emph{reachability}
problem consists in deciding whether $m$ is reachable from $m_0$
in $\mathcal N$. This problem is decidable~\cite{ReachaPbMayer}
but none of the associated algorithms are primitive recursive. 
Furthermore this procedure can be adapted to semi-linear sets
when markings are identified to vectors of $\N^{|P|}$.
% Based on reachability analysis, the authors of~\cite{dj89} design an algorithm  
% that decides whether a marking $m$ is a \emph{home state}, i.e. $m$ is reachable
% from any reachable
% marking. A more general problem
% is in fact decidable: given a subset of places $P'$ and a (sub)marking $m$ on
% this subset, is it possible from any reachable marking to reach
% a marking that coincides on $P'$ with $m$?

In~\cite{rackoff78}, the \emph{coverability} is shown 
to be EXPSPACE-complete. The coverability problem
consists in determining, given a net and a target marking, whether one can reach a marking
greater or equal than the target.  
In~\cite{VJ84} given a Petri net, several procedures have been designed to compute the minimal
set of markings of several interesting upward closed sets. In particular,
given an upward closed set $\transition{Target}$, by a backward analysis one can compute
the (representation of) upward closed set from which $\transition{Target}$ is reachable.
Using the results of~\cite{rackoff78}, this algorithm performs in EXPSPACE.

While in Petri nets, strong fairness is undecidable~\cite{carst87}, 
observationally weak fairness is decidable and more generally, 
the existence of an infinite sequence fulfilling a formula of the
following fragment of LTL is decidable~\cite{Jancar90}. The literals are (1) comparisons
between places markings and values, (2) transition firings and (3)  their negations. Formulas
are inductively defined as literals, conjunction or disjunction of formulas and $GF \varphi$
where $GF$ is the infinitely often operator and $\varphi$ is a formula. 


\begin{proposition}
\label{prop:decide:consuming}
Let $\pnN$ be a \MAIOPN and let $B \subseteq \chan$ be a subset of its channels. 
The satisfaction by $\pnN$ of
the message consuming and the necessarily message consuming properties w.r.t.\ $B$
are decidable.
\end{proposition}
\begin{proof}
	{\bf Consuming property.} 

\noindent
Let $a \in B$ and $E_a$ be the upward closed set of markings defined by:
\[
	E_a=\{m\mid \exists t \in \Tmust \mbox{ with }\lambda(t)=\inact{a} \mbox{ and } m\geq W^-(t)\}\; ,
\]   
$E_a$ is the set of markings from which (in all refinements) one can immediately consume some message~$a$.
Let $F_a$ be the upward closed set of markings defined by:\\ 
\[
	F_a=\{m\mid \exists m'\in E_a\ \exists \sigma \in (\Tmust)^*.\ \lambda(\sigma) \in (\Sigmat\setminus \inset)^* \wedge m \xrightarrow{\sigma} m'\}\;,
\]
$F_a$ is the set of markings from which one can later (in all refinements without the help of the environment)
consume some message $a$. One computes $F_a$ by backward analysis. Let $G$ be defined 
by: 
\[
	G=\{m \mid \exists a \in B. \ m(a)>0 \wedge m \notin F_a\}.
\]
$G$ is a semi-linear set corresponding to the markings
from which some message $a\in B$ will never be consumed. 
Then  $\mathcal N$ is not $B$-consuming
iff $G$ is reachable.


\smallskip\noindent
{\bf Necessarily consuming property.}

\noindent First one checks whether $\mathcal N$ is $B$-consuming,
a necessary condition for being necessarily $B$-consuming. 
If $\mathcal N$ is $B$-consuming, then one checks whether
for some $a \in B$, there exists a reachable marking $m$ fulfilling $m(a)>0$ from which an infinite
weakly fair run is fireable without occurrence of transitions labelled by $\inact{a}$. To perform
this test, one builds a net $\mathcal N'$ as follows.
\begin{itemize}
%
	\item $P'=P\uplus \{run\}$
%
	\item $T'=T\uplus \{stop\}$ 
%
	\item $\forall p \in P\ \forall t \in T\ W'^-(p,t)=W^-(p,t), W'^+(p,t)=W^+(p,t)$, $m'^0(p)=m^0(p)$
%
	\item $W'^-(run,stop)=1,$ $W'^+(run,stop)=0$, $m'^0(run)=1$
%
	\item $W'^-(a,stop)=W'^+(a,stop)=1$
%
	\item $\forall p \in P\setminus \{a\}\  W'^-(p,stop)=W'^+(p,stop)=0$
%
	\item $\forall t \in T\ \mbox{ such that }\lambda(t)=\inact{a}\ W'^-(run,t)=W'^+(run,t)=1$
%
	\item $\forall t \in T\ \mbox{ such that }\lambda(t)\neq\inact{a}\ W'^-(run,t)=W'^+(run,t)=0$
%
\end{itemize}
$\mathcal N'$ behaves as $\mathcal N$ as long as $stop$ is not fired.
Transition $stop$ can be fired only if $m(a)>0$.
When $stop$ is fired only transitions not labelled by $\inact{a}$ are fireable.
Then one checks whether there exists an infinite observationally weakly fair run 
that fulfils formula $GF run=0 \wedge GF \neg \mbox{fireable}(\inact{a})$
in $\mathcal N'$. Since $\mbox{fireable}(\inact{a})$ is expressible
by a boolean combination of comparisons of place markings with constants,
the formula belongs to the logic of~\cite{Jancar90}.
The first term witnesses the firing of $stop$
and the second term ensures that the sequence is also weakly
fair in $\mathcal N$. Then $\mathcal N$ is necessarily $B$-consuming 
iff there is no such run. 

\end{proof}

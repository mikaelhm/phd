Modal transition systems are an elegant way to formalise the design process of a system through refinement
and composition. Here we propose to adapt this methodology to asynchronous composition via Petri nets. 
The Petri nets that we consider have distinguished labels for inputs, outputs, internal communications and silent actions and  ``must'' and ``may''
modalities for transitions. The input/output labels show the interaction capabilities of a net to the outside used to build
larger nets by asynchronous composition via communication channels. The modalities express constraints for Petri net refinement
taking into account observational abstraction from silent transitions.
Modal I/O-Petri nets are equipped with a modal transition system semantics.
We show that refinement  is preserved by asynchronous composition and by hiding of communication channels.
We study compatibility properties which express communication
requirements for composed systems and we show that these properties are decidable,
they are preserved in larger contexts and also by modal refinement.
On this basis we propose a methodology for the specification of distributed systems
in terms of modal I/O-Petri nets which supports incremental design, encapsulation of components, stepwise refinement and independent implementability.

%!TEX root = ./main.tex

\subsection{Semantics: Modal Asynchronous I/O-Transition Systems}
\label{sec:semantics-tgc}

We extend the transition system semantics of modal Petri nets defined in Sect.~\ref{subsec:basics-modal} to \MAIOPNs.
For this purpose we introduce \emph{modal asynchronous I/O-transition system} (\MAIOTS) $\S = (\chan, \Sigma,Q,q^0,\may{},\must{},\val)$
which are modal transition systems such that $\chan$ is a finite set of channels, $\ioalpha{}$ is an I/O-alphabet over $\chan$,
and $\val: Q  \goes{} \N^{\chan}$  is a channel valuation function which determines for each state $q \in Q$ how many messages are actually pending
on each channel $a \in C$. Instead of $\val(q)(a)$ we write $\val(q,a)$. We require that initially all channels are empty, i.e.\ $\val(q^0,a) = 0$ for all $a \in \chan$, that for each $a \in \chan$  putting $\outact{a}$ and consuming $\inact{a}$ messages from $a$ has the expected behaviour, and that the open input/output actions have no effect on a channel, i.e.\
\begin{itemize}
            \item[~] $q \goes{\outact{a}} q' \implies \val(q') = \val(q)[a\mathrm{++}]$\footnote{$\val(q)[a\mathrm{++}]$ ($\val(q)[a\mathrm{--}]$ resp.) denotes the update of $\val$ which increments (decrements)
the value of $a$ and leaves the values of all other channels unchanged.},
            \item[~] $q \goes{\inact{a}} q' \implies \val(q,a) > 0 
            \text{ and } \val(q') = \val(q)[a\mathrm{--}]$, and
            \item[~] for all $x \in (\inset \cup \outset), q \goes{x} q' \implies \val(q') = \val(q)$.
\end{itemize}



The semantics $\maiots(\pnN)$ of a modal asynchronous I/O--Petri net $\pnN$ is given by the transition system semantics of modal Petri nets such that the reachable
markings are the states. Additionally we define the associated channel valuation function $\val: Q  \goes{} \N^{\chan}$ such that
the valuation of a channel $a$ in a current state $m$ is given by the number of tokens on $a$
under the marking $m$, i.e.\ $\val(m,a) = m(a)$.
For instance, the semantics of the \textsf{CompressorInterface} in Sect.~\ref{sec:example} and of the \textsf{Controller} leads to
 infinite state transition systems;
the transition systems associated with the two compressor components
have two reachable states and the transitions between them correspond directly to their Petri net representations 
in Fig.~\ref{fig:ex-gifCompressor} and  Fig.~\ref{fig:ex-txtCompressor}.
%For instance, the transition systems associated with the \MAIOPNs $\pnN_1$ and $\pnN_2$ in Fig.~\ref{fig:ex-aiopn1} and~\ref{fig:ex-aiopn2}
% have two reachable states and the transitions between them correspond directly to their Petri net representations. 
%The situation is different for the \MAIOPN $\pnN_3$ in Fig.~\ref{fig:ex-aiopn}. It has infinitly many reachable markings
%and hence its associated \MAIOTS has infinitely many reachable states. 
%Fig.~\ref{fig:ex-associatedMAIOTS} shows an excerpt of it. 
%The states indicate the number of tokens in each place in the order $p_0, p_1, msg, p_2, p_3$. The initial state is underlined.
%
%
%\begin{figure}
%    \centering
%    %!TEX root = ../main.tex
\newcommand*{\vdist}{1.0cm}%
\begin{tikzpicture}[scale=0.75, every node/.style={transform shape}]
\node (q00) {\underline{01010}};
\node[right= \vdist of q00] (q01) {10010};
\node[right= \vdist of q01] (q02) {01110};
\node[right= \vdist of q02] (q03) {10110};
\node[right= \vdist of q03] (q04) {01210};
\node[right= \vdist of q04] (q05) {10210};
\node[right= \vdist of q05] (q06) {01310};
% \node[right= \vdist of q06] (q07) {10310};
\node[right= \vdist of q06] (q08) {};
% \node[right= \vdist of q08] (q09) {};

\node[above=3.0cm of q00] (q10) {01001};
\node[right= \vdist of q10] (q11) {10001};
\node[right= \vdist of q11] (q12) {01101};
\node[right= \vdist of q12] (q13) {10101};
\node[right= \vdist of q13] (q14) {01201};
\node[right= \vdist of q14] (q15) {10201};
\node[right= \vdist of q15] (q16) {01301};
% \node[right= \vdist of q16] (q17) {10301};
\node[right= \vdist of q16] (q18) {};
\node[right= \vdist of q18] (q19) {};


\draw[arc] (q00) -- (q01) node[midway, below]{$in?$};
\draw[arc] (q01) -- (q02) node[midway, below]{$\outact{msg}$};
\draw[arc] (q02) -- (q03) node[midway, below]{$in?$};
\draw[arc] (q03) -- (q04) node[midway, below]{$\outact{msg}$};
\draw[arc] (q04) -- (q05) node[midway, below]{$in?$};
\draw[arc] (q05) -- (q06) node[midway, below]{$\outact{msg}$};
% \draw[arc] (q06) -- (q07) node[midway, above]{$in?$};
\draw[arc,dotted] (q06) -- (q08);
% \draw[dotted,thick] (q08) -- (q09){};

\draw[arc] (q10) -- (q11) node[midway, above]{$in?$};
\draw[arc] (q11) -- (q12) node[midway, above]{$\outact{msg}$};
\draw[arc] (q12) -- (q13) node[midway, above]{$in?$};
\draw[arc] (q13) -- (q14) node[midway, above]{$\outact{msg}$};
\draw[arc] (q14) -- (q15) node[midway, above]{$in?$};
\draw[arc] (q15) -- (q16) node[midway, above]{$\outact{msg}$};
% \draw[arc] (q16) -- (q17) node[midway, above]{$in?$};
\draw[arc,dotted] (q16) -- (q18);
% \draw[dotted,thick] (q18) -- (q19){};

\draw[arc] (q10) -- (q00) node[pos=0.11, left]{$out!$};
\draw[arc] (q11) -- (q01) node[pos=0.11, left]{$out!$};
\draw[arc] (q12) -- (q02) node[pos=0.11, left]{$out!$};
\draw[arc] (q13) -- (q03) node[pos=0.11, left]{$out!$};
\draw[arc] (q14) -- (q04) node[pos=0.11, left]{$out!$};
\draw[arc] (q15) -- (q05) node[pos=0.11, left]{$out!$};
\draw[arc] (q16) -- (q06) node[pos=0.11, left]{$out!$};
% \draw[arc] (q17) -- (q07) node[pos=0.11, left]{$out!$};
\draw[arc,dotted] (q18) -- (q08) node[midway, left]{};

\draw[arc] (q02) -- (q10) node[pos=0.19, above, yshift=1mm]{$\inact{msg}$};
\draw[arc] (q03) -- (q11) node[pos=0.19, above, yshift=1mm]{$\inact{msg}$};
\draw[arc] (q04) -- (q12) node[pos=0.19, above, yshift=1mm]{$\inact{msg}$};
\draw[arc] (q05) -- (q13) node[pos=0.19, above, yshift=1mm]{$\inact{msg}$};
\draw[arc] (q06) -- (q14) node[pos=0.19, above, yshift=1mm]{$\inact{msg}$};
% \draw[arc] (q07) -- (q15) node[pos=0.19, above]{$\inact{msg}$};
\draw[arc,dotted] (q08) -- (q15) node[pos=0.15, above]{};

\end{tikzpicture}
%    \caption{Part of the associated \MAIOTS for $\pnN_3$ in Fig.~\ref{fig:ex-aiopn}.}
%\label{fig:ex-associatedMAIOTS}
%\end{figure}

The \emph{asynchronous composition} $\S \compos \T$ of two \MAIOTSs $\S$ and $\T$ works as for asynchronous I/O--transition systems in~\cite{haddad-et-al-2013} taking additionally care that\linebreak must-transitions of $\S$ and $\T$ induce must-transitions of  $\S \compos \T$.
The composition introduces new channels $\chan_{\S\T} = \Sigma_\S \cap \Sigma_\T$ for shared input/output labels.
Every transition with a shared output label $a$ becomes a transition with the communication label $\outact{a}$,
and similarly transitions with input labels $a$ become transitions with label $\inact{a}$.
The state space of the composition is (the reachable part of) the Cartesian product of the underlying state spaces of $\S$ and $\T$ together with the set $\N^{\chan_{\S\T}}$ of valuations for the new channels such that transitions labelled by  $\outact{a}$ and $\inact{a}$ have the expected effect on the new channels; for details see~\cite{haddad-et-al-2013}.
The asynchronous composition of two \MAIOTSs is commutative and also associative under appropriate syntactic restrictions on the underlying alphabets.

We also introduce a \emph{hiding operator} on \MAIOTSs that hides communication channels
and moves all corresponding communication labels to $\tau$.
Let $\S = \maiotstup{\S}$ be a \MAIOTS with I/O-alphabet \ioalpha{\S}, and let $H \subseteq \chan_\S$ be a subset of its channels. The \emph{channel hiding of $H$ in $\S$} is the \MAIOTS $\S \setminus H = (\chan, \Sigma,Q_\S,q^0_\S,\may{},\must{},\val)$, such that 
$\chan = \chan_\S \setminus H$, $\Sigma = \inset_\S \uplus \outset_\S \uplus \com$ with $\com = \{\inact{a},\outact{a} \mid a \in \chan\}$,
 $\val(q,a) = \val_\S(q,a)$  for all $q \in Q, a \in \chan$, and the may-transition relation is given by:
\begin{align*}
    \may{} \; = &\{(q,a,q') \mid a \in (\Sigmat \setminus \{\inact{a},\outact{a} \mid a \in H\}) \text{ and } q \may{a}_\S q' \} \\
              &\cup \{ (q,\tau ,q') \mid \exists a \in H \text{ such that either } q \may{\outact{a}}_\S q'  \text{ or }   q \may{\inact{a}}_\S q' \},
\end{align*}
    \begin{itemize}
        \item[~] $\may{} \; = \{(q,a,q') \mid a \in (\Sigmat \setminus \{\inact{a},\outact{a} \mid a \in H\}) \text{ and } q \may{a}_\S q' \} \; \cup $\\
              $~~\qquad \{ (q,\tau ,q') \mid \exists a \in H \text{ such that either } q \may{\outact{a}}_\S q'  \text{ or }   q \may{\inact{a}}_\S q' \}$,
    \end{itemize}
The must-transition relation $\must{}$ is defined analogously.

For the results developed in the next sections it is important that our transition system semantics is compositional and
compatible with hiding as stated in the next theorem. 

\begin{theorem}\label{thm:comp-hide-semantics}
  Let $\pnN$ and $\M$ be two composable \MAIOPNs  and let $H \subseteq \chan_\pnN$
be a subset of the channels of $\pnN$. The following holds:
 \begin{enumerate}
\item\label{item:comp-semantics} 
$\maiots(\pnN \compospn \M) = \maiots(\pnN) \compos \maiots(\M)$ (up to isomorphic state spaces),
\item\label{item:hide-semantics}
$\maiots(\pnN \setminus_{pn} H) = \maiots(\pnN) \setminus H$.
 \end{enumerate}
\end{theorem}
\begin{proof}
	The proof of part \ref{thm:comp-hide-semantics}.\ref{item:comp-semantics} is ommited as it is almost identical to the proof in \cite{TechReportVersion}.

\paragraph{Proof of \ref{thm:comp-hide-semantics}.\ref{item:hide-semantics}}
Let $\maiots(\pnN \hidepn H) = A = \maiotstup{A}$ and $\maiots(\pnN) \hide H = B = \maiotstup{B}$. 
The following equalities follow directly from 
the definiton of channel hidding on \MAIOPNs, channel hidding on \MAIOTSs and the semantics of a \MAIOPN: 
% Def.~\ref{def:hiding-pn}, Def.~\ref{def:assoc-maiots} and Def.~\ref{def:hiding-ts}; 
$\chan_A = \chan_B$, $\Sigma_A = \Sigma_B$, $Q_A = Q_B$, $q^0_A = q^0_B$, and $\val_A = \val_B$. 
It remains to prove that the two transition relations coincide.

From 
the semantics of a \MAIOPN
% Def.~\ref{def:assoc-maiots} 
it follows that 
\[
    \may{}_A = \{(m,a,m') \mid a \in (\Sigma_A)_\tau \;.\; m \may{a}_{(\pnN \hidepn H)} m'\}\;,
\]
where $\may{}_{(\pnN \hidepn H)}$ is the may-transition relation of $\pnN \hidepn H$.

From the
definition of channel hiding on \MAIOPNs
%Def.~\ref{def:hiding-pn} 
we see that communication labels on hidden channels are renamed to $\tau$. Thus, 
\begin{align*}
    \may{}_B = &\{(m,\tau,m') \mid \exists a \in H \;.\; m \may{\inact{a}}_{\pnN} m' ~\vee~m \may{\outact{a}}_{\pnN} m' \}\\
    &\cup  \{(m,a,m') \mid a \in (\Sigmat \setminus \{\inact{a},\outact{a} \mid a \in H\}) \;.\; m \may{a}_{\pnN} m'\} \;,
\end{align*}

where $\Sigma$ is the I/O-alphabet of $\pnN$ and $\may{}_{\pnN}$ is the may-transition relation of $\pnN$.
It then follows from 
the definition of channel hidding on \MAIOTS
%Def.~\ref{def:hiding-ts} 
that $\may{}_A =~\may{}_B$. Similarly one can prove that $\must{}_A =~\must{}_B$.
\end{proof}
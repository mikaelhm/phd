%!TEX root = main.tex

\section{Modal Asynchronous I/O-Petri Nets}\label{sec:maiots}

In this section we formalise the syntax of \MAIOPNs and we define their transition system semantics. 
First we recall some basic definitions for modal Petri nets and modal transition systems.

\subsection{Modal Petri Nets and Modal Transition Systems}\label{subsec:basics-modal}

In the following we consider labelled Petri nets such that transitions are equipped
with labels of an alphabet $\Sigma$ or with the symbol $\tau$ that models silent transitions.
We write $\Sigmat$ for $\Sigma \uplus \{\tau\}$.
We assume the reader to be familiar with the basic notions of labelled Petri nets consisting of a finite set
$P$ of places, a finite set $T$ of transitions, a set of arcs between places and transitions (transitions and places resp.),
formalised by incidence matrices $W^-$ and $W^+$,
an initial marking $m^0$ and a labelling function $\lambda : T \rightarrow \Sigmat$. 
In~\cite{EHH12} we have introduced \emph{modal Petri nets}
$\mpndef$ such that $\Tmust \subseteq T$  is a distinguished subset  of must-transitions.
We write $m \may{t} m'$ if a transition $t \in T$ is firable from a marking $m$ leading to a marking $m'$.
If $t \in \Tmust$ we write $m \must{t} m'$. 
If $\lambda (t) = a$ we write $m \may{a} m'$ and for $t \in \Tmust$ we write $m \must{a} m'$.
The notation is extended as usual to firing sequences $m \may{\sigma} m'$ with $\sigma \in  T^{*}$
and to $m \must{\sigma} m'$ with $\sigma \in  (\Tmust)^{*}$.
A marking $m$ is reachable if there exists a firing sequence $\sigma \in  T^{*}$ such that  $m^{0} \may{\sigma} m$.

\emph{Modal transition systems} have been introduced in~\cite{DBLP:conf/lics/LarsenT88}. We will use them
to provide semantics for modal Petri nets.
A modal transition system is a tuple $\S = (\Sigma, Q, q^0, \may{},\must{})$, such that
$\Sigma$ is a set of labels, $Q$ is a set of states, $q^0 \in Q$ is the initial state,
$\may{} \; \subseteq Q \times \Sigmat \times Q$ is a may--transition relation, and
$\must{} \; \subseteq \; \may{}$ is a must--transition relation.
We explicitely allow the state space $Q$ to be infinite which is needed to give interpretations
to modal Petri nets that model infinite state systems.

The semantics of a modal Petri net $\mpndef$ is given by the modal transition system
$\mts(\pnN) =  (\Sigma, Q, q^0, \may{},\must{})$  representing the reachability graph of the net by taking into account modalities:
\begin{itemize}
	\item $Q \subseteq \N^P$ is the set of reachable markings of \pnN,
	\item $\may{} \; = \{(m, a, m') \mid a \in \Sigmat \text{ and } m \may{a} m'\}$,
	\item $\must{} \; = \{(m, a, m') \mid a \in \Sigmat \text{ and } m \must{a} m'\}$, and
	\item $q^0 = m^0$.
\end{itemize}

In the sequel we will use the following notations for modal transition systems.
We write may-transitions as $q \may{a} q'$ for $(q,a,q') \in \;\may{}$ and similarly must-transitions as $q \must{a} q'$. % for $(q,a,q') \in \;\must{}$.
The notation is extended in the usual way to finite sequences  $\sigma \in \Sigmat^*$ by the notations
$q \may{\sigma} q'$ and $q \must{\sigma} q'$.
The set of reachable states of $\S$ is given by $\reach(\S) = \{q \mid \exists \sigma \in \Sigmat^* \;.\; q^0 \may{\sigma} q\}$.
%A \emph{trace} of $\S$ starting in a state $q_1 \in Q$ is a finite or infinite sequence $\rho =q_1  \may{a_1} q_2 \may{a_2} q_3 \may{a_3} \cdots$,
%such that $a_i \in \Sigmat$ and $q_i \in Q$ for all $i$.  
%A \emph{must--trace} is a trace with must--transitions only.
%For $a \in \Sigmat$ we write $a \in \rho$, if there exists $a_i$ in the trace $\rho$ such that $a_i = a$.
%% , and $\sharp_\rho(a)$ denotes the (possibly infinite) number of occurrences of $a$ in $\rho$. 
%For $q \in Q$ we write $q \in \rho$, if there exists $q_i$ in the trace $\rho$ such that $q_i = q$.
%For $\sigma=a_1 a_2 \cdots a_n \in \Sigmat^*$ and $q,q'\in Q$ we write $q \may{\sigma} q'$ if there exists a (finite) trace
%$q \may{a_1}  \cdots  \may{a_n} q'$ and likewise for must--transitions. 
%The set of reachable states of $\S$ is given by $\reach(\S) = \{q \mid \exists \sigma \in \Sigmat^* \;.\; q^0 \may{\sigma} q\}$.
For a sequence $\sigma \in \Sigmat^*$ the observable projection $\obsa(\sigma) \in \Sigma^*$ is obtained from $\sigma$ by removing
all occurrences of $\tau$.   
Let $a \in \Sigma$ be a visible action and $q,q' \in Q$.
We write $q \mayw{a} q'$ if $q \may{\sigma} q'$ with  $\obsa(\sigma) = a$ and we call $q \mayw{a} q'$ a weak may-transition.
Similarly we write $q \mustw{a} q'$ if all transitions are must-transitions and call $q \mustw{a} q'$ a weak must-transition.
The notation is extended as expected to finite sequences $\sigma \in \Sigma^*$ of visible actions
by the notations $q \mayw{\sigma} q'$ and $q \mustw{\sigma} q'$.
In particular, $q \mayw{\epsilon} q'$ means that there is a (possibly empty) sequence of silent may-transitions from $q$ to $q'$
and  similarly $q\mustw{\epsilon}q'$ expresses a finite sequence of silent must-transitions.


%%%%%%%%%%%%%%%%%%%%%%%%%%%%%%%%%%%%%%%%%%5

\subsection{Modal Asynchronous I/O--Petri Nets, Composition and Hiding}\label{sec:MAIOPN}

In this paper we consider systems which may be open for communication with other systems and may be composed to larger systems.
The open actions are modelled by input labels (for the reception of messages from the environment) and output labels (for sending messages to the environment)
while communication inside an asynchronously composed system takes place by removing or putting messages to distinguished communication channels.
Given a finite set $\chan$ of channels, an \emph{I/O-alphabet over $\chan$} is the disjoint union $\ioalpha{}$ of pairwise disjoint sets of input labels, output labels and communication labels, such that $\com = \{\inact{a},\outact{a} \mid a \in \chan\}$.
For each channel $a \in \chan$, the label $\inact{a}$ represents consumption of a message from $a$ and $\outact{a}$ represents putting a message on $a$.
A \emph{modal asynchronous I/O-Petri net} (\MAIOPN) $\pnN = \mapntup{}$ is a modal Petri net such that
$\chan \subseteq P$ is a set of channel places which are initially empty, $\ioalpha{}$ is an I/O-alphabet over $\chan$, and
for all $a \in \chan$ and $t \in T$, there exists an (unweighted) arc from $a$ to $t$ iff $\lambda(t) = \inact{a}$ and 
 there exists an  (unweighted) arc from $t$ to $a$ iff $\lambda(t) = \outact{a}$. 

The \emph{asynchronous composition} of \MAIOPNs works as for asynchronous\linebreak I/O--Petri nets considered in~\cite{haddad-et-al-2013}
by introducing new channel places and appropropriate transitions for shared input/output labels.
%Every transition with shared output label $a$ becomes a transition with the communication label $\outact{a}$
%that produces a token on the (new) channel place $a$ and, similarly,  any transition with shared input label $a$ becomes a transition with the communication label $\inact{a}$
%that consumes a token from the (new) channel place $a$.
The non-shared  input and output labels remain open in the composition. 
Moreover, we require that the must-transitions of the composition are the union of the must-transitions of the single components. 
The asynchronous composition of two \MAIOPNs $\pnN$ and $\M$ is denoted by $\pnN \compospn \M$.
It is commutative and also associative under appropriate syntactic restrictions on the underlying alphabets.
An example of a composition of three \MAIOPNs is given in Fig.~\ref{fig:ex-async-comp}.
%Fig.~\ref{fig:ex-maiopns} shows three \MAIOPNs.
%The nets $\pnN_1$ and $\pnN_2$ model primitive
%components (without channels) and the $\pnN_3$ is obtained by their composition;
%it shows the new channel place $msg$.
%
%
%\begin{figure}%
%\centering
%\subfloat[$\pnN_1$.]{\label{fig:ex-aiopn1}\input{papers/tgc/figures/example-maiopn-before-comp.tex} }
%\subfloat[$\pnN_2$.]{\label{fig:ex-aiopn2}\begin{tikzpicture}[scale=0.675, every node/.style={transform shape}]



\node[place,tokens=1,label=above:$p_2$] at (6,0) (p2) {};
\node[transition,rotate=90,label={[yshift=13,xshift=0]180:$msg?$}] at (5,-1.5) (tmsg_in) {};
\node[place,tokens=0,label=below:$p_3$] at (6,-3) (p3) {};
\node[transition,rotate=90,label={[yshift=-2,xshift=7]-00:$out!$}] at (7,-1.5) (tb_out) {};

% \node[place,label=above:$msg$] at (3,-1.5) (pa) {};



% \draw[arc] (tmsg_out) -- (pa) {};
% \draw[arc] (pa) -- (tmsg_in) {};

\draw[arc] (p2) to[bend right] (tmsg_in) {};
\draw[arc] (tmsg_in) to[bend right] (p3) {};
\draw[arc] (p3) to[bend right] (tb_out) {};
\draw[arc] (tb_out) to[bend right] (p2) {};
% \node[above of=p0] {$\mathcal M_1$};
% \node[above of=p2] {$\mathcal M_2$};

% \begin{pgfonlayer}{background}
% \node [fill=white,inner sep=8mm,fit=(p0) (tmsg_out) (p1) (tb_in)] {};
% \node [fill=white,inner sep=8mm,fit=(p2) (tb_out) (p3) (tmsg_in)] {};
% \end{pgfonlayer}
\end{tikzpicture} }
%\subfloat[$\pnN_3$.]{\label{fig:ex-aiopn}%!TEX root = ../main.tex

\begin{tikzpicture}[scale=0.675, every node/.style={transform shape}]
\node[place,tokens=0,label=above:$p_0$] at (0,0) (p0) {};
\node[transition,rotate=90,label={[yshift=7,xshift=1]0:$\outact{msg}$}] at (1,-1.5) (tmsg_out) {};
\node[place,tokens=1,label=below:$p_1$] at (0,-3) (p1) {};
\node[transition,rotate=90,label={[yshift=4,xshift=-8]180:$in?$}] at (-1,-1.5) (ta_in) {};


\node[place,tokens=1,label=above:$p_2$] at (5,0) (p2) {};
\node[transition,rotate=90,label={[yshift=13,xshift=0]180:$\inact{msg}$}] at (4,-1.5) (tmsg_in) {};
\node[place,tokens=0,label=below:$p_3$] at (5,-3) (p3) {};
\node[transition,rotate=90,label={[yshift=-2,xshift=7]-00:$out!$}] at (6,-1.5) (tb_out) {};

\node[place,label=below:$msg$] at (2.5,-1.5) (pa) {};

\draw[arc] (p0) to[bend left] (tmsg_out) {};
\draw[arc] (tmsg_out) to[bend left] (p1) {};
\draw[arc] (p1) to[bend left] (ta_in) {};
\draw[arc] (ta_in) to[bend left] (p0) {};

\draw[arc] (tmsg_out) -- (pa) {};
\draw[arc] (pa) -- (tmsg_in) {};

\draw[arc] (p2) to[bend right] (tmsg_in) {};
\draw[arc] (tmsg_in) to[bend right] (p3) {};
\draw[arc] (p3) to[bend right] (tb_out) {};
\draw[arc] (tb_out) to[bend right] (p2) {};
% \node[above of=p0] {$\mathcal M_1$};
% \node[above of=p2] {$\mathcal M_2$};

% \begin{pgfonlayer}{background}
% \node [fill=white,inner sep=8mm,fit=(p0) (tmsg_out) (p1) (tb_in)] {};
% \node [fill=white,inner sep=8mm,fit=(p2) (tb_out) (p3) (tmsg_in)] {};
% \end{pgfonlayer}
\end{tikzpicture} }
%\caption{Modal asynchronous I/O-Petri nets.}%
%\label{fig:ex-maiopns}%
%\end{figure}

We introduce a \emph{hiding operator} on \MAIOPNs which allows us to hide communication channels.
In particular, it allows us to compute the black-box behaviour of an assembly when all channels are hidden.
    Let $\pnN$ be a \MAIOPN with I/O-alphabet \ioalpha{\pnN}, and let $H \subseteq \chan_\pnN$ be a subset of its channels. The \emph{channel hiding of $H$ in $\pnN$} is the \MAIOPN $\pnN \hidepn H$ with channels $\chan = \chan_\pnN \setminus H$, with I/O-alphabet $\Sigma = \inset_\pnN \uplus \outset_\pnN \uplus \com$ such that $\com = \{\inact{a},\outact{a} \mid a \in \chan\}$, and with the labelling function:

 \begin{itemize}
         \item[~] $\lambda(t) = \begin{cases}
            \tau &\text{ if } \exists a \in H \;.\; \lambda_\pnN(t)= \inact{a} \text{ or } \lambda_\pnN(t)= \outact{a},\\
            \lambda_\pnN(a) & \text{ otherwise. }
        \end{cases}$
    \end{itemize}

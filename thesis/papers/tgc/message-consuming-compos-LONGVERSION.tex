%!TEX root = main.tex

\subsection{Composition of Message Consuming Systems}\label{subsec:message-consuming-compos}

In order to prove that message consuming properties are preserved when systems are put into larger
contexts we need the next two lemmas.
%In order to derive message consuming properties of asynchronously composed systems from
%their constituent parts we need the next two lemmas.
The first one shows that autonomous executions of constituent parts (not involving inputs)
can be lifted to executions of compositions.
This is the essence to get compositionality of the message consuming property~\ref{def:cp-tgc-tgc:bas-consuming}.


\begin{lemma}\label{lem:reuse-sequence-tgc}
    Let $\S = \maiotstup{\S}, \T = \maiotstup{\T}$ be two composable \MAIOTSs, and let $\S\compos\T = \maiotstup{}$. For all $(q_\S,q_\T,\vec{v}) \in Q$  and $\sigma \in (\Sigma_\S \setminus \inset_\S)^*$ it holds:
    \\\centerline{$
        q_\S \mustw{\sigma}_\S q'_\S \implies \exists \vec{v'}\ .\ (q_\S,q_\T,\vec{v}) \mustw{\bar\sigma} (q'_\S,q_\T,\vec{v'}),
    $}\\[2mm]
    with $\bar\sigma \in (\Sigma \setminus \inset)^*$ obtained from $\sigma$ by replacing any occurrence of a shared label $a \in \outset_\S \cap \inset_\T$ by the communication label $\outact{a}$.
\end{lemma}
The proof of this lemma is ommited as is almost identical to the proof given in \cite{TechReportVersion}.
% The next lemma is crucial to prove compositionality of the
% ``necessarily'' properties of type (c) in Def.~\ref{def:cp-tgc-tgc} and the communication stopping properties in Def.~\ref{def:rp}.
% It shows that projections of weakly fair runs are weakly fair runs again.
% This result can only be achieved in the asynchronous context.
%In a synchronous environment strongly fair runs would have to be considered.

The second lemma is crucial to prove compositionality of the
necessarily consuming property~\ref{def:cp-tgc-tgc:nec-consuming}.
It shows that projections of observationally weakly fair runs to constituent parts of a composition are again observationally weakly fair runs.
Formally, let $\rho$ be a trace of $\S\compos\T$ starting from a state $q = (q_\S,q_\T,\vec{v}) \in Q$. The \emph{projection} of $\rho$ to $\S$, denoted by $\rho|_\S$, is the sequence of transitions of $\S$, starting from $q_\S$, which have triggered corresponding transitions in $\rho$.


\begin{lemma}\label{lem:projection-run-tgc}
    Let $\S$, $\T$ be two composable \MAIOTSs and $\S\compos\T = \maiotstup{}$. Let $q = (q_\S,q_\T,\vec{v}) \in Q$ and $\rho \in \frun{\S\compos\T}(q)$. Then $\rho|_\S \in \frun{\S}(q_\S)$ is an observationally weakly fair run of $\S$.
\end{lemma}
The following proof is a generalisation of the proof of the corresponding lemma in \cite{TechReportVersion}.

\begin{proof}
    Let $\rho \in \frun{\S\compos\T}(q)$. First we show that $\rho|_\S$ is observationally weakly fair as well. If $\rho|_\S$ is finite we are done. 
If $\rho|_\S$ is infinite, then assume that  from a certain state of $\rho|_\S$ on a visible autonomous action $a \in (\Sigma_\S \setminus in_\S)$ is always enabled by  a weak must-transition. Due to the asynchronous composition
we get that also in $\rho$ from a certain state on the action triggered by $a$ (i.e.\ $a$ or $\outact{a}$) is always enabled  by a weak must-transition. Hence the action triggered by $a$ will be infinitely often executed in $\rho$ and therefore $a$ will be infinitely often executed in $\rho|_\S$.

 It remains to prove that $\rho|_\S$ is a run of $\S$. 
    There are two main cases. Either $\rho$ is finite or infinite. 
    Assume $\rho$ is finite with its last state being $q' = (q'_\S,q'_\T,\vec{v'})$. Clearly $q'_\S$ must be the last state of $\rho|_\S$. By definition of a run, $q'$ is a potential deadlock state of $\S\compos\T$. Then $q'_\S$ must be a potential deadlock state of $\S$, hence $\rho|_\S$ is a run of $\S$.
    
    Now assume that $\rho$ is infinite and observationally weakly fair. In this case there are two subcases. Either $\rho|_\S$ is finite or infinite. If $\rho|_\S$ is infinite, then $\rho|_\S$  is a run of $\S$ and we are done.
Assume $\rho|_\S$ is finite such that its last state is $q'_\S$, with $(q'_\S,q'_\T,\vec{v'}) \in \rho$. Since $\rho$ is observationally weakly fair we can prove that $q'_\S$ is a potential deadlock state of $\S$ as follows: Assume the contrary, that $q'_\S$ is not a potential deadlock state, i.e. there exists $a \in (\Sigma_\S \setminus \inset_\S)$ such that $q \mustw{a} \stateTGC$.
As $\rho$ is infinite we get that from $(q'_\S,q'_\T,\vec{v'}) \in \rho$ on $a$ (or $\outact{a}$) is  always enabled by a weak must-transition.
This is a contradiction, since $\rho$ is observationally weakly fair and $a$ ($\outact{a}$ resp.) does not occur infinitely often after $(q'_\S,q'_\T,\vec{v'}) \in \rho$ since $\rho|_\S$ is finite. Now, knowing that $q'_\S$ is a potential deadlock state of $\S$, we get that $\rho|_\S$ is a run of $\S$. 
\end{proof}


\begin{proposition}\label{prop:cp:compos-tgc}
   Let $\S$ and $\T$ be two composable \MAIOTSs such that $\chan_\S$ is the set of channels of $\S$. Let $B \subseteq \chan_\S$.
   If $\S$ is (necessarily) consuming w.r.t.\ $B$, then $\S \compos \T$  is (necessarily) consuming w.r.t.\ $B$.
\end{proposition}
\begin{proof}
	This proof follows the same shape as the proof of the corresponding proposition in \cite{TechReportVersion}
but using Lemma~\ref{lem:reuse-sequence-tgc} and Lemma~\ref{lem:projection-run-tgc}.


      Let $\S =  \maiotstup{\S}$, $\T =  \maiotstup{\T}$ and $\S \compos \T = \maiotstup{}$.
    Let $a \in B$ and $(q_\S,q_\T,\vec{v}) \in \reach(\S\compos\T)$ such that $\val((q_\S,q_\T,\vec{v}),a)>0$.
    \begin{enumerate}      
        \item Assume that $\S$ is $B$-consuming. Obviously $q_\S \in \reach(\S)$. 
Then there exist $q'_\S,q''_\S \in Q_\S$ and $\sigma \in (\Sigma \setminus \inset)^*$
such that $q_\S \mustw{\sigma}_\S q'_\S \must{\inact{a}}_\S q''_\S$.
As a direct consequence of Lem.~\ref{lem:reuse-sequence-tgc}, we get that there exist $\vec{v'}, \vec{v''}$ such that
$(q_\S,q_\T,\vec{v}) \mustw{\bar\sigma}  (q'_\S,q_\T,\vec{v'}) \must{\inact{a}}  (q''_\S,q_\T,\vec{v''})$.
Hence, $\S \compos \T$ is $B$-consuming.

        \item Assume that $\S$ is necessarily $B$-comsuming.  Let $\rho \in \frun{\S\compos\T}((q_\S,q_\T,\vec{v}))$
 be an observationally weakly fair run. 
        By Lem.~\ref{lem:projection-run-tgc} we get that $\rho|_\S$ is an observationally weakly fair run of $\S$. By assumption $\inact{a} \in \rho|_\S$, hence it follows that $\inact{a} \in \rho$. 
    \end{enumerate}
\end{proof}

Proposition~\ref{prop:cp:compos-tgc} leads to the desired modular verification result
which allows us to check consuming properties in an incremental manner:
To show that a composed system $\S \compos \T$ is (necessarily) message consuming
it is sufficient to know that both constituent parts $\S$ and $\T$ have this property
and to check that $\S \compos \T$ is (necessarily) message consuming w.r.t.\ the
new channels established for the communication between $\S$ and $\T$, i.e.\ that $\S$ and $\T$
are compatible.
%starting
%from primitive components and proving step by step that successively added new components are
%compatible to the already given system.

%Proposition~\ref{prop:cp:compos-tgc} leads to the desired modular verification result
%which allows us to check the (necessarily) consuming properties of \MAIOPNs in an incremental manner starting
%from primitive components without communication channels and adding successively new components and communication channels
%by asynchronous composition.

\begin{definition}[Compatibility]\label{def:compatibility}
Two composable \MAIOTSs $\S$ and $\T$ with shared actions  $\Sigma_\S \cap \Sigma_\T$ are \emph{(necessarily) message consuming compatible} if $\S \compos \T$  is (necessarily) message consuming w.r.t.\ $\Sigma_\pnN \cap \Sigma_\M$.
\end{definition}

\begin{theorem}[Incremental Design]\label{thm:inc-design-tgc}
Let $\S$ and $\T$ be (necessarily) message consuming compatible. 
If both $\S$ and $\T$ are (necessarily) message consuming, then $\S \compos \T$ is (necessarily) message consuming.
\end{theorem}

%\begin{theorem}[Incremental Design]\label{thm:inc-design-tgc}
%Two composable \MAIOTSs $\S$ and $\T$ with shared actions  $\Sigma_\S \cap \Sigma_\T$ are called \emph{(necessarily) message consuming compatible} if $\S \compos \T$  is (necessarily) message consuming w.r.t.\ $\Sigma_\pnN \cap \Sigma_\M$.
%If both $\S$ and $\T$ are (necessarily) message consuming and and if they are  (necessarily) message consuming compatible, then $\S \compos \T$ is (necessarily) message consuming.
%\end{theorem}

All results holds analogously for asynchronous I/O-Petri nets due to the compositional semantics of \MAIOPNs;
see Thm.~\ref{thm:comp-hide-semantics}.\ref{item:comp-semantics}
An application of incremental design has been discussed in Sect.~\ref{sec:example}. 


%\vspace{3mm}
%\textbf{From Rolf:}
%Something on Hiding here as well?\\
%No! The interesting case would be:\\
% if $(\S \setminus \chan_\S) \compos (\T \setminus \chan_\T)$  is necessarily message consuming, then
%$\S \compos \T$  is necessarily message consuming w.r.t.\ $\Sigma_\pnN \cap \Sigma_\M$.\\
%But this does (unfortunately) not hold! Would probably need strong fairness!
%!TEX root = main.tex
\section{Message Consuming Systems}\label{sec:mcs}

In this section we consider generic properties concerning the asynchronous communication via channels
inspired by the various channel properties studied in~\cite{haddad-et-al-2013}.
%for asynchronous I/O-Petri nets.
We focus on the \emph{message consuming} and the \emph{necessarily message consuming} properties
and generalise them to take into account modalities and observational abstraction w.r.t.\ silent transitions.
Our goal is that the properties scale up to larger contexts (to support incremental design)
and that they are preserved by modal refinement. Moreover we consider their decidability.
For the definitions %of the communication properties
we rely on the semantics of \MAIOPNs, i.e.\ on \MAIOTSs.

\subsection{Concepts and Definitions}\label{subsec:mcs}

%The basic idea behind the \emph{message consuming} property is that whenever a message has been put on a communication channel
%it must be possible to consume it, possibly after some delay. This requirement can be strengthened
%such that any message pending on a communication channel must eventually be consumed which is the idea behind
%\emph{necessarily message consuming}.
Let us first discuss the message consuming property~\ref{def:cp-tgc-tgc:bas-consuming}
of Def.~\ref{def:cp-tgc-tgc} for a \MAIOTS $\S$ and a subset $B$ of its channels.
It requires,
for each channel $a \in B$, that if in an arbitrary reachable state $q$ of $\S$ there is a message on $a$, then $\S$ must be able
to consume it, possibly after some delay caused by the execution of autonomous must-transitions.
All transitions that do not depend on the environement, i.e.\ are not related to input labels,
are considered to be autonomous. Our approach follows a ``pessimistic'' assumption taking into account arbitrary environments
that can let the system go where it wants and can also stop to provide inputs at any moment.
 It is important that we require must-transitions
since the consuming property should be preserved by modal refinement.
It is inspired by the notion of ``output compatibility'' studied for synchronously composed transition systems in~\cite{hennicker_knapp_2011}.

%Let us comment on the role of the environment for the formulation of this property.
%First, we consider arbitrary reachable states of $\S$.
%This means that we take into account the worst environment which can let $\S$ go everywhere by providing (non-deterministically)
%all inputs that $\S$ can accept.
%Then, at some point at which a message is available on channel $a$, the environment can stop
%to provide further inputs and waits whether $\S$ can \emph{autonomously} reach a state
%$q'$ by a sequence of must-transitions in which it  must be able to consume $a$, i.e.\ execute $\inact{a}$.
%To allow autonomous must-transitions before consumption

A central role when components run in parallel is played by fairness assumptions; see,e.g.,~\cite{berard-et-al-2001}.
First we must define what we mean by a run and then we will explain our fairness notion.
A run is a finite or infinite sequence of state transitions which satisfies a maximality condition.
In principle a run can only stop when a deadlock is reached. However we must be careful
since (1) we are dealing with open systems whose executions depend on the input from the environment,
(2) we must take into account that transitions with a may-modality can be skipped in a refinement,
and (3) we must be aware that also silent must-transitions without successive mandatory visible actions
can be omitted in a refinement; cf. Def.~\ref{def:wmr}, rule~\ref{def:wmr:must:tau}.
In particular divergence in an abstract state could be implemented by a deadlock.
If one of the above conditions holds in a certain state it is called a \emph{potential deadlock state}.
%States which just take into account (1) are called ``pure input states'' in~\cite{haddad-et-al-2013}
%and correspond to markings that ``stop except for inputs'' in~\cite{Stahl12}. This is however not sufficient in the context of
%modal refinement and observational abstraction considered here.

%%%old text:
%For the formalisation of the necessarily consuming property~\ref{def:cp-tgc-tgc:nec-consuming}
%we have to consider system runs
%since the required consumption should take place in all possible executions,
%starting from a state in which the channel is not empty, and this should hold in any environment
%and should be valid in any refinement. Let us first define potential deadlock states,
%i.e.\ states in which an execution might stop.
%Let $\S = \maiotstup{}$ be a \MAIOTS with $\ioalpha{}$.\linebreak A state $q \in Q$ is a \emph{potential deadlock state}
%if for all $a \in (\Sigma \setminus \inset)$, % $q \centernot{\mustw{a}} \stateTGC$, i.e.\
%there is no state $q' \in Q$ such that $q \mustw{a} q'$. 
% This means that there is no visible autonomous weak must-transition enabled in $q$. 
%This is a potential deadlock for three reasons: (1) the environment of $\S$ might not serve any inputs for \S, (2) the enabled proper may-transitions might be omitted in a refinement and (3) sequences of must $\tau$-transitions without successive mandatory visible actions can also be omitted in a refinement.
%In particular divergence in an abstract state can be refined to a deadlock.
%States which just take into account (1) are called ``pure input states'' in~\cite{haddad-et-al-2013}
%and correspond to markings that ``stop except for inputs'' in~\cite{Stahl12}. This is however not sufficient in the context of
%modal refinement and observational abstraction considered here.

Formally, let $\S = \maiotstup{}$ be a \MAIOTS with $\ioalpha{}$. A state $q \in Q$ is a potential deadlock state
if for all $a \in (\Sigma \setminus \inset)$, % $q \centernot{\mustw{a}} \stateTGC$, i.e.\
there is no state $q' \in Q$ such that $q \mustw{a} q'$. 
A \emph{run} of $\S$ starting in a state $q_1 \in Q$ is a finite or infinite sequence $\rho =q_1  \may{a_1} q_2 \may{a_2} q_3 \may{a_3} \cdots$ with $a_i \in \Sigmat$ and $q_i \in Q$ %for all $i$
such that, if the sequence is finite, its last state is a potential deadlock state.
%We denote the set of all runs of $\S$ starting from $q_1$ as $\run{\S}(q_1)$. 
We assume that system runs are executed in a runtime infrastructure which follows a fair scheduling policy. In our context this means that any visible autonomous action $a$, that is always enabled by a weak must-transition from a certain state on, will infinitely often be executed. 
Formally, a run $\rho = q_1 \may{a_1} q_2 \may{a_2} \cdots$ %\in \run{\S}(q_1)$
is called \emph{observationally weakly fair} if it is finite or if it is infinite and then for all $a \in (\Sigma \setminus \inset)$ the following holds:
\[
    (\exists k \geq 1\;.\; \forall i \geq k \;.\; \exists q'  \;.\; q_i \mustw{a} q') \implies (\forall k \geq 1 \;.\; \exists i \geq k \;.\; a_i = a).
\]
\noindent We denote the set of all observationally weakly fair runs of $\S$ starting from $q_1$ by $\frun{\S}(q_1)$.
For instance, for the MAIOPN $\M$ in Fig.~\ref{fig:obsFairness-con} on page~\pageref{fig:obsFairness-con} an infinite run which always executes
$\outact{a}, \inact{a}, \ldots$ is observationally weakly fair.
A (diverging) run of $\M$ which always executes $\tau$ from a certain state on
is not observationally weakly fair since $\inact{a}$ is then always enabled by a weak must-transition but never taken.

Note that for our results it is sufficient to use a weak fairness property instead of strong fairness.
We are now ready to define also the necessarily consuming property~\ref{def:cp-tgc-tgc:nec-consuming} which requires
that whenever a message is pending on a communication channel then the message must eventually be consumed on
all observationally weakly fair runs.%Both consuming properties are defined relative to a subset of the channels.

\begin{definition}[Message consuming]\label{def:cp-tgc-tgc}
    Let $\S = \maiotstup{}$ be a \MAIOTS with I/O-alphabet $\ioalpha{}$ and let $B \subseteq \chan$ be a subset of its channels.
            \begin{enumerate}[label=\alph*), ref=(\alph*), leftmargin=*, itemsep=1pt]
                \item\label{def:cp-tgc-tgc:bas-consuming}$\S$ is \emph{message consuming} w.r.t.\ $B$ if for all $a \in B$  and all $q \in \reach(\S)$,
                \\\hspace*{20pt}$
                    \val(q,a)>0 \implies \exists q',q''\in Q \;.\; \exists \sigma \in (\Sigma \setminus \inset)^* \: .\:q \mustw{\sigma} q' \must{\inact{a}} q''.
                 $
\vspace{1mm}
                \item\label{def:cp-tgc-tgc:nec-consuming}$\S$ is \emph{necessarily message consuming} w.r.t.\ $B$ if for all $a \in B, q \in \reach(\S)$,
\vspace{1mm}
                \\\hspace*{20pt}$
                  \val(q,a)>0 \implies \forall \rho \in \frun{\S}(q) \; .\;\;\inact{a} \in \rho \;$.\footnote{ i.e.\ there is a transition in $\rho$ labelled by $\inact{a}$.}
            \end{enumerate}
$\S$ is \emph{(necessarily) message consuming} if  $\S$ is (necessarily) message consuming w.r.t.\ $\chan$.

\end{definition}

In the special case, in which all may-transitions are must-transitions and no silent transitions occur
observationally weakly fair runs coincide with weakly fair runs and the two consuming properties coincide
with the corresponding properties in~\cite{haddad-et-al-2013}.

\begin{proposition}\label{prop:properties}
    Let $\S$ be a \MAIOTS. If $\S$ is necessarily message consuming w.r.t $B$ then $\S$ is message consuming w.r.t $B$.
\end{proposition}
\begin{proof}
    This is an adoptation of the proof in~\cite{TechReportVersion}.
    The proof relies on the fact that for each $q \in \reach(\S)$ there exists an observationally weakly fair run $\rho \in \wfrun{\S}(q)$, such that $\rho$ is a must-trace, $a \not\in \rho$ for all $a \in \inset$, and there is no infinite $\tau$ sequence in $\rho$. This run can be constructed by choosing, in an observationally weakly fair manner in each reached state, some enabled weak non-input must-action. If no such action is enabled in the last visited state, the last state is a deadlock state and we are done. Otherwise the resulting infinite run has the required property.

    With this fact, we can prove the proposition as follows: Let $q \in \reach(\S)$, $a \in B$ such that $\val(q,a) > 0$. By necessarily message consuming, for all $\rho \in \wfrun{\S}(q)$ we have $\inact{a} \in \rho$. Since we know there exists a weakly fair run $\rho$ without input actions, constructed as described above, we get that there exists $\sigma \in (\Sigma \setminus \inset)^*$ and $q'\in Q$ such that $q \mustw{\sigma} q' \must{\inact{a}}$. Thus $\S$ is message consuming w.r.t. $B$.
\end{proof}



%\vspace{3mm}
%\textbf{From Rolf:} Should we explain a counterexample for the converse direction?\\[3mm]

The definitions and the proposition are propagated to modal asynchronous I/O-Petri nets in the obvious way.
For instance, a \MAIOPN $\pnN$ is \emph{(necessarily) message consuming} if $\maiots(\pnN)$ is (necessarily) message consuming.
All \MAIOPNs considered in Sect.~\ref{sec:example} are necessarily message consuming.


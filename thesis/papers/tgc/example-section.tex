%!TEX root = ../../thesis.tex
\section{Illustrating Example}\label{sec:example}

We introduce an illustrating example to motivate our notions of modal asynchronous I/O-Petri nets (\MAIOPNs),
their composition, hiding and refinement. For this purpose we consider a top down approach to the design
of a simple compressing system  (inspired by \cite{BCD02}) which is able to receive files for compression and outputs either zip- or jpg-files.
We start with an interface specification of the system  modelled by the \textsf{CompressorInterface} in  Fig.~\ref{fig:compr-intf}.

\begin{figure}[ht]
	\centering
	%!TEX root = ../main.tex
\begin{tikzpicture}
% \node[place,tokens=0] at (-2,0) (p0) {};
\node[mustt,rotate=0,label={above:$\trans{file?}$}] at (0,0) (toa) {};
\node[place,tokens=0] at (2,0) (pa) {};
\node[mustt,rotate=0,label={below:$\trans{comprZip!}$}] at (4,-.5) (zip) {};
\node[mayt,rotate=0,label={above:$\trans{comprJpg!}$}] at (4,.5) (jpg) {};
\node[place,white] at (2,-1.4) (pb) {};


% \draw[arc] (p0) to (toa) {};
\draw[arc] (toa) to[] (pa) {};
\draw[arc] (pa) to[] (jpg) {};
\draw[arc] (pa) to[] (zip) {};
\end{tikzpicture}
	\caption{\textsf{CompressorInterface}.\label{fig:compr-intf}}
\end{figure}

The interface specification is presented by a labelled Petri net with distinguished input and output labels and with modalities on the transitions.
The label \textsf{file} suffixed with ``?'' indicates an input action and the labels \textsf{comprJpg}, \textsf{comprZip}
 suffixed with ``!'' indicate output actions. Following the idea of modal transition systems introduced by Larsen and Thomsen
in~\cite{DBLP:conf/lics/LarsenT88} transitions are equipped with ``must'' or ``may'' modalities. A must-transition, drawn black, indicates that
this transition is required for any refinement while a may-transition, drawn white, may also be removed or turned
into a must-transition. Models contain only must-transitions represent implementations.
In the example it is required that input files must always be received
and that the option to produce zipped text files is always available while
a refinement may or may not support the production of compressed jpg-files for graphical data.
Our interface specification models an infinite state system since an unbounded number of files can be received. 

\begin{figure}[ht]
	\centering
	%!TEX root = ../slides.tex
\begin{tikzpicture}
\tikzstyle{channel} = [circle,inner sep=0pt,minimum size=2mm,fill=white,draw=black]

%%% boxes and text

\draw[fill=lightlightblue] (0,-1) rectangle (6,-2) {};
\node at (3,-1.5) (cont) {\textsf{Controller}};

\draw[fill=lightlightblue] (1,0) rectangle (5,1) {};
\node at (3,.5) (gifC) {\textsf{GifCompressor}};

\draw[fill=lightlightblue] (1,-3) rectangle (5,-4) {};
\node at (3,-3.5) (txtC) {\textsf{TxtCompressor}};

%%%%%%%% channels gif
\node [channel,label=left:$\trans{gif}$] at (2.3333,-.5) (gif) {};
\node [channel,label=left:$\trans{jpg}$] at (3.6667,-.5) (jpg) {};
\node [channel,label=right:$\trans{fail}$] at (4,-.5) (fail) {};

\draw[arc] (2.3333,-1) to[] (gif) {};
\draw[arc] (gif) to[] (2.333,0) {};

\draw[arc] (3.6667,0) to[] (jpg) {};
\draw[arc] (jpg) to[] (3.6667,-1) {};

\draw[arc] (4,0) to[] (fail) {};
\draw[arc] (fail) to[] (4,-1) {};

%%%%%%%% channels txt
\node [channel,label=left:$\trans{txt}$] at (2.3333,-2.5) (txt) {};
\node [channel,label=right:$\trans{zip}$] at (4,-2.5) (zip) {};

\draw[arc] (2.3333,-2) to[] (txt) {};
\draw[arc] (txt) to[] (2.3333,-3) {};

\draw[arc] (4,-3) to[] (zip) {};
\draw[arc] (zip) to[] (4,-2) {};

%%%%%%% open channels
\draw[arc] (-2.6,-1.5) -- (0,-1.5) node[midway,above]{$\trans{file?}$};
\draw[arc] (6,-1.3) -- (8.6,-1.3) node[midway,above]{$\trans{comprJpg!}$};
\draw[arc] (6,-1.7) -- (8.6,-1.7) node[midway,below]{$\trans{comprZip!}$};

\end{tikzpicture}

	\caption{\textsf{CompressorAssembly} (architecture).\label{fig:ex-static-structure}}
\end{figure}

In the next step we propose an architecture for the realisation of the compressing system as shown in Fig.~\ref{fig:ex-static-structure}.
It is given by an assembly of three connected components, a \textsf{Controller} component which delegates the compression tasks,
a \textsf{GifCompressor} component which actually performs the compression of gif-files into jpg-files
and a \textsf{TxtCompressor} component which produces zip-files from text files.
The single components are connected by unbounded and unordered channels  \textsf{gif, jpg, ...} for asynchronous communication.

\begin{figure}[h!]
	\centering
	\subfloat[\textsf{Controller}.]{\label{fig:ex-controller}%!TEX root = ../main.tex
\begin{tikzpicture}


\node[mustt,rotate=0,label={left:$\trans{file?}$}] at (-2,-1) (filein) {};
\node[place,tokens=0] at (-1,-1) (p0) {};

\node[mustt,rotate=0,label={above:$\trans{gif!}$}] at (0,0) (gifout) {};
\node[mustt,rotate=0,label={below:$\trans{txt!}$}] at (0,-2) (txtout) {};

\node[place,tokens=0] at (1,0) (p1) {};
\node[place,tokens=0] at (1,-2) (p2) {};

\node[mustt,rotate=0,label={above:$\trans{jpg?}$}] at (3,0) (jpgin) {};
\node[mustt,rotate=0,label={right:$\trans{fail?}$}] at (1.8,-1) (fail) {};
\node[mustt,rotate=0,label={below:$\trans{zip?}$}] at (3,-2) (zipin) {};

\node[place,tokens=0] at (4,0) (p3) {};
\node[mustt,rotate=0,label={right:$\trans{comprJpg!}$}] at (5,0) (comprjpg) {};

\node[place,tokens=0] at (4,-2) (p4) {};
\node[mustt,rotate=0,label={right:$\trans{comprZip!}$}] at (5,-2) (comprzip) {};


\draw[arc] (filein) to[] (p0) {};

\draw[arc] (p0) to[bend left] (gifout) {};
\draw[arc] (p0) to[bend right] (txtout) {};

\draw[arc] (gifout) to[] (p1) {};
\draw[arc] (txtout) to[] (p2) {};

\draw[arc] (p1) to[] (jpgin) {};
\draw[arc] (p1) to[] (fail) {};
\draw[arc] (p2) to[] (zipin) {};

\draw[arc] (jpgin) to[] (p3) {};
\draw[arc] (fail) to[] (p0) {};
\draw[arc] (zipin) to[] (p4) {};

\draw[arc] (p3) to[] (comprjpg) {};
\draw[arc] (p4) to[] (comprzip) {};
\end{tikzpicture}}\qquad 
	\subfloat[\textsf{GifCompressor}.]{\label{fig:ex-gifCompressor}%!TEX root = ../../../thesis.tex
\begin{tikzpicture}
\node[place,tokens=1] at (1.5,4.2) (p7) {};

\node[mustt,rotate=0,label={left:$\trans{{gif?}}$}] at (0,3) (gifin) {};
\node[place,tokens=0] at (1,3) (p8) {};

\node[mayt,rotate=0,label={right:$\trans{{fail!}}$}] at (1.8,3) (failout) {};
\node[mustt,rotate=0,label={right:$\trans{{jpg!}}$}] at (3,3) (jpgout) {};


\draw[arc] (p7) to[bend right] (gifin) {};
\draw[arc] (gifin) to[] (p8) {};
\draw[arc] (p8) to[bend right=45] (jpgout) {};
\draw[arc] (p8) to[] (failout) {};
\draw[arc] (jpgout) to[bend right] (p7) {};
\draw[arc] (failout) to[bend right=15] (p7) {};


\end{tikzpicture}}	\qquad
	\subfloat[\textsf{TxtCompressor}.]{\label{fig:ex-txtCompressor}%!TEX root = ../main.tex
\begin{tikzpicture}

\node[place,tokens=1] at (-2,-2) (p0) {};

\node[mustt,rotate=0,label={left:$\trans{txt?}$}] at (-3,-1) (txtin) {};
\node[place,tokens=0] at (-2,-1) (p1) {};

\node[mustt,rotate=0,label={right:$\trans{zip!}$}] at (-1,-1) (zipout) {};

\draw[arc] (p0) to[bend left] (txtin) {};
\draw[arc] (txtin) to[] (p1) {};
\draw[arc] (p1) to[] (zipout) {};
\draw[arc] (zipout) to[bend left] (p0) {};
\end{tikzpicture}}
	\caption{\MAIOPN components.}
\end{figure}

The behaviour of the single components is modelled by the \MAIOPNs shown  in Figs.~\ref{fig:ex-controller},~\ref{fig:ex-gifCompressor} and~\ref{fig:ex-txtCompressor}.
The behaviour of the \textsf{CompressorAssembly} is given by the asynchronous composition of the single Petri nets
shown in Fig.~\ref{fig:ex-async-comp}.
For each pair of shared input and output actions a new place is introduced, called communication channel.
Transitions with a shared output label $a$ (of a given component) are connected to the new place $a$ and the transition label is renamed to
$\outact{a}$ in the composition. Similarly the place $a$ is connected to transitions with the corresponding input label $a$
which is then renamed to $\inact{a}$  in the composition. 
The result of our composition is very similar to the composition of open Petri nets,
see e.g.~\cite{Lohmann07}, which relies on matching of interface places.
But our approach is methodologically different since we introduce the communication places
only when the composition is constructed. In that way our basic components are not biased to
asynchronous composition but could be used for synchronous composition as well.
In this work we focus on asynchronous composition and we are particularly interested in the analysis
of generic communication properties ensuring
that messages pending on communication channels are eventually consumed. 
Therefore our notion of modal asynchronous I/O-Petri net will comprise an explicit discrimination of
channel places and, additionally to input/output labels we use, for each channel $a$,
distinguished communication labels $\outact{a}$ for putting messages on the channel and
$\inact{a}$ for consuming messages from the channel.  If the set of channels is empty a \MAIOPN
models an interface or a primitive component from which larger systems can be constructed by asynchronous composition.

\begin{figure}[ht]
	\centering
	%!TEX root = ../main.tex
\begin{tikzpicture}[
    background fill/.store in=\lightlightblue,
    background fill=lightlightblue,   % Default value
    fill background/.style={
       local bounding box=bbox,
       execute at end scope={
         \begin{pgfonlayer}{background}
           \coordinate[rectangle,fill=\lightlightblue,fit=(bbox)];
         \end{pgfonlayer}} }]

%%%%% Controller %%%%%
\begin{scope}[fill background]

\node[mustt,rotate=0,label={left:$\trans{file?}$}] at (-2,-1) (filein) {};
\node[place,tokens=0] at (-1,-1) (p0) {};

\node[mustt,rotate=0,label={below:$\trans{\outact{gif}}$}] at (0,0) (gifout) {};
\node[mustt,rotate=0,label={above:$\trans{\outact{txt}}$}] at (0,-2) (txtout) {};

\node[place,tokens=0] at (1,0) (p1) {};
\node[place,tokens=0] at (1,-2) (p2) {};

\node[mustt,rotate=0,label={below:$\trans{\inact{jpg}}$}] at (3,0) (jpgin) {};
\node[mustt,rotate=0,label={right:$\trans{\inact{fail}}$}] at (1.8,-1) (failin) {};
\node[mustt,rotate=0,label={above:$\trans{\inact{zip}}$}] at (3,-2) (zipin) {};

\node[place,tokens=0] at (4,0) (p3) {};
\node[mustt,rotate=0,label={right:$\trans{comprJpg?}$}] at (5,0) (comprjpg) {};

\node[place,tokens=0] at (4,-2) (p4) {};
\node[mustt,rotate=0,label={right:$\trans{comprZip?}$}] at (5,-2) (comprzip) {};

\draw[arc] (filein) to[] (p0) {};

\draw[arc] (p0) to[bend left] (gifout) {};
\draw[arc] (p0) to[bend right] (txtout) {};

\draw[arc] (gifout) to[] (p1) {};
\draw[arc] (txtout) to[] (p2) {};

\draw[arc] (p1) to[] (jpgin) {};
\draw[arc] (p1) to[] (failin) {};
\draw[arc] (p2) to[] (zipin) {};

\draw[arc] (jpgin) to[] (p3) {};
\draw[arc] (failin) to[] (p0) {};
\draw[arc] (zipin) to[] (p4) {};

\draw[arc] (p3) to[] (comprjpg) {};
\draw[arc] (p4) to[] (comprzip) {};
\end{scope}

%%%%% TxtCompressor %%%%%
\begin{scope}[fill background]


\node[place,tokens=1] at (1.5,-4.8) (p5) {};

\node[mustt,rotate=0,label={left:$\trans{\inact{txt}}$}] at (0,-4) (txtin) {};
\node[place,tokens=0] at (1.5,-4) (p6) {};

\node[mustt,rotate=0,label={right:$\trans{\outact{zip}}$}] at (3,-4) (zipout) {};

\draw[arc] (p5) to[bend left] (txtin) {};
\draw[arc] (txtin) to[] (p6) {};
\draw[arc] (p6) to[] (zipout) {};
\draw[arc] (zipout) to[bend left] (p5) {};
\end{scope}


%%%% Channels TXTCompressor %%%%%
\node[place,tokens=0,label=left:$\trans{txt}$] at (0,-3) (txt) {};
\node[place,tokens=0,label=right:$\trans{zip}$] at (3,-3) (zip) {};
\draw[arc] (txtout) to[] (txt) {};
\draw[arc] (txt) to[] (txtin) {};
\draw[arc] (zipout) to[] (zip) {};
\draw[arc] (zip) to[] (zipin) {};


%%%% GifCompressor %%%%

\begin{scope}[fill background]
\node[place,tokens=1] at (1.5,3.2) (p7) {};

\node[mustt,rotate=0,label={left:$\trans{\inact{gif}}$}] at (0,2.1) (gifin) {};
\node[place,tokens=0] at (1,2.1) (p8) {};

\node[mayt,rotate=0,label={right:$\trans{\outact{fail}}$}] at (1.8,2.1) (failout) {};
\node[mustt,rotate=0,label={right:$\trans{\outact{jpg}}$}] at (3,2.1) (jpgout) {};


\draw[arc] (p7) to[bend right] (gifin) {};
\draw[arc] (gifin) to[] (p8) {};
\draw[arc] (p8) to[bend right=35] (jpgout) {};
\draw[arc] (p8) to[] (failout) {};
\draw[arc] (jpgout) to[bend right] (p7) {};
\draw[arc] (failout) to[bend right=15] (p7) {};
\end{scope}


%%%% Channels GifCompressor %%%%%
\node[place,tokens=0,label=left:$\trans{gif}$] at (0,1) (gif) {};
\node[place,tokens=0,label=left:$\trans{fail}$] at (1.8,1) (fail) {};
\node[place,tokens=0,label=right:$\trans{jpg}$] at (3,1) (jpg) {};

\draw[arc] (gifout) to[] (gif) {};
\draw[arc] (gif) to[] (gifin) {};
\draw[arc] (failout) to[] (fail) {};
\draw[arc] (fail) to[] (failin) {};
\draw[arc] (jpgout) to[] (jpg) {};
\draw[arc] (jpg) to[] (jpgin) {};

\end{tikzpicture}
	\caption{\textsf{CompressorAssembly}.\label{fig:ex-async-comp}}
\end{figure}

In our example the behaviour of the \textsf{CompressorAssembly} is given by the asynchronous composition
\textsf{Controller} $\compospn$ \textsf{TxtCompressor}  $\compospn$ \textsf{GifCompressor}
shown in Fig.~\ref{fig:ex-async-comp}. It models a highly parallel system such that compressing of files can be executed concurrently and
new files can be obtained at the same time. 
Each single compressing tool, however, is working sequentially.
Its behaviour should be clear from the specifications.
The \textsf{GifCompressor} in Fig.~\ref{fig:ex-gifCompressor} has an optional behaviour modelled by a may-transition 
to indicate a compressing failure and then the controller will submit the file again.

After the assembly has been established we are interested in whether communication works properly in the sense
that pending messages on communication channels will be consumed. We will distinguish between two variants of
consumption requirements (see Sect.~\ref{sec:mcs}) expressing that for non-empty channels there must be a possibility for consumption or,
more strongly, that consumption must always happen on each (observationally weakly fair) run.
The fairness assumption is essential to support incremental design (Thm.~\ref{thm:inc-design-tgc}); for instance we can first check
that \textsf{Controller} $\compospn$ \textsf{TxtCompressor} has the desired communication properties for the
channels $\{$\textsf{txt, zip}$\}$, then we check that the full assembly
 has the desired communication properties for its
channel subset $\{$\textsf{gif, jpg, fail}$\}$ and from this we can automatically derive that the assembly has the
properties for \emph{all} its channels. 

It remains to show that the \textsf{CompressorAssembly} is indeed a realisation of the \textsf{CompressorInterface}.
For this purpose we consider the black-box behaviour of the assembly obtained by hiding the
communication channels. This is done by applying our hiding operator to the \textsf{CompressorAssembly}
denoted by \textsf{CompressorAssembly}$\setminus_{pn}\{$\textsf{gif, jpg, fail, txt, zip}$\}$; see Fig.~\ref{fig:ex-method}.
Hiding moves all communication labels
$\outact{a}$ and $\inact{a}$  for the hidden channels $a$ to the invisible action $\tau$.
In this way producing and consuming messages from hidden channels become silent transitions.
Now we have to establish a refinement relation between \textsf{CompressorAssembly}$\setminus_{pn}\{$\textsf{gif, jpg, fail, txt, zip}$\}$
and the \textsf{CompressorInterface} by taking into account the modalities on the transitions such that
must-transitions of the abstract specification must be available in the refinement and all transitions of the refinement
must be allowed by corresponding may-transitions of the abstract specification.
In our example the assembly has implemented the optional jpg compression of the interface by a must-transition.
Obviously we must also deal with silent transitions which, in our example,
occur in \textsf{CompressorAssembly}$\setminus_{pn}\{$\textsf{gif, jpg, fail, txt, zip}$\}$. For this purpose we use
a modal refinement relation, denoted by $\mrw$, which supports observational abstraction. 
In our case study this is expressed by the proof obligation (1) in Fig.~\ref{fig:ex-method}.

Fig.~\ref{fig:ex-method} illustrates that after the assembly is proven to be a correct realisation of the interface
one can still further refine the assembly by component-wise refinement of
its constituent parts. For instance, we can locally refine the \textsf{GifCompressor} by resolving
the may-modality for producing failures. There are basically two possibilities: Either the failure option is removed
or it is turned into a must-transition. The component \textsf{GifCompressorRef} in Fig.~\ref{fig:ex-method}
represents such a refinement of \textsf{GifCompressor} indicated by (2).
We will show in Thm.~\ref{thm:wmr}.\ref{thm:copos-wmr} that refinement is compositional,
i.e.\ we obtain automatically that the assembly \textsf{CompressorAssemblyRef} obtained by composition
with the new gif compressor is a refinement of
 \textsf{CompressorAssembly} indicated by (3) in  Fig.~\ref{fig:ex-method}. 
As a crucial result we will also show in Thm.~\ref{thm:wmr-properties}
that refinement preserves communication properties which then can be automatically
derived for \textsf{CompressorAssemblyRef}.
Finally, we must be sure that
the refined assembly provides a realisation of the original interface specification \textsf{CompressorInterface}.
This can be again automatically achieved since hiding preserves refinement
(Thm.~\ref{thm:wmr}.\ref{thm:wmr-hiding})
which leads to (4)  in  Fig.~\ref{fig:ex-method} and since refinement is transitive.

\begin{figure}%
\centering
%!TEX root = ../../../thesis.tex
\begin{tikzpicture}[scale=0.65, every node/.style={transform shape}]

%%%%% CompressorAssembly %%%%%
\node at (0,0) (structure) {%!TEX root = ../slides.tex
\begin{tikzpicture}
\tikzstyle{channel} = [circle,inner sep=0pt,minimum size=2mm,fill=white,draw=black]

%%% boxes and text

\draw[fill=lightlightblue] (0,-1) rectangle (6,-2) {};
\node at (3,-1.5) (cont) {\textsf{Controller}};

\draw[fill=lightlightblue] (1,0) rectangle (5,1) {};
\node at (3,.5) (gifC) {\textsf{GifCompressor}};

\draw[fill=lightlightblue] (1,-3) rectangle (5,-4) {};
\node at (3,-3.5) (txtC) {\textsf{TxtCompressor}};

%%%%%%%% channels gif
\node [channel,label=left:$\trans{gif}$] at (2.3333,-.5) (gif) {};
\node [channel,label=left:$\trans{jpg}$] at (3.6667,-.5) (jpg) {};
\node [channel,label=right:$\trans{fail}$] at (4,-.5) (fail) {};

\draw[arc] (2.3333,-1) to[] (gif) {};
\draw[arc] (gif) to[] (2.333,0) {};

\draw[arc] (3.6667,0) to[] (jpg) {};
\draw[arc] (jpg) to[] (3.6667,-1) {};

\draw[arc] (4,0) to[] (fail) {};
\draw[arc] (fail) to[] (4,-1) {};

%%%%%%%% channels txt
\node [channel,label=left:$\trans{txt}$] at (2.3333,-2.5) (txt) {};
\node [channel,label=right:$\trans{zip}$] at (4,-2.5) (zip) {};

\draw[arc] (2.3333,-2) to[] (txt) {};
\draw[arc] (txt) to[] (2.3333,-3) {};

\draw[arc] (4,-3) to[] (zip) {};
\draw[arc] (zip) to[] (4,-2) {};

%%%%%%% open channels
\draw[arc] (-2.6,-1.5) -- (0,-1.5) node[midway,above]{$\trans{file?}$};
\draw[arc] (6,-1.3) -- (8.6,-1.3) node[midway,above]{$\trans{comprJpg!}$};
\draw[arc] (6,-1.7) -- (8.6,-1.7) node[midway,below]{$\trans{comprZip!}$};

\end{tikzpicture}
};
\node[above=-1mm of structure] (structuretitle){\textsf{CompressorAssembly}};
\node[fit=(structure)(structuretitle), draw] (comprAssBox){};

%%%%% MR 3 %%%%%
\node at ($ (comprAssBox.south) + (0,-.35) $) (mr3) {\rotatebox{90}{$\mrw$}};
\node[left=0.1mm of mr3] {3)};

%%%%% CompressorAssemblyRef %%%%%
\node at (0,-6.95) (structureref) {%!TEX root = ../slides.tex
\begin{tikzpicture}
\tikzstyle{channel} = [circle,inner sep=0pt,minimum size=2mm,fill=white,draw=black]

%%% boxes and text

\draw[fill=lightlightblue] (0,-1) rectangle (6,-2) {};
\node at (3,-1.5) (cont) {\textsf{Controller}};

\draw[fill=lightlightblue] (1,0) rectangle (5,1) {};
\node at (3,.5) (gifC) {\textsf{GifCompressorRef}};

\draw[fill=lightlightblue] (1,-3) rectangle (5,-4) {};
\node at (3,-3.5) (txtC) {\textsf{TxtCompressor}};

%%%%%%%% channels gif
\node [channel,label=left:$\trans{gif}$] at (2.3333,-.5) (gif) {};
\node [channel,label=left:$\trans{jpg}$] at (3.6667,-.5) (jpg) {};
\node [channel,label=right:$\trans{fail}$] at (4,-.5) (fail) {};

\draw[arc] (2.3333,-1) to[] (gif) {};
\draw[arc] (gif) to[] (2.333,0) {};

\draw[arc] (3.6667,0) to[] (jpg) {};
\draw[arc] (jpg) to[] (3.6667,-1) {};

\draw[arc] (4,0) to[] (fail) {};
\draw[arc] (fail) to[] (4,-1) {};

%%%%%%%% channels txt
\node [channel,label=left:$\trans{txt}$] at (2.3333,-2.5) (txt) {};
\node [channel,label=right:$\trans{zip}$] at (4,-2.5) (zip) {};

\draw[arc] (2.3333,-2) to[] (txt) {};
\draw[arc] (txt) to[] (2.3333,-3) {};

\draw[arc] (4,-3) to[] (zip) {};
\draw[arc] (zip) to[] (4,-2) {};

%%%%%%% open channels
\draw[arc] (-2.6,-1.5) -- (0,-1.5) node[midway,above]{$\trans{file?}$};
\draw[arc] (6,-1.3) -- (8.6,-1.3) node[midway,above]{$\trans{comprJpg!}$};
\draw[arc] (6,-1.7) -- (8.6,-1.7) node[midway,below]{$\trans{comprZip!}$};

\end{tikzpicture}
};
\node[above=-1mm of structureref] (structurereftitle){\textsf{CompressorAssemblyRef}};
\node[fit=(structureref)(structurereftitle), draw] (comprAssRefBox) {};



%%%%% CompressorHide1 %%%%%
\draw[fill=lightlightblue] ($ (comprAssBox.south east) + (2,0) $) rectangle ($ (comprAssBox.south east) + (6.35,1.5) $) node[fitting node] (comprHideBox) {};
\node at (comprHideBox) (comprHide) {
\begin{tabular}{c}
     \textsf{CompressorAssembly} \\
     $\hidepn \{\trans{gif},\trans{jpg},\trans{fail},\trans{txt},\trans{zip}\}$
  \end{tabular}
 };

%%%%% MR 4 %%%%%
\node at ($ (comprHideBox.south) + (0,-.35) $) (mr4) {\rotatebox{90}{$\mrw$}};
\node[left=0.1mm of mr4] {4)};

%%%%% CompressorHide2 %%%%%
\draw[fill=lightlightblue] ($ (comprHideBox.south west) + (0,-.85) $) rectangle ($ (comprHideBox.south east) + (0,-2.35) $) node[fitting node] (comprHide2Box) {};
\node at (comprHide2Box) (comprHide) {
\begin{tabular}{c}
     \textsf{CompressorAssemblyRef} \\
     $\hidepn \{\trans{gif},\trans{jpg},\trans{fail},\trans{txt},\trans{zip}\}$
  \end{tabular}
 };

%%%%% CompressorInterface %%%%%
\draw[fill=lightlightblue] ($ (comprHideBox.north east) + (0,1) $) rectangle ($ (comprHideBox.north west) + (0,2.15) $) node[fitting node] (comprIntBox) {};
\node at (comprIntBox) (comprInt) {\textsf{CompressorInterface}};

%%%%% MR 1 %%%%%
\node at ($ (comprIntBox.south) + (0,-.5) $) (mr1) {\rotatebox{90}{$\mrw$}};
\node[left=0.1mm of mr1] {1)};


%%%%% Hide 1 %%%%%
\draw[arc,thin] ($ (comprHideBox.west) + (-1.6,0) $) -- ($ (comprHideBox.west) + (-.4,0) $)  node[midway,above]{{hide}} {};

%%%%% Hide 2 %%%%%
\draw[arc,thin] ($ (comprHide2Box.west) + (-1.6,0) $) -- ($ (comprHide2Box.west) + (-.4,0) $)  node[midway,above]{{hide}} {};


%%%%% GifCompressor %%%%%
\draw[fill=lightlightblue] ($ (comprAssBox.south west) + (-2,0) $) rectangle ($ (comprAssBox.south west) + (-5.8,1.0) $) node[fitting node] (gifComprBox) {};
\node at (gifComprBox) (gifCompr) {\textsf{GifCompressor}};

%%%%% MR 2 %%%%%
\node at ($ (gifComprBox.south) + (0,-.35) $) (mr2) {\rotatebox{90}{$\mrw$}};
\node[left=0.1mm of mr2] {2)};

%%%%% GifCompressorRef %%%%%
\draw[fill=lightlightblue] ($ (gifComprBox.south west) + (0,-.85) $) rectangle ($ (gifComprBox.south east) + (0,-2) $) node[fitting node] (gifCompr2Box) {};
\node at (gifCompr2Box) (gifCompr2) {\textsf{GifCompressorRef}};

%%%%% Compose 1 %%%%%
\draw[arc,thin] ($ (gifComprBox.east) + (0.4,0) $) -- ($ (gifComprBox.east) + (1.6,0) $)  node[midway,above]{{compose}} {};

%%%%% Compose 1 %%%%%
\draw[arc,thin] ($ (gifCompr2Box.east) + (0.4,0) $) -- ($ (gifCompr2Box.east) + (1.6,0) $)  node[midway,above]{{compose}} {};
\end{tikzpicture}
\caption{System development methodology.}%
\label{fig:ex-method}%
\end{figure}

In the next sections we will formally elaborate the notions discussed above.
We hope that their intended meaning is already sufficiently explained
such that we can keep the presentation short.
An exception concerns the consideration of the message consuming properties in Sect.~\ref{sec:mcs}
which must still be carefully studied to ensure incremental design and preservation by refinement.





\section{Decidability of Channel Properties}
\label{sec:decidability}

We begin this section by recalling some  information related to
semi-linear sets and decision procedures in Petri nets that we use in our proofs.

Let $E \subseteq \N^k$, $E$ is a \emph{linear set}
if there exists a finite set of vectors of $\N^k$ $\{v_0,\ldots,v_n\}$
such that $E=\{v_0 + \sum_{1 \leq i \leq n} \lambda_i v_i\mid \forall i\ \lambda_i \in \N\}$.
A \emph{semi-linear set}~\cite{gs66} is a finite union of linear sets; a representation of it is given
by the family of finite sets of vectors defining the corresponding linear sets.
Semi-linear sets are \emph{effectively} closed w.r.t. union, intersection and complementation.
This means that one can compute a representation of the union, intersection and complementation
starting from a representation of the original semi-linear sets. 
$E$ is an \emph{upward closed set} if $\forall v \in E.\ v'\geq v \Rightarrow v' \in E$.
An upward closed set has a finite set of minimal vectors denoted $\min(E)$. 
An upward closed set is a semi-linear set which has a representation that can be derived
from the equation $E=\min(E)+\N^k$ if $\min(E)$ is computable.

Given a Petri net $\mathcal N$  and a marking $m$, the \emph{reachability}
problem consists in deciding whether $m$ is reachable from $m_0$
in $\mathcal N$. This problem is decidable~\cite{ReachaPbMayer}
but none of the associated algorithms are primitive recursive. 
Furthermore this procedure can be adapted to semi-linear sets
when markings are identified to vectors of $\N^{|P|}$.
Based on reachability analysis, the authors of~\cite{dj89} design an algorithm  
that decides whether a marking $m$ is a \emph{home state}, i.e. $m$ is reachable
from any reachable
marking. A more general problem
is in fact decidable: given a subset of places $P'$ and a (sub)marking $m$ on
this subset, is it possible from any reachable marking to reach
a marking that coincides on $P'$ with $m$?

In~\cite{rackoff78}, the \emph{coverability} is shown 
to be EXPSPACE-complete. The coverability problem
consists in determining, given a net and a target marking, whether one can reach a marking
greater or equal than the target.  
In~\cite{VJ84} given a Petri net, several procedures have been designed to compute the minimal
set of markings of several interesting upward closed sets. In particular,
given an upward closed set $Target$, by a backward analysis one can compute
the (representation of) upward closed set from which $Target$ is reachable.
Using the results of~\cite{rackoff78}, this algorithm performs in EXPSPACE.

While in Petri nets, strong fairness is undecidable~\cite{carst87}, weak fairness is decidable and more generally, 
the existence of an infinite sequence fulfilling a formula of the
following fragment of LTL is decidable~\cite{Jancar90}. The literals are (1) comparisons
between places markings and values, (2) transition firings and (3)  their negations. Formulas
are inductively defined as literals, conjunction or disjunction of formulas and $GF \varphi$
where $GF$ is the infinitely often operator and $\varphi$ is a formula. 
%In~\cite{esparza94}, the \emph{model checking} of several temporal logic formulas on Petri nets  is
%studied. In particular, it is shown that the model checking problem of event-based LTL formulas w.r.t.
%infinite sequences of the net belongs to EXPSPACE. 



The next theorem establishes the decidability of  the strong properties
of type (b) of Def.~\ref{def:cp}. 
Observe that their proofs are closely related and that they rely on the decidability
of reachability and coverability problems.
\begin{theorem}
\label{thm:strong-dec}
The following problems are decidable for AIOPNs: Is an AIOPN $\mathcal N$ 
% \begin{itemize}
% \item 
strongly $B$-consuming,
% \item 
strongly $B$-decreasing, 
% \item 
strongly $B$-emptying, 
% \item 
strongly $B$-wholly emptying?
% \end{itemize}
\end{theorem}
\begin{proof}
	{\bf Strongly $B$-consuming (and strongly $B$-decreasing).}
	Given an AIOPN $\mathcal N$ and $B$ a subset of its channels, one
	decides whether $\mathcal N$ is strongly $B$-consuming as follows.
	Let $E$ be the set of markings defined by:\\
	\[E=\{m\mid \exists a \in B\ m(a)>0 \mbox{ and } \forall t \in T
	\mbox{ with }\lambda(t)=\inact{a}\ m\not\geq W^-(t)\}\]
	$\mathcal N$ is strongly $B$-consuming iff $E$ is not reachable.
	Since $E$ is a semi-linear set, its reachability is decidable. 


	\smallskip\noindent {\bf Strongly $B$-emptying.} 
	Given an AIOPN $\mathcal N$ and $B$ a subset of its channels, one
	decides whether $\mathcal N$ is strongly $B$-emptying as follows.

	\smallskip\noindent
	First by coverability analysis, one decides whether the following upward closed
	subset $E$ is reachable:\\
	\[E=\{m\mid \exists a \in B\ m(a)>1\}.\]
	If $E$ is reachable then $\mathcal N$ is not strongly $B$-emptying.
	Otherwise one checks whether $\mathcal N$ is strongly $B$-consuming
	in order to decide.

	\smallskip\noindent {\bf Strongly $B$-wholly emptying.} 
	Given an AIOPN $\mathcal N$ and $B$ a subset of its channels, one
	decides whether $\mathcal N$ is strongly $B$-wholly emptying as follows.

	\smallskip\noindent
	First by coverability analysis, one decides whether the following upward closed
	subset $E$ is reachable:\\
	\[E=\{m\mid m(B)>1\}.\]
	If $E$ is reachable then $\mathcal N$ is not strongly $B$-wholly emptying.
	Otherwise one checks whether $\mathcal N$ is strongly $B$-consuming
	in order to decide.
\end{proof}


\smallskip
The next theorem establishes the decidability of the properties
of type (a) of Def.~\ref{def:cp}. 
Observe that their proofs rely on  (1) the effectiveness of backward analysis 
for upward closed marking sets (2) the decidability
of reachability and home space problems and (3)  appropriate
transformations of the net.


\begin{theorem}
The following problems are decidable for AIOPNs: Is an AIOPN $\mathcal N$  $B$-consuming,
$B$-decreasing,  $B$-emptying,  $B$-wholly emptying?
\end{theorem}
\begin{proof}~\\
{\bf $B$-consuming.}
Given an AIOPN $\mathcal N$ and $B$ a subset of its channels, one
decides whether $\mathcal N$ is $B$-consuming as follows.

\smallskip\noindent
Let $a \in B$ and $E_a$ be the upward closed set of markings defined by:\\
\[E_a=\{m\mid \exists t \in T \mbox{ with }\lambda(t)=\inact{a} \mbox{ and } m\geq W^-(t)\}\] 
$E_a$ is the set of markings from which one can immediately consume some message~$a$.
Let $F_a$ be the upward closed set of markings defined by:\\ 
\[
	F_a=\{m\mid \exists m'\in E_a\ \exists \sigma \in T^*.\ \lambda(\sigma) \in (\Sigma\setminus \inset)^* \wedge m \xrightarrow{\sigma} m'\}
\]
$F_a$ is the set of markings from which one can later (without the help of the environment)
consume some message $a$. One computes $F_a$ by backward analysis. Let $G$ be defined 
by: 
\[
G=\{m \mid \exists a \in B. \ m(a)>0 \wedge m \notin F_a\}
\]
$G$ is a semi-linear set corresponding to the markings
from which some message $a\in B$ will never be consumed. 
Then  $\mathcal N$ is not $B$-consuming
iff $G$ is reachable.

\smallskip\noindent {\bf $B$-emptying (and  $B$-decreasing).}
Given an AIOPN $\mathcal N$ and $B$ a subset of its channels, one
decides whether $\mathcal N$ is $B$-emptying as follows.
First one builds a net~$\mathcal N'$: 

\begin{itemize}
%
	\item $P'=P\uplus \{run\}$
%
	\item $T'=T\uplus \{stop\}$
%
	\item $\forall p \in P\ \forall t \in T\ W'^-(p,t)=W^-(p,t), W'^+(p,t)=W^+(p,t)$, $m'^0(p)=m^0(p)$
%
	\item $W'^-(run,stop)=1,$, $W'^+(run,stop)=0$, $m'^0(run)=1$
%
	\item $\forall p \in P\  W'^-(p,stop)=W'^+(p,stop)=0$
%
	\item $\forall t \in T\ \mbox{ such that }\lambda(t)\in \inset\ W'^-(run,t)=W'^+(run,t)=1$
%
	\item $\forall t \in T\ \mbox{ such that }\lambda(t)\notin \inset\ W'^-(run,t)=W'^+(run,t)=0$
%
\end{itemize}

$\mathcal N'$ behaves as $\mathcal N$ as long as $stop$ is not fired.
When $stop$ is fired only transitions not labelled by inputs are fireable.
Thus $\mathcal N$ is $B$-emptying iff for all $a\in B$
the set of markings $Z_a=\{m \mid m(a)=0\}$ is a home space for $\mathcal N'$.

\smallskip\noindent {\bf $B$-wholly emptying.}
Using the same construction $\mathcal N$ is $B$-weakly wholly emptying
if $Z_B=\{m \mid m(B)=0\}$ is a home space for $\mathcal N'$.
\end{proof}

\smallskip
The next theorems, 
establish the decidability of the necessarily properties
of type (c) of Def.~\ref{def:cp} and the communication stopping properties. 

\begin{theorem}
\label{thm:nec-dec}
The following problems are decidable for AIOPNs: Is an AIOPN $\mathcal N$  necessarily $B$-consuming,
necessarily $B$-decreasing,  necessarily $B$-emptying,  necessarily $B$-wholly emptying?
\end{theorem}
\begin{proof}
	{\bf Necessarily $B$-consuming.}
Given an AIOPN $\mathcal N$ and $B$ a subset of its channels, one
decides whether $\mathcal N$ is necessarily $B$-consuming as follows.

\noindent First one checks whether $\mathcal N$ is $B$-consuming,
a necessary condition for being necessarily $B$-consuming. 
If $\mathcal N$ is $B$-consuming, then one checks whether
for some $a \in B$, there exists a reachable marking $m$ fulfilling $m(a)>0$ from which an infinite
weakly fair run is fireable without occurrence of transitions labelled by $\inact{a}$. To perform
this test, one builds a net $\mathcal N'$ as follows.

\begin{itemize}
%
	\item $P'=P\uplus \{run\}$
%
	\item $T'=T\uplus \{stop\}$ 
%
	\item $\forall p \in P\ \forall t \in T\ W'^-(p,t)=W^-(p,t), W'^+(p,t)=W^+(p,t)$, $m'^0(p)=m^0(p)$
%
	\item $W'^-(run,stop)=1,$, $W'^+(run,stop)=0$, $m'^0(run)=1$
%
	\item $W'^-(a,stop)=W'^+(a,stop)=1$
%
	\item $\forall p \in P\setminus \{a\}\  W'^-(p,stop)=W'^+(p,stop)=0$
%
	\item $\forall t \in T\ \mbox{ such that }\lambda(t)=\inact{a}\ W'^-(run,t)=W'^+(run,t)=1$
%
	\item $\forall t \in T\ \mbox{ such that }\lambda(t)\neq\inact{a}\ W'^-(run,t)=W'^+(run,t)=0$
%
\end{itemize}

Transition $stop$ can be fired only if $m(a)>0$.
When $stop$ is fired only transitions not labelled by $\inact{a}$ are fireable.
Then one checks whether there exists an infinite weakly fair run 
that fulfils formula $GF run=0 \wedge GF \neg \mbox{fireable}(\inact{a})$
in $\mathcal N'$. Since $\mbox{fireable}(\inact{a})$ is expressible
by a boolean combination of comparisons of place markings with constants,
the formula belongs to the logic of~\cite{Jancar90}.
The first term witnesses the firing of $stop$
and the second term ensures that the sequence is also weakly
fair in $\mathcal N$. Then $\mathcal N$ is necessarily $B$-consuming 
iff there is no such run. 

\smallskip\noindent {\bf Necessarily $B$-emptying (and $B$-decreasing).}
Given an AIOPN $\mathcal N$ and $B$ a subset of its channels, one
decides whether $\mathcal N$ is necessarily $B$-emptying as follows.

\noindent First one checks whether $\mathcal N$ is $B$-emptying,
a necessary condition for being necessarily $B$-emptying. 
If $\mathcal N$ is $B$-emptying, then one checks whether
for some $a \in B$, there exists a reachable marking $m$ from which an infinite weakly fair run 
is fireable such that for every marking $m'$ visited, $m'(a)>0$. To perform
this test, one builds a net $\mathcal N'$ as follows.

\begin{itemize}
%
	\item $P'=P\uplus \{run\}$
%
	\item $T'=T\uplus \{stop\}$ with $\lambda(stop)=stop$
%
	\item $\forall p \in P\ \forall t \in T\ W'^-(p,t)=W^-(p,t), W'^+(p,t)=W^+(p,t)$, $m'^0(p)=m^0(p)$
%
	\item $W'^-(run,stop)=1$, $W'^+(run,stop)=0$, $m'^0(run)=1$,
%
	\item $W'^-(a,stop)=1$, $W'^+(a,stop)=0$
%
	\item $\forall p \in P \setminus \{a\}\  W'^-(p,stop)=W'^+(p,stop)=0$
%
	\item $\forall t \in T\  W'^-(run,t)=W'^+(run,t)=0$
%
\end{itemize}

Transition $stop$ can be fired only if $m(a)>0$ and it consumes one token of $a$.
Transition $stop$ can be fired only once due to place $run$.
Then one checks whether there exists an infinite weakly fair run 
that fulfils formula $GF run=0$ (witnessing the firing of $stop$) in
$\mathcal N'$. 
Then $\mathcal N$ is necessarily $B$-emptying 
iff there is no such run. 
Indeed there is an infinite weakly fair run in $\mathcal N'$
after the firing of $stop$ iff from some marking $m$ in $\mathcal N$,
there is an infinite weakly fair run where the marking of $a$ is never zero
from some state.

\smallskip\noindent {\bf Necessarily $B$-wholly emptying.}
Given an AIOPN $\mathcal N$ and $B$ a subset of its channels, one
decides whether $\mathcal N$ is necessarily $B$-wholly emptying as follows.

\noindent First one checks whether $\mathcal N$ is $B$-wholly emptying,
a necessary condition for being necessarily $B$-wholly emptying. 
If $\mathcal N$ is $B$-wholly emptying, then one checks whether
there exists a reachable marking $m$ from which an infinite
weakly fair run is fireable such that for every marking $m'$ visited, $m'(B)>0$. To perform
this test, one builds a net $\mathcal N'$ as follows. 

\begin{itemize}
%
	\item $P'=P\uplus \{run,B\}$
%
	\item $T'=T\uplus \{stop\}$ with $\lambda(stop)=stop$
%
	\item $\forall p \in P\ \forall t \in T\ W'^-(p,t)=W^-(p,t), W'^+(p,t)=W^+(p,t)$, $m'^0(p)=m^0(p)$
%
	\item $\forall t \in T\ W'^-(B,t)=\sum_{a\in B} W^-(a,t),$
	$W'^+(B,t)=\sum_{a\in B} W^+(a,t)$, $m'^0(B)=0$
%
	\item $W'^-(run,stop)=1$, $W'^+(run,stop)=0$, $m'^0(run)=1$,
%
	\item $W'^-(B,stop)=1$, $W'^+(B,stop)=0$
%
	\item $\forall p \in P\  W'^-(p,stop)=W'^+(p,stop)=0$
%
	\item $\forall t \in T\  W'^-(run,t)=W'^+(run,t)=0$
%
\end{itemize}


In $\mathcal N'$ there is an additional place $B$ containing the sum of tokens
of places $a \in B$ whose management is straightforward.
As in the previous constructions, there is a control transition
$stop$ that modifies the behaviour of $\mathcal N$. As long
as $stop$ is not fired, $\mathcal N'$ behaves as $\mathcal N$.
In order to fire $stop$ (which can be done only once), $B$ is decreased. 
Then one checks whether there exists an infinite weakly fair run 
that fulfils formula $GF run=0$ (witnessing the firing of $stop$) in
$\mathcal N'$. 
$\mathcal N$ is necessarily $B$-wholly emptying 
iff there is no such run. 
\end{proof}


\begin{theorem}
\label{thm:management-dec}
The following problems are decidable for AIOPNs: Is an AIOPN $\mathcal N$
$B$-communication stopping, strongly $B$-communication stopping? 
\end{theorem}
\begin{proof}
	{\bf $B$-communication stopping.}
Given an AIOPN $\mathcal N$ and $B$ a subset of its channels, one
decides whether $\mathcal N$ is $B$-communication stopping as follows.
Observe that this property is expressed by the event-based LTL
formula $\varphi=\bigwedge_{a \in B} GF \outact{a} \Rightarrow GF \inact{a}$. 
By definition, this property is satisfied for finite runs. 
However $\neg \varphi$ \emph{is not a formula of the fragment
of~\cite{Jancar90}}. In order to check whether there exists
a weakly fair infinite run satisfying the formula $\neg \varphi$,
one builds net $\mathcal N_a$ as follows.

\begin{itemize}
%
	\item $P'=P\uplus \{run\}$
%
	\item $T'=T\uplus \{stop\}$ 
%
	\item $\forall p \in P\ \forall t \in T\ W'^-(p,t)=W^-(p,t), W'^+(p,t)=W^+(p,t)$, $m'^0(p)=m^0(p)$
%
	\item $W'^-(run,stop)=1,$, $W'^+(run,stop)=0$, $m'^0(run)=1$
%
	\item $\forall p \in P\  W'^-(p,stop)=W'^+(p,stop)=0$
%
	\item $\forall t \in T\ \mbox{ such that }\lambda(t)=\inact{a}\ W'^-(run,t)=W'^+(run,t)=1$
%
	\item $\forall t \in T\ \mbox{ such that }\lambda(t)\neq\inact{a}\ W'^-(run,t)=W'^+(run,t)=0$
%
\end{itemize}

$\mathcal N_a$ behaves as $\mathcal N$ as long as $stop$ is not fired.
Transition $stop$ can be fired only once.
When $stop$ is fired only transitions not labelled by $\inact{a}$ are fireable.
Then one checks whether there exists an infinite weakly fair run 
that fulfils formula $GF run=0 \wedge  GF \outact{a} $ in
$\mathcal N'$. Then $\mathcal N$ is $B$-communication stopping 
iff there is no such run in any $\mathcal N_a$.

\smallskip\noindent {\bf Strongly $B$-communication stopping.}
Given an AIOPN $\mathcal N$ and $B$ a subset of its channels, one
decides whether $\mathcal N$ is strongly $B$-communication stopping as follows.

\smallskip\noindent
For all $a \in B$, one builds a net $\mathcal N_a$
starting from $\mathcal N$.
Since the formal specification of  $\mathcal N_a$ is cumbersome, one describes
it in words.

\begin{itemize}
%
	\item $\mathcal N_a$ has two additional places $run$ and $wit$, 
	with $m'^0(run)=1$, $m'^0(wit)=0$.
%
	\item Place $run$ loops around all transitions labelled by $\inact{a}$.
%
	\item Every transition $t$ labelled by $\outact{a}$ has a copy $t'$
	with the same inputs and outputs, plus an additional input $run$
	and an additional output $wit$.
%
\end{itemize}

$\mathcal N_a$ behaves as $\mathcal N$ until a transition
$t'$ is fired. This firing is possible only once. Once a transition $t'$
is fired place $wit$ is marked and transitions labelled by $\inact{a}$
are no more fireable. Define the semi-linear set $E_a$ by:\\
\[E_a=\{m \mid m(wit)= 1 \wedge \forall t\in T\ .\  
\lambda(t)\notin \inset\ \implies m\not\geq W^-(t)\}\]
Observe that a marking of $E_a$ is a pure input state of the original net 
(i.e. when restricted to $P$)
that has been reached by a finite run when an occurrence of $\outact{a}$ has not
been followed later by an occurrence of $\inact{a}$.  

\smallskip
So $\mathcal N$ is strongly $B$-communication stopping iff  for all $a \in B$, in $\mathcal N_a$ 
there does not exist an infinite weakly fair run fulfilling $GF run=0$ 
(witnessing a \emph{bad} weakly fair infinite run in $\mathcal N$)
and $E_a$ is not reachable (witnessing a \emph{bad} finite run in $\mathcal N$).
\end{proof}
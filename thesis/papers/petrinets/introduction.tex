%!TEX root = ./main.tex
\section{Introduction}
\label{sec:introduction}


{\bf (A)synchronous composition.} 
The design of hardware and software systems
is often component-based which has well-known advantages: management
of complexity, reusability, separation of concerns, collaborative design, etc.
One critical feature of such systems is the protocol
supporting the communication between components and, in particular,
the way they synchronise. Synchronous composition ensures
that both parts are aware that communication has taken place
and then simplifies the validation of the system.
However in a large scale distributed environment synchronous
composition may lead to redhibitory inefficiency during execution
and thus asynchronous composition should be adopted.
The FIFO requirement of communication channels is often not appropriate
in this context.
This is illustrated 
by the concept of a software bus where applications push and pop messages
in mailboxes. Also on the modelling level FIFO ordering is often not assumed,
like for the composition of UML state machines which relies on event pools without specific
requirements.


\noindent
{\bf Compositions of Petri nets.}
In the context of Petri nets, composition has been studied both from
theoretical and practical points of view. The process algebra approach
has been investigated by several works leading to the Petri net 
algebra~\cite{BDK01}. Such an approach is closely related to synchronous composition.
In~\cite{Souissi90} and \cite{SM89} asynchronous composition of nets are performed
via a set of places or, more generally, via a subnet modelling some medium. Then
structural restrictions on the subnets are proposed in order to preserve
global properties like liveness or deadlock-freeness.
In~\cite{reisig09} a general composition operator is proposed and
its associativity is established. A closely related concept to composition
is the one of open Petri nets which has been used in different contexts
like the analysis of web services~\cite{DK2009}.
Numerous compositional approaches have been proposed for the modelling of complex applications
but most of them are based on high-level Petri nets; see~\cite{GB05} for a detailed
survey.


\noindent
{\bf Channel properties.} With the development of component-based
applications, one is interested in verifying behavioural properties
of the communication and, in the asynchronous
case, in verifying the properties related to communication channels.
Channel properties naturally occur when reasonning about distributed
mechanisms, algorithms and applications (e.g. management of sockets in UNIX,
maintaining unicity of a token in a ring based algorithm, recovery points
with empty channels for fault management, guarantee of email reading, etc.).



\noindent
{\bf Our contributions.}
In this work we are interested in general channel properties and not in specific
system properties related to particular applications.
The FIFO requirement for channels
potentially can decrease the performance
of large scale distributed systems. Thus we restrict ourselves to
\emph{unordered} channels which can be naturally modelled
by places of Petri nets. We propose asynchronously composed Petri nets (AIOPNs)
by (1) explicitely representing channels for internal communication inside the net and (2) defining
communication capabilities to the outside in terms of (open) input and output labels
with appropriate transitions. 
Then we define an asynchronous composition operator which introduces
new channels for the communication between the composed nets.
AIOPNs are equipped with a semantics in terms of
asynchronously composed I/O- transition systems (AIOTS).
We show that  this semantics is fully compositional, i.e.\  it commutes with asynchronous composition.

In our study two kinds of channel properties are considered which are related to
consumption requirements and to the termination of communication. Consumption
properties deal with requirements that messages sent to a communication channel
should also be consumed. They can be classified w.r.t.
two criteria. The first criterion is the nature of the
requirement: consuming messages, decreasing the number of messages on a channel,
and emptying channels. The second criterion expresses the way the requirement
is achieved: possibly immediately, possibly after some delay, or necessarily in each weakly fair run.
Communication termination deals with (immediate or delayed) closing of communication channels
when the receiver is not ready to consume any more.
%Management
%properties are related to implementation:
%boundedness of some channel, communication termination, etc.
We establish useful relations between the channel properties and prove
that all channel properties considered here are compositional, i.e. preserved
by asynchronous composition, which is an important prerequisite for incremental design.

From a verification point of view, we study the decidability
of properties in the framework of AIOPN. Thanks to several complementary
works on decidability for Petri net problems, we show that all
channel properties considered in this work are decidable, though with a high
computational complexity.

\noindent
{\bf Related work.} To the best of our knowledge, no work has considered
channels explicitely defined for communication inside composite components. However there have been several works 
where channels are associated with asynchronous composition. They can be
roughly classified depending whether their main feature is an 
algorithmic or a semantic one.

From an algorithmic point of view,
in the seminal work of~\cite{Brand:1983:CFM:322374.322380}, the authors discus several properties
like \emph{channel boundedness} and \emph{specified receptions} and propose
methods to analyse them. In~\cite{Cece:2005:VPH:1124524.1709549}, a two-component based
system is studied using a particular (decidable) channel property,
the \emph{half-duplex property}: at any time at most one channel
is not empty. 


From a semantic point view, in~\cite{hennicker2010} 
``connection-safe'' component 
assemblies have been studied incorporating both synchronous and asynchronous communication.
More recently in~\cite{BTO12} \emph{synchronisability},
a property of asynchronous systems, is introduced such that when it holds
the system can be safely abstracted by a synchronous one.
In the framework of Petri nets, E. Kindler has defined Petri net components
where the interface is composed by places and  composition consists in
merging places with same identities~\cite{Kindler97}. He proposed a
partial order semantics for such components and proved that the semantic
is fully compositional. Furthermore, for a restricted linear temporal logic
he established that properties of this logic are preserved by composition;
however such a logic cannot express some of the channel properties
we introduce here due to their branching kind.

The Petri net based formalism of open nets  is the closest formalism to ours.
Several works~\cite{Lohmann07},\cite{DK2009}, and \cite{Stahl12} address both the semantic point of view 
and the algorithmic one but only when the nets are assumed to be bounded which is not required
here. We postpone to Section~\ref{sec:aiopns-defs} a more detailed comparison with our formalism.

\noindent
{\bf Organisation.}
In Section~\ref{sec:AIOPN}, we introduce AIOPNs and their asynchronous composition.
Then, we provide a compositional semantic for AIOPNs in Section~\ref{sec:semantics} 
in terms of AIOTSs.
In Section~\ref{sec:communication}, we define the channel properties
and study their relationships and their preservation under asynchronous composition.
In Section~\ref{sec:decidability}, we establish that all channel properties
are decidable for AIOPNs. Finally, in Section~\ref{sec:conclusion}, we conclude and give some
perspectives for future work.
%!TEX root = ./main.tex
\subsection{Compositionality of Channel Properties}\label{subsec:comp-prop}

Modular verification of systems is an important goal in any development method.
In our context this concerns the question whether channel properties are preserved in arbitrary environments
or, more precisely, whether they are preserved under asynchronous composition.
In this section we show that indeed all channel properties defined above are compositional.
This can be utilised to get a method for incremental design.



In order to relate channel properties of asynchronous compositions to channel properties
of their constituent parts we need the next two lemmas.
The first one shows that autonomous executions of constituent parts (not involving inputs)
can be lifted to executions of compositions.
This is the essence to prove compositionality of the
properties of type (a) and type (b) in Def.~\ref{def:cp}.
%\footnote{This also justifies 
%why we did not allow inputs in the properties of type (a) and type (b) in Def.~\ref{def:cp}. In fact for any input action $a$ of a single AIOTS there is always an environment which will
%not serve the input and therefore $a$ will not induce a transition with communication label $\inact{a}$ in \emph{any} composition.}

\begin{lemma}\label{lem:reuse-sequence}
    Let $\S = \aiotstup{\S},\T = \aiotstup{\T}$ be two composable \AIOTSs, and let $\S\compos\T = \aiotstup{}$. For all $(q_\S,q_\T,\vec{v}) \in \post^*(q^0)$ and $\sigma \in (\Sigma_\S \setminus \inset_\S)^*$ it holds that
    \[
        q_\S \goes{\sigma}_\S q'_\S \implies \exists \vec{v'}\ .\ (q_\S,q_\T,\vec{v}) \goes{\bar\sigma} (q'_\S,q_\T,\vec{v'}),
    \]
    with $\bar\sigma \in (\Sigma \setminus \inset)^*$ obtained from $\sigma$ by replacing any occurrence of a shared label $a \in \outset_\S \cap \inset_\T$ by the communication label $\outact{a}$.
\end{lemma}
\begin{proof}
	Obviously it is sufficient to show the claim for an arbitrary $a \in (\Sigma_\S \setminus \inset_S)$. The general result then follows by induction on the length of $\sigma$.
    \begin{enumerate}
        \item$a \in \outset_\S \cap \inset_\T$: By rule \ref{def:trans:stots} in Def.~\ref{def:async-comp-aiots} it follows that there exists $\vec{v'}$ such that
                    $q_\S \goes{a}_\S q'_\S \implies (q_\S,q_\T,\vec{v}) \goes{\outact{a}} (q'_\S,q_\T,\vec{v'}).$
        \item$a \not\in \outset_\S \cap \inset_\T$: There are two subcases, either $a \in \com_\S$, or $a \in \outset_\S \setminus \inset_\T$. In both cases we get from rule \ref{def:trans:s} in Def.~\ref{def:async-comp-aiots}, \\
            $q_\S \goes{a}_\S q'_\S \implies (q_\S,q_\T,\vec{v}) \goes{a} (q'_\S,q_\T,\vec{v}) .$  
    \end{enumerate} 
\end{proof}

The next lemma is crucial to prove compositionality of the
``necessarily'' properties of type (c) in Def.~\ref{def:cp} and the communication stopping properties in Def.~\ref{def:rp}.
It shows that projections of weakly fair runs are weakly fair runs again.
This result can only be achieved in the asynchronous context.
%In a synchronous environment strongly fair runs would have to be considered.

\begin{lemma}\label{lem:projection-run}
    Let $\S$, $\T$ be two composable \AIOTSs, and $\S\compos\T = \aiotstup{}$. Let $q = (q_\S,q_\T,\vec{v}) \in Q$ and $\rho \in \frun{\S\compos\T}(q)$ be a weakly fair run. Then $\rho|_\S \in \frun{\S}(q_\S)$, is a weakly fair run.
\end{lemma}
\begin{proof}
	Let $\rho \in \frun{\S\compos\T}(q)$. First we show that $\rho|_\S$ is weakly fair as well. If $\rho|_\S$ is finite it is weakly fair. 
If $\rho|_\S$ is infinite, then assume that an autonomous action $a \in (\Sigma_\S \setminus in_\S)$ is always enabled in $\rho|_\S$ from a certain state on. Due to the asynchronous composition
we get that $a$ is also always enabled in $\rho$ from a certain state on. Hence $a$ will be infinitely often executed in $\rho$ and therefore also in $\rho|_\S$.

 It remains to prove that $\rho|_\S \in \run{\S}(q_\S)$. 
    There are two main cases. Either $\rho$ is finite or infinite. 
    Assume $\rho$ is finite with its last state being $q' = (q'_\S,q'_\T,\vec{v'})$. Clearly $q'_\S$ must be the last state of $\rho|_\S$. By definition of a run, $q'$ is a pure input state. Then $q'_\S$ must be a pure input state of $\S$, hence $\rho|_\S \in \run{\S}(q_\S)$.
    
    Now assume that $\rho$ is infinite and weakly fair. In this case there are two subcases. Either $\rho|_\S$ is finite or infinite. If $\rho|_\S$ is infinite, then $\rho|_\S \in \run{\S}(q_\S)$.

    Assume $\rho|_\S$ is finite such that its last state is $q'_\S$, with $(q'_\S,q'_\T,\vec{v'}) \in \rho$. Since $\rho$ is weakly fair we can prove that $q'_\S$ is a pure input state as follows. Assume the contrary, that $q'_\S$ is not a pure input state, i.e. there exists $a \in (\Sigma \setminus \inset)$ such that $q'_\S \goes{a}$. As $\rho$ is infinite we get that $a$ is enabled always from $(q'_\S,q'_\T,\vec{v'}) \in \rho$, which is a contradiction, since $\rho$ is fair and $a$ is not occurring infinitely often after $(q'_\S,q'_\T,\vec{v'}) \in \rho$ as $\rho|_\S$ is finite. Now, knowing that $q'_\S$ is a pure input state, we get that $\rho|_\S \in \run{\S}(q_\S)$.
\end{proof}

\begin{proposition}[Compositionality of Channel Properties]\label{prop:cp:compos}
   Let $\S$ and $\T$ be two composable \AIOTSs such that $\chan_\S$ is the set of channels of $\S$. Let $B \subseteq \chan_\S$ and let P be an arbitrary channel property as defined in Sec.~\ref{sec:def-communication-prop}.
   If $\S$ has property P with respect to the channels $B$, then $\S \compos \T$ has property P with respect to the channels $B$.
This holds analogously for asynchronous I/O-Petri nets (due to the compositional semantics of \AIOPNs; see Thm.~\ref{thm:comp-semantics}).
\end{proposition}
\begin{proof}
	Let $\S = \aiotstup{\S}$, $\T = \aiotstup{\T}$, $\S\compos\T = \aiotstup{}$ with $\ioalpha{\S}$ and $\ioalpha{}$.
    We split this proof into the following three parts: ``non-necessarily'' channel properties of types (a) and (b) in Def.~\ref{def:cp}, necessarily channel properties of type (c)  in Def.~\ref{def:cp}, and communication stopping properties in Def.~\ref{def:rp}.

    \smallskip\noindent {\bf Non-necessarily channel properties:}
    Before we prove each case we state two simple consequences of Lem.~\ref{lem:reuse-sequence}:
    For all $(q_\S,q_\T,\vec{v}) \in \post^*(q^0)$ and $q'_\S \in Q_\S$, 
    \begin{enumerate}[label=(\arabic*), ref=(\arabic*), leftmargin=*]
        \item\label{eq:poststar} 
        $q'_\S \in \post^*(q_\S,\Sigma_\S\!\setminus\inset_\S) \Rightarrow \exists \vec{v'}.(q'_\S,q_\T,\vec{v'}) \in \post^*((q_\S,q_\T,\vec{v}),\Sigma \setminus \inset)$
        \item\label{eq:post}$q'_\S \in \post(q_\S,\Sigma_\S\!\setminus\inset_\S) \;\,\Rightarrow \exists \vec{v'}.(q'_\S,q_\T,\vec{v'}) \in \post((q_\S,q_\T,\vec{v}),\Sigma \setminus \inset)$
    \end{enumerate}

For the proof of the P1 - P3 properties let $(q_\S,q_\T,\vec{v}) \in \post^*(q^0)$ and $a \in B$ such that $\val((q_\S,q_\T,\vec{v}),a) > 0$. 
    \begin{enumerate} [label=P1.\alph*:,leftmargin=2.7em]       
        \item Assume that $\S$ is $B$-consuming. 
        Obviously $q_\S \in \post^*(q^0_\S)$. By assumption there exists $q'_\S \in \post^*(q_\S,\Sigma_\S \setminus \inset_{\S})$ such that $q'_\S \goes{\inact{a}}_\S$. By \ref{eq:poststar} above  there exists $\vec{v'}$ such that $(q'_\S,q_\T,\vec{v'}) \in \post^*((q_\S,q_\T,\vec{v}),\Sigma \setminus \inset)$, and by definition of $\compos$ we get $(q'_\S,q_\T,\vec{v'})\goes{\inact{a}}$. 

        \item If $\S$ is strongly $B$-consuming, we get by assumption that $q_\S \goes{\inact{a}}$. By definition of $\compos$ this means that $(q_\S,q_\T,\vec{v})\goes{\inact{a}}$.
    \end{enumerate}
    \begin{enumerate} [label=P2.\alph*:,leftmargin=2.7em]   
        \item Assume that $\S$ is $B$-decreasing. 
        Obviously $q_\S \in \post^*(q^0_\S)$. By assumption there exists $q'_\S \in \post^*(q_\S,\Sigma_\S \setminus \inset_{\S})$ such that $\val_\S(q'_\S,a) < \val_\S(q_\S,a)$. By \ref{eq:poststar} there exists $\vec{v'}$ such that $(q'_\S,q_\T,\vec{v'}) \in \post^*((q_\S,q_\T,\vec{v}),\Sigma \setminus \inset)$. Finally as $a \in\chan_\S$ and $\val_\S(q'_\S,a) < \val_\S(q_\S,a)$ we get, by definition of $val$, that $\val((q'_\S,q_\T,\vec{v'}),a) < \val((q_\S,q_\T,\vec{v}),a)$. 

        \item If $\S$ is strongly $B$-decreasing the claim follows by the same reasoning using \ref{eq:post} from above.
    \end{enumerate}
    \begin{enumerate} [label=P3.\alph*:,leftmargin=2.7em]  
        \item The proof is analogous to the case P2.a.
        \item The proof is analogous to the case P2.b.
    \end{enumerate}
For the proof of P4  let $(q_\S,q_\T,\vec{v}) \in \post^*(q^0)$ such that $\val((q_\S,q_\T,\vec{v}),B) > 0$. 
    \begin{enumerate} [label=P4.\alph*:,leftmargin=2.7em]  
        \item Assume that $\S$ is $B$-wholly emptying.
        Obviously $q_\S \in \post^*(q^0_\S)$. By assumption there exists $q'_\S \in \post^*(q_\S,\Sigma_\S \setminus \inset_{\S})$ such that $\val_\S(q'_\S,B) = 0$. By \ref{eq:poststar} there exists $\vec{v'}$ such that $(q'_\S,q_\T,\vec{v'}) \in \post^*((q_\S,q_\T,\vec{v}),\Sigma \setminus \inset)$. Finally as $B \subseteq \chan_\S$ we get, by definition of $val$, that $\val((q'_\S,q_\T,\vec{v'}),B) = \val_\S(q'_\S,B)=0$.

        \item If $\S$ is strongly $B$-wholly emptying the claim follows by the same reasoning using \ref{eq:post} from above.
    \end{enumerate}

    \smallskip\noindent {\bf Necessarily channel properties:}
    We will only prove the claim for property \ref{def:cp:nec-consuming}, as it can be proven analogously for the other necessarily properties.

    Assume that $\S$ is necessarily $B$-consuming. Let $\S\compos\T = \aiotstup{}$, $(q_\S,q_\T,\vec{v}) \in \post^*(q^0)$ and $a \in B$, such that $\val((q_\S,q_\T,\vec{v}),a)>0$. Let \linebreak $\rho \in \frun{\S\compos\T}((q_\S,q_\T,\vec{v}))$ be weakly fair. 
 By Lem.~\ref{lem:projection-run} we get that $\rho|_\S$ is a weakly fair run of $\S$. By assumption we know $\inact{a} \in \rho|_\S$ and hence it follows that $\inact{a} \in \rho$.

    \smallskip\noindent {\bf Communication stopping properties:}
    We will only prove the claim for property \ref{def:rp:bas-stopping}, it can be proven analogously for \ref{def:rp:str-stopping}.

   Assume that $\S$ is $B$-communication stopping. Let $\S\compos\T = \aiotstup{}$, $(q_\S,q_\T,\vec{v}) \in \post^*(q^0)$, $a \in B$ and $\rho \in \frun{\S\compos\T}((q_\S,q_\T,\vec{v}))$, such that $\sharp_\rho(\inact{a}) = 0$. 
 By Lem.~\ref{lem:projection-run} we get that $\rho|_\S \in \frun{\S}(q_\S)$. By assumption we know that $\sharp_{\rho|_\S}(\outact{a}) < \infty$. Since $\outact{a}$ is triggered by $\S$ in $\S\compos\T$, we get that $\sharp_\rho(\outact{a}) < \infty$.
\end{proof}
%As a consequence of the compositional semantics of \AIOPNs (Thm.~\ref{thm:comp-semantics}), Proposition~\ref{prop:cp:compos} holds analogously for \AIOPNs:
%
%\begin{corollary}[Compositionality of Channel Properties for AIOPNs]\label{cor:cp:compos}
%   Let $\pnN$ and $\M$ be two composable \AIOPNs such that $\chan_\pnN$ is the set of channels of $\pnN$. Let $B \subseteq \chan_\pnN$ and let P be an arbitrary channel property as defined in Sec.~\ref{sec:def-communication-prop}.
%If $\pnN$ has property P with respect to the channels $B$, then $\pnN \compospn \M$ has property P with respect to the channels $B$.
%\end{corollary}

Proposition~\ref{prop:cp:compos} leads to the desired modular verification result for all properties except
wholly emptying~\ref{def:cp:wholly-emptying}:
In order to check that a composition $\pnN \compospn \M$ of two \AIOPNs has a channel property $P$, i.e. $P$ holds for all channels of the composition,
it is sufficient to know that $\pnN$ and $\M$ have property $P$ and to prove that
$\pnN \compospn \M$ has property $P$ with respect to the new channels introduced by the asynchronous composition.

\begin{theorem}[Incremental Design]\label{thm:inc-design}
 Let $\pnN$ and $\M$ be two composable\linebreak \AIOPNs with shared actions  $\Sigma_\pnN \cap \Sigma_\M$
and let  $P$ be an arbitrary channel property but~\ref{def:cp:wholly-emptying}.
If both $\pnN$ and $\M$ have property $P$ and if $\pnN \compospn \M$ has property $P$ with respect to the new channels
$\Sigma_\pnN \cap \Sigma_\M$, then $\pnN \compospn \M$ has property $P$.
\end{theorem}

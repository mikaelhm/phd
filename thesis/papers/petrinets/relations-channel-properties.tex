%!TEX root = ./main.tex
\subsection{Relationships Between Channel Properties}\label{sec:relations}



Table~\ref{tab:prop-rel} shows relationships between the channel properties and pointers to examples of \AIOPNs from Fig.~\ref{fig:ex-aiopns} and Fig.~\ref{fig:counterex} which have the indicated properties.

%\vspace{-4mm}
%!TEX root = ../main.tex

\begin{table}[ht]
\small
\begin{center}
   \begin{tabular}{ccccc}
     % \hline
     % Necessarily & & Simple \\ \hline 
     \multicolumn{1}{l}{b)} &   & 
        \multicolumn{1}{l}{c)} &  & 
            \multicolumn{1}{l}{a)}  \\   

     \hhline{|-|~|-|~|-|}

     \multicolumn{1}{|c|}{{\cellcolor{lightlightblue}}Strongly wholly emptying} & $\stackrel{4}{\Rightarrow}$  &  
        \multicolumn{1}{|c|}{{\cellcolor{lightlightblue}}Necessarily wholly emptying}& $\stackrel{11}{\Rightarrow}$ &
            \multicolumn{1}{|c|}{{\cellcolor{lightlightblue}}Wholly emptying} \\
     
     \multicolumn{1}{|c|}{{\cellcolor{lightlightblue}}} &   &  
        \multicolumn{1}{|c|}{{\cellcolor{lightlightblue}}}&  &
            \multicolumn{1}{|c|}{{\cellcolor{lightlightblue}}(e.g. $\pnN_3,\pnN_4,\pnN_6$)} \\
     
     \multicolumn{1}{|c|}{{\cellcolor{lightlightblue}}$\Downarrow^1$}           &   &      
        \multicolumn{1}{|c|}{{\cellcolor{lightlightblue}}$\Downarrow^{8}$} & & 
            \multicolumn{1}{|c|}{{\cellcolor{lightlightblue}}$\Downarrow^{15}$}\\ 
   
     \multicolumn{1}{|c|}{{\cellcolor{lightlightblue}}Strongly emptying}        & $\stackrel{5}{\Rightarrow}$  &
        \multicolumn{1}{|c|}{{\cellcolor{lightlightblue}}Necessarily emptying}& $\stackrel{12}{\Rightarrow}$ & 
            \multicolumn{1}{|c|}{{\cellcolor{lightlightblue}}Emptying}  \\
     
     \multicolumn{1}{|c|}{{\cellcolor{lightlightblue}}} &   &  
        \multicolumn{1}{|c|}{{\cellcolor{lightlightblue}}}&  &
            \multicolumn{1}{|c|}{{\cellcolor{lightlightblue}}(e.g. $\pnN_3,\pnN_4,\pnN_5,\pnN_6$)} \\

     \multicolumn{1}{|c|}{{\cellcolor{lightlightblue}}$\Downarrow^2$}           &   &     
        \multicolumn{1}{|c|}{{\cellcolor{lightlightblue}}$\Updownarrow^{9}$} & & 
            \multicolumn{1}{|c|}{{\cellcolor{lightlightblue}}$\Updownarrow^{16}$}\\ 
    
     \multicolumn{1}{|c|}{{\cellcolor{lightlightblue}}Strongly decreasing}      & $\stackrel{6}{\not\Rightarrow}$   &    
        \multicolumn{1}{|c|}{{\cellcolor{lightlightblue}}Necessarily decreasing} & $\stackrel{13}{\Rightarrow}$ & 
            \multicolumn{1}{|c|}{{\cellcolor{lightlightblue}}Decreasing}  \\
     
     \multicolumn{1}{|c|}{{\cellcolor{lightlightblue}}(e.g. $\pnN_4$)} &   &  
        \multicolumn{1}{|c|}{{\cellcolor{lightlightblue}}}&  &
            \multicolumn{1}{|c|}{{\cellcolor{lightlightblue}}(e.g. $\pnN_3,\pnN_4,\pnN_5,\pnN_6$)} \\

     \multicolumn{1}{|c|}{{\cellcolor{lightlightblue}}$\Updownarrow^3$}         &   &
        \multicolumn{1}{|c|}{{\cellcolor{lightlightblue}}$\Downarrow^{10}$} & & 
            \multicolumn{1}{|c|}{{\cellcolor{lightlightblue}}$\Downarrow^{17}$}\\ 
   
     \multicolumn{1}{|c|}{{\cellcolor{lightlightblue}}Strongly consuming}       &  $\stackrel{7}{\Rightarrow}$  &    
        \multicolumn{1}{|c|}{{\cellcolor{lightlightblue}}Necessarily consuming} & $\stackrel{14}{\Rightarrow}$ & 
            \multicolumn{1}{|c|}{{\cellcolor{lightlightblue}} Consuming}  \\
     
     \multicolumn{1}{|c|}{{\cellcolor{lightlightblue}}(e.g. $\pnN_4$)} &   &  
        \multicolumn{1}{|c|}{{\cellcolor{lightlightblue}}(e.g. $\pnN_3,\pnN_4,\pnN_5$)}&  &
            \multicolumn{1}{|c|}{{\cellcolor{lightlightblue}}(e.g. $\pnN_3,\pnN_4,\pnN_5,\pnN_6$)} \\

    \hhline{|-|~|-|~|-|}

    &   &  \multicolumn{1}{c}{\rule{0pt}{3.3ex}$\Downarrow^{18}$} & & 
            \\ 
    
    \hhline{|~|~|-|~|~|}                
            

    &   &   
        \multicolumn{1}{|c|}{{\cellcolor{lightlightblue}}Strongly com. stopping} & & 
            \\

    &   &   
        \multicolumn{1}{|c|}{{\cellcolor{lightlightblue}}(e.g. $\pnN_3,\pnN_4,\pnN_5$)} & & 
            \\

    &   &     
        \multicolumn{1}{|c|}{{\cellcolor{lightlightblue}}$\Downarrow^{19}$} & & 
            \\ 

    &   &  
        \multicolumn{1}{|c|}{{\cellcolor{lightlightblue}}Com. stopping} & & 
            \\
    &   &   
        \multicolumn{1}{|c|}{{\cellcolor{lightlightblue}}(e.g. $\pnN_3,\pnN_4,\pnN_5$)} & & 
            \\
    \hhline{|~|~|-|~|~|}                 
    \end{tabular}
\vspace{2mm}
\caption{Relationships between channel properties and examples.}\label{tab:prop-rel}
 \end{center}
    
\end{table} %table file


%\vspace{-10mm}
All the downward implications inside the boxes are direct consequences of the definitions. It is trivial to see that downward implication 3 is an equivalence, since \emph{immediate} consumption leads to a decreasing valuation. Downward implications 9 and 16 are equivalences, since repeated decreasing of messages on a channel will eventually lead to an empty channel.
The implications 4, 5 and 7 will be proved in Prop.~\ref{prop1}, the implications 11-14 in Prop.~\ref{prop2}, and implication 18 in Prop.~\ref{prop3}.
Additionally we have that all properties in box b) of Tab.~\ref{tab:prop-rel} imply the strongest property in box a), since if $\S$ is strongly $B$-consuming we can by repeated consumption empty all channels in $B$.

In the following propositions we assume given  an \AIOTS $\S = \aiotstup{}$ and a subset $B \subseteq \chan$.
%and implication 20 in  Prop.~\ref{prop4}.

\begin{proposition}\label{prop1}
 If $\S$ is  strongly $B$-wholly emptying (strongly $B$-emptying,\linebreak strongly $B$-consuming resp.), then $\S$ is necessarily $B$-wholly emptying (necessarily $B$-emptying, necessarily $B$-consuming resp.).
\end{proposition}
\begin{proof}
    We only prove that strongly consuming implies necessarily consuming. The other implications are simple extensions of this proof.
 Let $q \in \post^*(q^0)$, $a \in B$ such that $\val(q,a) > 0$, and let $\rho \in \run{\S}(q)$ be a weakly fair run. Assume for the contrary that $\inact{a} \not\in \rho$. By definition of \AIOTS we get that for all $q' \in \rho$, $\val(q',a), \geq \val(q,a)$. By assumption $\S$ is strongly $B$-consuming, which implies that $q' \goes{\inact{a}}$ for all $q' \in \rho$. This is a contradiction to $\rho$ being a weakly fair run. 
\end{proof}

\begin{proposition}\label{prop2}
 If $\S$ is necessarily $B$-wholly emptying (necessarily\linebreak $B$-emptying, necessarily decreasing, necessarily $B$-consuming resp.), then $\S$ is $B$-wholly emptying ($B$-emptying, $B$-decreasing, $B$-consuming resp.).
\end{proposition}
\begin{proof}
    The proof relies on the fact that for each $q \in \post^*(q^0)$ there exists a weakly fair run $\rho \in \frun{\S}(q)$, such that for all $a \in \inset, a \not\in \rho$. This run can be constructed by choosing, in a weakly fair manner in each reached state, some enabled non-input action. If no such action is enabled in the last visited state, the last state is a pure input state and we are done. Otherwise the resulting infinite run has the required property.

    With this fact, we can prove the implication 14 in Tab.~\ref{tab:prop-rel} as follows: Let $q \in \post^*(q^0)$, $a \in B$ such that $\val(q,a) > 0$. By necessarily consuming, for all $\rho \in \frun{\S}(q)$ we have $\inact{a} \in \rho$. Since we know there exists a weakly fair run $\rho$ without input actions, we get that there exists $q' \in \post^*(q,\Sigma \setminus \inset)$ such that $q'\goes{\inact{a}}$.
    The other implications are proven in the same way. 
\end{proof}

\begin{proposition}\label{prop3}
 If $\S$ is necessarily $B$-consuming then $\S$ is strongly \linebreak $B$-communication stopping.
\end{proposition}
\begin{proof}
    Assume the contrary, that $\S$ is not strongly $B$-communication stopping. This means that there exists $q \in \post^*(q^0) $, $a \in B$ and a weakly fair run $\rho \in \frun{\S}(q)$ such that $\sharp_\rho(\inact{a}) = 0$ and $\sharp_\rho(\outact{a}) > 0$. Then there exists $q' \in \rho$ reached after $q$, such that $\val(q',a) > 0$. Let $\rho'$ be the suffix of $\rho$ starting from $q'$ which is weakly fair as well. Since $\sharp_{\rho'}(\inact{a}) = 0$, $\S$ is not necessarily consuming.
\end{proof}

Let us now discuss some counterexamples.  As discussed in Sect.~\ref{sec:def-communication-prop}, a counterexample for the converse of implication 7 is the \AIOPN $\pnN_3$ in Fig.~\ref{fig:ex-aiopn} on page~\pageref{fig:ex-aiopn}.
An obvious counterexample for the converse of the implications 2, 10, 11, 12, 13 is given by the \AIOPN $\pnN_4$ shown in Fig.~\ref{fig:counterex1}.
$\pnN_4$ is also a counterexample for implication 6.
The \AIOPN $\pnN_5$  in Fig.~\ref{fig:counterex2} with channels $a$ and $b$ is a counterexample for the converse of implication 15. The net can empty each single channel $a$ and $b$ but it can never have both channels empty at the same time
(after the first message has been produced on a channel).
A counterexample for the converse of implication 14 is shown by the net $\pnN_6$ in Fig.~\ref{fig:counterex14}. The net can put a token on the channel $a$, but afterwards the transition $\inact{a}$ is not necessarily always enabled which means there exists a weakly fair run such that there is always a token in $a$ and $\inact{a}$ is never fired.

Counterexamples for the converse of implication 17 rely on the idea to produce twice while consuming once.
A counterexample for the converse of implication 18 is provided by a net that first produces a finite number $n$ of messages on a channel, then it consumes less than $n$ of these messages and then it stops. Counterexamples for the remaining converse implications are straightforward to construct.

 
\begin{figure}[ht]
\centering
% \subfloat[Counterexample 1.]{\label{fig:counterex1}%!TEX root = /Users/mikaelhm/projects/PNInterfaceTheory/main.tex
\begin{tikzpicture}[scale=0.675, every node/.style={transform shape}]

\node[place,tokens=1] at (-1,0) (p0) {};
\node[transition,rotate=0,label={above:$\outact{a}$}] at (0,0) (toa) {};
\node[place,tokens=0,label=above:$a$] at (2,0) (pa) {};
\node[transition,rotate=0,label={above:$\inact{a}
$}] at (4,0) (tia) {};

\draw[arc] (p0) to (toa) {};
\draw[arc] (toa) to[] (pa) {};
\draw[arc] (pa) to[] (tia) {};




\node[place,tokens=1] at (-1,-1) (p1) {};
\node[transition,rotate=0,label={below:$\outact{b}$}] at (0,-1) (tob) {};
\node[place,tokens=0,label=below:$b$] at (2,-1) (pb) {};
\node[transition,rotate=0,label={below:$\inact{b}
$}] at (4,-1) (tib) {};

\draw[arc] (p1) to (tob) {};
\draw[arc] (tob) to[] (pb) {};
\draw[arc] (pb) to[] (tib) {};
\end{tikzpicture} }
\subfloat[$\pnN_4$]{\label{fig:counterex1}%!TEX root = /Users/mikaelhm/projects/PNInterfaceTheory/main.tex
\begin{tikzpicture}
% \node[place,tokens=0] at (-2,0) (p0) {};
\node[transition,rotate=0,label={above:$\outact{a}$}] at (0,0) (toa) {};
\node[place,tokens=0,label=above:$a$] at (2,0) (pa) {};
\node[transition,rotate=0,label={above:$\inact{a}
$}] at (4,0) (tia) {};

\node[place,white] at (2,-1.4) (pb) {};


% \draw[arc] (p0) to (toa) {};
\draw[arc] (toa) to[] (pa) {};
\draw[arc] (pa) to[] (tia) {};
\begin{pgfonlayer}{background}
\node [compBackground,inner sep=6mm,fit=(toa)] {};
\node [compBackground,inner sep=6mm,fit=(tia)] {};
\end{pgfonlayer}
\end{tikzpicture} }\quad
% \subfloat[Counterexample 3.]{\label{fig:counterex1}%!TEX root = /Users/mikaelhm/projects/PNInterfaceTheory/main.tex
\begin{tikzpicture}[scale=0.675, every node/.style={transform shape}]
% \node[place,tokens=0] at (-2,0) (p0) {};
\node[transition,rotate=0,label={above:$\outact{a}$}] at (0,0) (toa) {};
\node[place,tokens=0,label=above:$a$] at (2,0) (pa) {};
\node[transition,rotate=0,label={above:$\inact{a}$}] at (4,0) (tia) {};

% \draw[arc] (p0) to (toa) {};
\draw[arc] (toa) to[] (pa) {};
\draw[arc] (pa) to[] (tia) {};

\node[place,tokens=1] at (4,-1) (p0) {};
\node[transition,rotate=0,label={right:$\tau$}] at (5,-1) (ttau) {};
\node[place,tokens=0] at (5,0) (p1) {};

\draw[arc] (tia) to[] (p0) {};
\draw[arc] (p0) to[] (ttau) {};
\draw[arc] (ttau) to[] (p1) {};
\draw[arc] (p1) to[] (tia) {};

\end{tikzpicture} }\\
% \subfloat[Counterexample 4.]{\label{fig:counterex1}%!TEX root = /Users/mikaelhm/projects/PNInterfaceTheory/main.tex
\begin{tikzpicture}[scale=0.675, every node/.style={transform shape}]


\node[place,tokens=1] at (-1,-2) (p0) {};


\node[transition,rotate=0,label={above:$\outact{a}$}] at (0,0) (toa) {};
\node[place,tokens=0] at (0,-1) (p1) {};
\node[transition,rotate=0,label={right:$\outact{a}$}] at (0,-2) (toa2) {};
\node[place,tokens=0] at (0,-3) (p2) {};
\node[transition,rotate=0,label={below:$\outact{b}$}] at (0,-4) (tob) {};


\node[place,tokens=0,label=above:$a$] at (2,0) (pa) {};
\node[place,tokens=0,label=above:$b$] at (2,-4) (pb) {};

\node[transition,rotate=0,label={above:$\inact{a}$}] at (4,0) (tia) {};
\node[place,tokens=0] at (4,-2) (p3) {};
\node[transition,rotate=0,label={below:$\inact{b}$}] at (4,-4) (tib) {};



\draw[arc] (p0) to[bend left] (toa) {};
\draw[arc] (toa) to[] (p1) {};
\draw[arc] (p1) to[] (toa2) {};
\draw[arc] (toa2) to[] (pa) {};
\draw[arc] (toa2) to[] (p2) {};
\draw[arc] (p2) to[] (tob) {};
\draw[arc] (tob) to[bend left] (p0) {};

\draw[arc] (toa) to[] (pa) {};
\draw[arc] (pa) to[] (tia) {};
\draw[arc] (tob) to[] (pb) {};
\draw[arc] (pb) to[] (tib) {};


\draw[arc] (tib) to[] (p3) {};
\draw[arc] (p3) to[] (tia) {};

% \draw[arc] (p0) to[] (tia) {};



\end{tikzpicture} }
\subfloat[$\pnN_5$]{\label{fig:counterex2}%!TEX root = /Users/mikaelhm/projects/PNInterfaceTheory/main.tex
\begin{tikzpicture}


\node[place,tokens=2] at (-1,-1) (p0) {};


\node[transition,rotate=0,label={above:$\outact{a}$}] at (0,0) (toa) {};
\node[place,tokens=0] at (0,-1) (p1) {};
\node[transition,rotate=0,label={below:$\inact{b}$}] at (0,-2) (tib) {};


\node[place,tokens=0,label=above:$a$] at (2,0) (pa) {};
\node[place,tokens=0,label=above:$b$] at (2,-2) (pb) {};

\node[transition,rotate=0,label={above:$\inact{a}$}] at (4,0) (tia) {};
\node[place,tokens=0] at (4,-1) (p3) {};
\node[transition,rotate=0,label={below:$\outact{b}$}] at (4,-2) (tob) {};
\node[place,tokens=1] at (5,-1) (p4) {};


\draw[arc] (p0) to[bend left] (toa) {};
\draw[arc] (toa) to[] (p1) {};
\draw[arc] (p1) to[] node[right,midway] {2} (tib) {};
\draw[arc] (tib) to[bend left] node[auto,midway] {2} (p0) {};

\draw[arc] (toa) to[] (pa) {};
\draw[arc] (pa) to[] (tia) {};
\draw[arc] (tob) to[] (pb) {};
\draw[arc] (pb) to[] (tib) {};


\draw[arc] (tob) to[] (p3) {};
\draw[arc] (p3) to[] (tia) {};
\draw[arc] (tia) to[bend left] (p4) {};
\draw[arc] (p4) to[bend left] (tob) {};
% \draw[arc] (p0) to[] (tia) {};

\begin{pgfonlayer}{background}
\node [compBackground,inner sep=6mm,fit=(p0)(toa)(tib)] {};
\node [compBackground,inner sep=6mm,fit=(tia)(tob)(p4)] {};
\end{pgfonlayer}

\end{tikzpicture} }\\
\subfloat[$\pnN_6$]{\label{fig:counterex14}%!TEX root = ../main.tex
\begin{tikzpicture}
% \node[place,tokens=0,label=above:$p_0$] at  (0,0) (p0) {};
\node[transition,label={[yshift=0,xshift=0]90:$\outact{a}$}] at (1.5,0) (ta_out) {};
% \node[place,tokens=1,label=below:$p_1$] at (0,-2) (p1) {};
% \node[transition,rotate=90,label={[yshift=3,xshift=-6.5]180:$in?$}] at (-1,-1) (t_in) {};
% \node[transition,label={[yshift=0,xshift=0]-90:$\outact{b}$}] at (1.5,-2) (tb_out) {};

\node[transition,label={[yshift=0,xshift=0]90:$\inact{a}$}] at (6.5,0) (ta_in) {};
\node[place,tokens=1,label=below:$p_0$] at (8,0) (p2) {};
\node[transition,label={[yshift=0,xshift=1]90:$out!$}] at (9,.5) (t_out) {};
\node[transition,label={[yshift=0,xshift=1]-90:$out!$}] at (9,-.5) (t_out2) {};
\node[place,tokens=0,label=below:$p_1$] at (10,0) (p3) {};

\node[place,label=below:$a$] at (4,0) (pa) {};
% \node[place,label=below:$b$] at (4,-2) (pb) {};

% \draw[arc] (p0) to (ta_out) {};
% \draw[arc] (ta_out) to (p1) {};
% \draw[arc] (p1) to[bend left] (t_in) {};
% \draw[arc] (t_in) to[bend left] (p0) {};
% \draw[arc] (p1) to[bend left] (tb_out) {};
% \draw[arc] (tb_out) to[bend left] (p1) {};

\draw[arc] (ta_out) -- (pa) {};
\draw[arc] (pa) -- (ta_in) {};
% \draw[arc] (tb_out) -- (pb) {};
% \draw[arc] (pb) -- (tb_in) {};

\draw[arc] (ta_in) to[bend left] (p2) {};
\draw[arc] (p2) to[bend left] (ta_in) {};
\draw[arc] (p2) to (t_out2) {};
\draw[arc] (t_out2) to (p3) {};
\draw[arc] (p3) to (t_out) {};
\draw[arc] (t_out) to (p2) {};
\node at (1.55,0) (tmsg_outl) {};
\node at (5.05,0) (tmsg_inl) {};
\node at (7.95,0) (ta_outl) {};
\node at (-1.45,0) (ta_inl) {};
\begin{pgfonlayer}{background}
\node [compBackground,inner sep=6mm,fit=(ta_out)] {};
\node [compBackground,inner sep=6mm,fit=(ta_in)(t_out)(t_out2)(p3)] {};
\end{pgfonlayer}
\end{tikzpicture}}
\caption{Examples of AIOPNs.}%
\label{fig:counterex}%
\end{figure}
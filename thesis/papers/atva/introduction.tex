\section{Introduction}
The  specification  formalisms   of  Modal  Transition  Systems  (MTS)
\cite{DBLP:conf/lics/LarsenT88,AHLNW:EATCS:08} grew out of a series of
attempts   to  achieve  a   flexible  and   easy-to-use  compositional
development methodology  for reactive systems.  In  fact the formalism
of   MTS   may   be   seen   as   a   fragment   of  a 
temporal   logic~\cite{DBLP:journals/tcs/BoudolL92},   
while   having   a   behavioural
semantics  allowing  for an easy  composition with  respect  to  process
constructs.

In short, MTS are labelled  transition systems equipped with two types
of  transitions: \emph{must}  transitions which are mandatory  for any
implementation, and  \emph{may} transitions  which are optional  for an
implementation.   Refinement  of an   MTS  now  essentially  consists of
iteratively resolving the unsettled  status of may transitions: either
by removing them or by turning them into must transitions.

It is  well admitted (see e.g. \cite{RBBCP:ACSD:09}) that MTS  and their
extensions              like              disjunctive              MTS
(DMTS)~\cite{DBLP:conf/lics/LarsenX90}, %,fossacs-techrep},
1-selecting MTS  (1MTS)~\cite{FS:JLAP:08} and transition  systems with
obligations   (OTS)~\cite{benes_et_al:OASIcs:2011:3070}   provide  strong
support  for a specification  formalism allowing  for step-wise refinement
process.   Moreover,  the MTS  formalisms  have  applications in  other
contexts,   which   include   verification  of   product   lines
\cite{DBLP:conf/fmoods/GrulerLS08,LarsenNW07},
interface theories~\cite{UC:FSE:04,RBBCP:ACSD:09}
and
modal abstractions in 
program analysis~\cite{GHJ:CONCUR:01,HJS:ESOP:01,NNN:SAS:08}.
%                       and
%conter-example-guided-abstraction-refinement  techique  for transition
%systems.

Unfortunately, all  of these formalisms lack the  capability to express
some intuitive specification  requirements like exclusive, conditional
and  persistent choices.  In this  paper we  extend considerably the
expressiveness of MTS and its variants so that it can model
arbitrary Boolean conditions on transitions and also allows to instantiate persistent
transitions. Our model, called \emph{parametric modal transition systems} (PMTS),
is equipped with a finite set of parameters that
are fixed prior to the instantiation of the transitions in the specification.
The generalized notion of modal refinement is designed to handle the parametric
extension and it specializes to
the well-studied modal refinements on all the subclasses of our model
like
MTS, disjunctive MTS and MTS with obligations.

To the best of our knowledge, this is the first sound attempt to introduce
persistence into a specification formalism based on modal transition systems.
The most related work is by Fecher and Schmidt 
on 1-selecting MTS~\cite{FS:JLAP:08} where the authors
allow to model exclusive-or and briefly mention the desire %outline a possibility 
to extend the formalism with persistence. However, 
as in detail explained in the apendix of
% %Appendix~\ref{ap:mts}, 
\cite{pmts-techrep},
their definition does not capture the %intended 
notion of persistence.
Our formalism is in several aspects semantically more general and handles persistence
in a complete and uniform manner.

The main technical contribution, apart from the formalism itself, 
is a comprehensive complexity   characterization   of modal refinement 
checking on all of the practically relevant subclasses of PMTS. 
We show that the complexity ranges from P-completeness to 
$\PiP_4$-completeness, depending on the requested generality of the PMTS specifications 
on the left-hand and right-hand sides.

%*** someone please complete this -- is it the work of Harald Fecher?**
%For  related  work  we  mention  \cite{}  that  attempt  to  introduce
%persistence, but lack ... with  technical problems and do not allow as
%general expressiveness as our PMTS. 


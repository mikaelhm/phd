\subsection{Definition of Parametric Modal Transition System}

We shall now formally capture the intuition behind parametric MTS 
introduced above.
%in the previous section. 
First, we recall the standard
propositional logic.
 
%We start with a description of the formulae used for the specification of 
%the required behaviour.

%\begin{definition}
A~Boolean formula over a set $X$ of atomic propositions is given by
the following abstract syntax 
\[ 
\varphi ::= \true \mid x \mid \neg \varphi \mid \varphi \wedge \psi 
\mid \varphi \vee \psi \]
where $x$ ranges over $X$. 
The set of all Boolean formulae over the set $X$
is denoted by $\B(X)$. 
%The satisfaction relation is given as follows.
Let $\Valid \subseteq X$ be a truth assignment, 
i.e.~a~set of variables with value true, then the satisfaction relation
$\Valid \models \varphi$ is given by
%\begin{align*}
$\Valid \models \true$, $\Valid \models x$ iff $x \in \Valid $,
and the satisfaction of the remaining Boolean connectives is defined in the standard way.
%$\Valid X \models \neg \varphi$ if $\Valid X \not\models \varphi$,
%$\Valid X \models \varphi \wedge \psi$ if $\Valid X \models \varphi$
%and
%$\Valid X \models \psi$, and 
%$\Valid X \models \varphi \vee \psi$ if $\Valid X \models \varphi$
%or $\Valid X \models \psi$.
%\end{align*}
We also use the standard derived operators like
exclusive-or  
$\varphi \oplus \psi = (\varphi \wedge \neg \psi) \vee (\neg \varphi 
\wedge \psi)$, implication
$\varphi \Rightarrow \psi = \neg \varphi \vee \psi$ and equivalence $\varphi \Leftrightarrow \psi = (\neg \varphi \vee \psi)\wedge(\varphi \vee \neg \psi)$.
%\end{definition}

We can now proceed with the definition of parametric MTS.

\begin{definition}
A~\emph{parametric MTS} (PMTS) over an action alphabet $\USigma$ is 
a~tuple $(S,T,P,\UPhi)$ such that
	\begin{itemize}
	\item 
$S$ is a~set of \emph{states},
\item 
$T \subseteq S \times \USigma \times S$ is 
a~\emph{transition relation}, 
\item 
$P$ is a~finite set of \emph{parameters}, and
\item 
$\UPhi : S \to \mathcal B((\USigma \times S) \cup P)$ is a satisfiable
\emph{obligation function} over the atomic propositions containing 
outgoing transitions and parameters. 
\end{itemize}
We implicitly assume that whenever $(a,t) \in \UPhi(s)$ then 
$(s,a,t) \in T$.
By $T(s) = \{(a,t) \mid (s,a,t) \in T\}$ we denote the set of 
all outgoing transitions of $s$.
\end{definition}

% In what follows, 
We recall the agreement that whenever the obligation function for some node 
is not listed in the system description then it is implicitly understood as 
$\UPhi(s) = \bigwedge T(s)$, with the empty conjunction being $\true$.

We call a~PMTS \emph{positive} if, for all $s \in S$, any negation 
occurring in $\UPhi(s)$ is applied only to a parameter. 
A~PMTS is called \emph{parameter-free} if $P = \emptyset$.
We can now instantiate the previously studied specification formalisms
as subclasses of PMTS.

\begin{definition}
A PMTS is called
\begin{itemize}
\item \emph{transition system with obligation} (OTS)
if it is parameter-free and positive,
\item
\emph{disjunctive modal transition system} (DMTS)
if it is an OTS and $\UPhi(s)$ is in the 
conjunctive normal form for all $s \in S$,
% (conjunction of clauses, each of them being a disjunction of literals),
\item 
\emph{modal transition system} (MTS)
if it is a~DMTS and $\UPhi(s)$ is a~conjunction of positive literals 
(transitions) for all $s \in S$, and
\item \emph{implementation} (or simply a labelled
transition system) if it is an MTS and 
$\UPhi(s) = \bigwedge T(s)$ for all $s \in S$. 
\end{itemize}
\end{definition}

%In the case of DMTS and OTS, this definition yields the so 
%called \emph{syntactically consistent} variants.
%Further, 

Note that positive PMTS, despite the absence of a general negation and the
impossibility to define for example exclusive-or, can still express 
useful requirements like 
$\UPhi(s)=p\Rightarrow (a,t) \wedge \neg p \Rightarrow (b,u)$
requiring in a state $s$ a conditional presence of certain transitions.
Even more interestingly, %referring to our traffic light example, the
%positive formulae $\UPhi(\red)=p\Rightarrow (\prepareToStop,\yellowRed)$ 
%together with
%$\UPhi(\green)=p\Rightarrow (\getReady,\yellow)$ guarantee 
we can enforce binding of actions in different states, 
thus ensuring certain functionality. Take a~simple two state-example:
$\UPhi(s) = p \Rightarrow (\mathit{request},t)$ and
$\UPhi(t) = p \Rightarrow (\mathit{response},s)$.
%Setting $p$ to true 
%enforces 
%e.g.~the temporal property ``whenever in \yellowRed, \yellow is reachable''.
%\todo{JS: I still dont understand this temporal property. Why it looks assymetric
%while the definition of the formulae is symmetric ...???}
We shall further study OTS with formulae in the disjunctive normal form
that are dual to DMTS and whose complexity of parallel composition 
is lower~\cite{benes_et_al:OASIcs:2011:3070} while still being 
as expressive as DMTS.
%able to specify interesting
%properties including those given by MTS.

\subsection{Modal Refinement}\label{ss:ref}

A fundamental advantage of MTS-based formalisms is the presence of 
\emph{modal refinement} 
that allows for a step-wise system design (see e.g.~\cite{AHLNW:EATCS:08}).
We shall now provide such a refinement notion for
our general PMTS model so that it will specialize to the well-studied
refinement notions on its subclasses. In the definition,
the parameters are fixed first (persistence) followed by 
all valid choices modulo the fixed parameters that now behave as
constants. % are analysed mimicking the ideas of the refinement process.

First we set the following notation.
%\begin{definition}\label{def:pers-val}
Let $(S,T,P,\UPhi)$ be a~PMTS and $\Valid \subseteq P$ be 
a truth assignment. For $s \in S$, we denote 
by $\val_{\Valid}(s) = \{ E \subseteq T(s) \mid E \cup \Valid 
\models \UPhi(s)\}$ the set of all admissible sets of transitions from $s$
under the fixed truth values of the parameters.
%\end{definition}

We can now define the notion of modal refinement between PMTS.
%As the semantics is two-stage, this is also reflected in the refinement using two alternations.

\begin{definition}[Modal Refinement]\label{def:pers-mr}
Let $(S_1,T_1,P_1,\UPhi_1)$ and $(S_2,T_2,P_2,\UPhi_2)$ be two PMTSs. 
A binary relation
$R \subseteq S_1 \times S_2$ is a~\emph{modal refinement} if 
for each $\mu \subseteq P_1$ there exists $\Valid \subseteq P_2$
such that for every $(s,t) \in R$ holds %the following holds
\begin{align*}  
 \forall M \in \val_{\mu}(s) \exists N \in \val_{\Valid}(t) : 
        & \ \forall (a,s') \in M : \exists (a,t') \in N : (s',t') \in R \ \ \wedge \\
                &\ \forall (a,t') \in N : \exists (a,s') \in M : (s',t') \in R\;.
\end{align*}
We say that $s$ modally refines $t$, denoted by $s \mr t$, if there
exists a~modal refinement $R$ such that $(s,t) \in R$.
\end{definition}

\begin{figure}[b!]
\centering
\begin{tikzpicture}[->,>=stealth',initial text=,xscale=0.8,yscale=0.8,transform shape]
    \tikzstyle{every node}=[font=\large] \tikzstyle{every
      state}=[fill=black,shape=rectangle,inner sep=.5mm,minimum height=12mm, minimum width=6mm]
    
%%%%  Most general specification %%%%%
   % nodes with traffic lights
    \node[initial,state,draw=white,fill=white] (green3) at (8,0) {};
    \trafficlight{white}{white}{green}{8}{0}
    
    \node[state,draw=white,fill=white] (red3) at (12,0) {};
    \trafficlight{red}{white}{white}{12}{0}
    
    \node[state,draw=white,fill=white] (yellow3) at (10,-2) {};
    \trafficlight{white}{yellow}{white}{10}{-2}
    
    \node[state,draw=white,fill=white] (yellowred3) at (10,2) {};
    \trafficlight{red}{yellow}{white}{10}{2}
    
    \path (red3) edge [bend right=10] node [above] {\go} (green3);
    \path (green3) edge [bend right=10] node [below] {\stop} (red3);
    \path (red3.north) edge [bend right] node [above,sloped] {\getReady} (yellowred3.east);
    \path (yellowred3.west) edge [bend right] node [above,sloped] {\go} (green3.north);
    \path (green3.south) edge [bend right] node [below,sloped] {\prepareToStop} (yellow3.west);
    \path (yellow3.east) edge [bend right] node [below,sloped] {\stop} (red3.south);
    
    \node [right,font=\footnotesize] at (6.6,3.7) {\bfseries Parameters:};
    \node [right,font=\footnotesize] at (6.6,3.2) {$\{\mustGoYellowRed,\mustGoYellow\}$};
    \node [right,font=\footnotesize] at (6.6,-3) {\bfseries Obligation function:};
    \node [right,font=\footnotesize] at (6.6,-3.5) {$\UPhi(\green) = ((\stop,\red) \oplus (\prepareToStop,\yellow))$};
    \node [right,font=\footnotesize] at (7.2,-4) {$\wedge ( \mustGoYellow \Leftrightarrow (\prepareToStop,\yellow)) $};
    \node [right,font=\footnotesize] at (6.6,-4.5) {$ \UPhi(\red) = ((\go,\green) \oplus (\getReady,\yellowRed))$};
    \node [right,font=\footnotesize] at (7.2,-5) {$ \wedge ( \mustGoYellowRed \Leftrightarrow (\getReady,\yellowRed))$};


%%%% Refined Specification %%%%    
    % nodes with traffic lights
    \node[initial,state,draw=white,fill=white] (green2) at (1,0) {};
    \trafficlight{white}{white}{green}{1}{0}
    
    \node[state,draw=white,fill=white] (red2) at (5,0) {};
    \trafficlight{red}{white}{white}{5}{0}
    
    \node[state,draw=white,fill=white] (yellow2) at (3,-2) {};
    \trafficlight{white}{yellow}{white}{3}{-2}
    
    \node[state,draw=white,fill=white] (yellowred2) at (3,2) {};
    \trafficlight{red}{yellow}{white}{3}{2}
    
    \path (red2) edge [bend right=10] node [above] {\go} (green2);
    \path (green2) edge [bend right=10] node [below] {\stop} (red2);
    \path (red2.north) edge [bend right] node [above,sloped] {\getReady} (yellowred2.east);
    \path (yellowred2.west) edge [bend right] node [above,sloped] {\go} (green2.north);
    \path (green2.south) edge [bend right] node [below,sloped] {\prepareToStop} (yellow2.west);
    \path (yellow2.east) edge [bend right] node [below,sloped] {\stop} (red2.south);
    
    \node [right,font=\footnotesize] at (1.0,3.7) {\bfseries Parameters:};
    \node [right,font=\footnotesize] at (1.0,3.2) {$\{\mustGoYellowBoth\}$};
    \node [right,font=\footnotesize] at (-1.0,-3) {\bfseries Obligation function:};
    \node [right,font=\footnotesize] at (-1.0,-3.5) {$\UPhi(\green) = ((\stop,\red) \oplus (\prepareToStop,\yellow))$};
    \node [right,font=\footnotesize] at (-.4,-4) {$\wedge ( \mustGoYellowBoth \Leftrightarrow (\prepareToStop,\yellow)) $};
    \node [right,font=\footnotesize] at (-1.0,-4.5) {$ \UPhi(\red) = ((\go,\green) \oplus (\getReady,\yellowRed))$};
    \node [right,font=\footnotesize] at (-.4,-5) {$ \wedge ( \mustGoYellowBoth \Leftrightarrow (\getReady,\yellowRed))$};

%%%% Small implementation with no yellow light %%%%%
    % nodes with traffic lights
    \node[initial,state,draw=white,fill=white] (green0) at (-5,3) {};
    \trafficlight{white}{white}{green}{-5}{3}
    
    \node[state,draw=white,fill=white] (red0) at (-2,3) {};
    \trafficlight{red}{white}{white}{-2}{3}
    
       
    \path (red0) edge [bend right=10] node [above] {\go} (green0);
    \path (green0) edge [bend right=10] node [below] {\stop} (red0);

%%%% Implementation with yellow light. %%%%-
 % nodes with traffic lights
    \node[initial,state,draw=white,fill=white] (green) at (-5,-2) {};
    \trafficlight{white}{white}{green}{-5}{-2}
    
    \node[state,draw=white,fill=white] (red) at (-2,-2) {};
    \trafficlight{red}{white}{white}{-2}{-2}
    
    \node[state,draw=white,fill=white] (yellow) at (-3.5,-3.5) {};
    \trafficlight{white}{yellow}{white}{-3.5}{-3.5}
    
    \node[state,draw=white,fill=white] (yellowred) at (-3.5,-.5) {};
    \trafficlight{red}{yellow}{white}{-3.5}{-.5}
    
   
    \path (red.north) edge [bend right] node [above,sloped] {\getReady} (yellowred.east);
    \path (yellowred.west) edge [bend right] node [above,sloped] {\go} (green.north);
    \path (green.south) edge [bend right] node [below,sloped] {\prepareToStop} (yellow.west);
    \path (yellow.east) edge [bend right] node [below,sloped] {\stop} (red.south);

%%%% Modal Refinements %%%%
\node[font=\LARGE] at (6.4,0) {$\mr$} ;
\node[font=\LARGE, rotate=-20] at (-0.6,1.9) {$\mr$} ;    
\node[font=\LARGE, rotate=20] at (-0.6,-1.5) {$\mr$} ;
\end{tikzpicture}
\caption{Example of modal refinement.}\label{fig:mr-ex}
\end{figure}



\begin{example}
Consider the rightmost PMTS in Figure~\ref{fig:mr-ex}.
% (the one already presented in Figure~\ref{fig:TL-PMTS}). 
It has two parameters
\mustGoYellow and \mustGoYellowRed whose values can be set independently
and it can be refined by the system in the middle of the figure having
only one parameter \mustGoYellowBoth. This single parameter simply binds 
the two original parameters to the same value. The PMTS in the middle
can be further refined into the implementations where either 
\yellow is always used in both cases, or never at all.
%Recall the agreement that we omit the obligation function when describing 
%implementations because it is uniquely determined as a conjunction 
%of all outgoing transitions.  
Notice that there are in principle infinitely many implementations
of the system in the middle, however, they are all %strongly 
bisimilar to
either of the two implementations depicted in the left of Figure~\ref{fig:mr-ex}.
\end{example}

In the next section, we shall investigate the complexity of positive subclasses of PMTS.
For this reason we prove the following lemma showing how the definition of
modal refinement can be simplified in this particular case.

We shall first realize that in positive PMTS and for any truth assignment $\Valid$, 
$\val_{\Valid}(s)$ is \emph{upward closed},
meaning that if $M \in \val_{\Valid}(s)$ and
$M \subseteq  M' \subseteq T(s)$ then $M' \in \val_{\Valid}(s)$. 


\begin{lemma} \label{lem:ref-positive}
Consider Definition~\ref{def:pers-mr} where the right-hand side PMTS
is positive.
Now the condition in Definition~\ref{def:pers-mr} can be equivalently 
rewritten %as two conditions
as a \nopagebreak conjunction of conditions~\eqref{eqn:three} and \eqref{eqn:four}
\begin{align}
       &\forall M \in \val_{\mu}(s) : \forall (a,s') \in M : \exists
        (a,t') \in T(t) : (s',t') \in R \label{eqn:three} \\
     &\forall M \in \val_{\mu}(s) : \match_t(M) \in \val_{\nu}(t) \label{eqn:four}
\end{align}
where $\match_t(M)$ denotes the set $\{ (a,t') \in T(t) \mid \exists (a,s') \in M :
(s',t') \in R\}$.
If the left-hand side PMTS is moreover positive too, 
Condition~\eqref{eqn:three} is equivalent to
\begin{align}
\forall (a,s') \in T(s) : \exists (a,t') \in T(t) : (s',t') \in R \ .\label{eqn:five}
\end{align}
\end{lemma}
\begin{proof}
We shall first argue that the condition of modal refinement is equivalent to
the conjunction of Conditions 
\eqref{eqn:one} and \eqref{eqn:two}.
\begin{align}
        \forall M \in \val_{\mu}(s) : \exists N \in \val_{\Valid}(t) :
          \forall (a,s') \in M : \exists (a,t') \in N : (s',t') \in R \label{eqn:one}\\
     \forall M \in \val_{\mu}(s) : \exists N \in \val_{\Valid}(t) :
       \forall (a,t') \in N : \exists (a,s') \in M : (s',t') \in R \label{eqn:two}
\end{align}

Let $\mu$, $\nu$, $R$, $s$ and $t$ be fixed. 
Definition~\ref{def:pers-mr} 
trivially implies both Conditions~\eqref{eqn:one} and \eqref{eqn:two}.
We now prove that \eqref{eqn:one} and \eqref{eqn:two} imply  the condition in
Definition~\ref{def:pers-mr}.

Let $M \in \val_{\mu}(s)$ be arbitrary.
There is some $N_1 \in \val_{\Valid}(t)$ satisfying \eqref{eqn:one} and
some $N_2 \in \val_{\Valid}(t)$ satisfying \eqref{eqn:two}.
Let now $N'_1 = \{(a,t') \in N_1 \mid \exists (a,s') \in M : (s',t') \in R \}$.
Consider $N = N'_1 \cup N_2$. Clearly, as $\val_{\Valid}(t)$ is upward closed, 
$N \in \val_{\Valid}(t)$.
%Let now $(a,m) \in M$ be arbitrary. 
Moreover, due to Condition \eqref{eqn:one} we have some $(a,t') \in N_1$
such that $(s',t') \in R$. Clearly, $(a,t') \in N'_1$ and thus also in $N$.

Now let $(a,t') \in N$ be arbitrary. If $(a,t') \in N_2$, due to Condition~\eqref{eqn:two}
we have some $(a,s') \in M$ such that $(s',t') \in R$.
  If $(a,t') \not\in N_2$ then $(a,t') \in N'_1$.
  The existence of $(a,s') \in M$ such that $(s',t') \in R$ is then
  guaranteed by the definition of~$N'_1$.

Let us now proceed with proving the claims of the lemma.
Condition \eqref{eqn:one} is trivially equivalent 
to \eqref{eqn:three} since $\val_{\Valid}(t)$ is upward closed.
Condition \eqref{eqn:two} is equivalent to \eqref{eqn:four}. 
Indeed, \eqref{eqn:four} clearly implies \eqref{eqn:two} and
we show that also \eqref{eqn:two} implies \eqref{eqn:four}. 
Let $M$ be arbitrary. We then have some $N$ satisfying \eqref{eqn:two}.
Clearly, $N \subseteq \match_t(M)$. Since $\val_{\Valid}(t)$ is upward closed,
$N \in \val_{\Valid}(t)$ implies $\match_t(M) \in \val_\nu(t)$.
Due to the upward closeness of both $\val_{\mu}(s)$ and $\val_{\Valid}(t)$
in the case of a~positive left-hand side, 
the equivalence of \eqref{eqn:three} and \eqref{eqn:five} follows.
\end{proof}

\begin{theorem}\label{thm:ref-coincide}
Modal refinement as defined on PMTS coincides with the standard
modal refinement notions on MTS, DMTS and OTS.
On implementations it coincides with %strong 
bisimulation.
\end{theorem}
\begin{proof}%[for MTS; for the rest, see Appendix.]
The fact that Definition~\ref{def:pers-mr} coincides with modal refinement
on OTS as defined in~\cite{benes_et_al:OASIcs:2011:3070} is a~straightforward
corollary of Lemma~\ref{lem:ref-positive} and its proof. 
Indeed, the two conditions given
in~\cite{benes_et_al:OASIcs:2011:3070} are exactly conditions~\eqref{eqn:five}
and~\eqref{eqn:two}. As the definition of modal refinement on OTS
coincides with modal refinement on DMTS (as shown in~\cite{benes_et_al:OASIcs:2011:3070}) and thus also on MTS, the proof 
is done.

However, for the reader's convenience, we present a~direct proof
that Definition~\ref{def:pers-mr} coincides with modal refinement
on MTS.
Assume a parameter-free PMTS $(S,T,P,\UPhi)$ where
$\UPhi(s)$ is a conjunction of transitions for all $s \in S$, in
other words it is a standard MTS where the must transitions
are listed in the conjunction and the may transitions are simply
present in the underlying transition system but not a part of
the conjunction.
Observe that %$P = \emptyset$, and 
every transition $(s,a,t) \in T$ is contained in some 
$M \in \val_\emptyset(s)$. 
Further, each must transition $(s,a,t) \in T$  is contained in all 
$M \in \val_\emptyset(s)$. Therefore, the first conjunct in 
Definition~\ref{def:pers-mr} requires that for all may transition from $s$ 
there be a corresponding one from $t$ with the successors in the refinement 
relation. Similarly, the second conjunct now requires that for all must 
transitions from $t$ there be a corresponding must transition from $s$.
This is exactly the standard notion of modal refinement as introduced
in~\cite{DBLP:conf/lics/LarsenT88}.  
\end{proof}
 

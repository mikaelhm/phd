\subsection{Motivation} 
Modal transition systems 
%~\cite{DBLP:conf/lics/LarsenT88,AHLNW:EATCS:08} 
and their extensions described in the literature
%like disjunctive MTS (DMTS)
%~\cite{DBLP:conf/lics/LarsenX90}, %,fossacs-techrep}, 
%1-selecting MTS (1MTS)%~\cite{FS:JLAP:08} 
%and transition systems with obligations (OTS)
%~\cite{benes_et_al:OASIcs:2011:3070}
are lacking the capability to express several 
specification requirements like exclusive, conditional and persistent choices.
%\footnote{For more details, see Appendix~\ref{ap:mts}.}
% (except for 1-selecting MTS being capable of expressing exclusive-or).  
We shall now discuss these limitations on an example
as a motivation for the introduction of parametric MTS formalism with
general Boolean conditions in specification requirements.

Consider a simple specification of a traffic light controller that can 
be at any moment in one of the four predefined states:
\red, \green, \yellow or \yellowRed.
%, as is some national variants there is a \yellowRed light signal 
%preceding the \green light. 
The requirements of the specification are: when \green is on the traffic 
light may either change to \red or \yellow and if it turned \yellow it must 
go to \red afterward; when \red is on it may either turn to 
\green or \yellowRed, and if it turns \yellowRed (as it is the case in some
countries) it must go to \green afterwords. 
%Intuitively, this means that the two yellow states are optional.


\begin{figure}[ht]
  \centering
  \subfloat[MTS specification $S_1$.]{\label{fig:TL-mts}
   % \scalebox{0.8}{
  \begin{tikzpicture}[->,>=stealth',initial text=,xscale=0.8,yscale=0.8,transform shape]
    \tikzstyle{every node}=[font=\large,label distance=-1.2mm] \tikzstyle{every
      state}=[fill=black,shape=rectangle,inner sep=.5mm,minimum height=12mm, minimum width=6mm]
    % background, as picture bound
    \draw [fill=white,draw=white] (-1,-2) rectangle (3,2);
    
    % nodes with traffic lights
    \node[initial,state,draw=white,fill=white,label=120:{\green}] (green) at (0,0) {};
    \trafficlight{white}{white}{green}{0}{0}
    
    \node[state,draw=white,fill=white,label=right:{\red}] (red) at (4,0) {};
    \trafficlight{red}{white}{white}{4}{0}
    
    \node[state,draw=white,fill=white,label=below:{\yellow}] (yellow) at (2,-2) {};
    \trafficlight{white}{yellow}{white}{2}{-2}
    
    \node[state,draw=white,fill=white,label=90:{\yellowRed}] (yellowred) at (2,2) {};
    \trafficlight{red}{yellow}{white}{2}{2}
    
    \path (red) edge [bend right=10,dashed] node [above] {\go} (green);
    \path (green) edge [bend right=10,dashed] node [below] {\stop} (red);
    \path (red.north) edge [bend right,dashed] node [above,sloped] {\getReady} (yellowred.east);
    \path (yellowred.west) edge [bend right] node [above,sloped] {\go} (green.north);
    \path (green.south) edge [bend right,dashed] node [below,sloped] {\prepareToStop} (yellow.west);
    \path (yellow.east) edge [bend right] node [below,sloped] {\stop} (red.south);
\end{tikzpicture}} 
% }
\qquad
\subfloat[DMTS specification $S_2$.]{\label{fig:TL-DMTS}
% \scalebox{0.8}{
  \begin{tikzpicture}[->,>=stealth',initial text=,xscale=0.8,yscale=0.8,transform shape]
    \tikzstyle{every node}=[font=\large] \tikzstyle{every
      state}=[fill=black,shape=rectangle,inner sep=.5mm,minimum height=12mm, minimum width=6mm]

    % nodes with traffic lights
    \node[initial,state,draw=white,fill=white] (green) at (0,0) {};
    \trafficlight{white}{white}{green}{0}{0}
    
    \node[state,draw=white,fill=white] (red) at (4,0) {};
    \trafficlight{red}{white}{white}{4}{0}
    
    \node[state,draw=white,fill=white] (yellow) at (2,-2) {};
    \trafficlight{white}{yellow}{white}{2}{-2}
    
    \node[state,draw=white,fill=white] (yellowred) at (2,2) {};
    \trafficlight{red}{yellow}{white}{2}{2}
    
    \path (red) edge [bend right=10,dashed] node [above] {\go} (green);
    \path (green) edge [bend right=10,dashed] node [below] {\stop} (red);
    \path (red.north) edge [bend right,dashed] node [above,sloped] {\getReady} (yellowred.east);
    \path (yellowred.west) edge [bend right] node [above,sloped] {\go} (green.north);
    \path (green.south) edge [bend right,dashed] node [below,sloped] {\prepareToStop} (yellow.west);
    \path (yellow.east) edge [bend right] node [below,sloped] {\stop} (red.south);
    \draw (green.south) -- (1.4,-1.2) -- (yellow.west);
    \path (1.4,-1.2) edge (red.west);
    \draw (red.north) -- (2.6,1.2) -- (yellowred.east);
    \path (2.6,1.2) edge (green.east);
    \node at (4.5,-3) {};
\end{tikzpicture} }
% } 

\caption{Specifications of a traffic light controller.}
\end{figure}

%\paragraph{MTS}
Figure~\ref{fig:TL-mts} shows an obvious MTS specification (defined
formally later on) of the proposed specification. 
The transitions in the standard MTS formalism are either of type
may (optional transitions depicted as dashed lines) or must 
(required transitions depicted as solid lines). 
In Figure~\ref{fig:TL-none}, Figure~\ref{fig:TL-both} and 
Figure~\ref{fig:TL-alternation} we present three different implementations 
of the MTS specification where there are no more optional transitions.
The implementation $I_1$ does not implement any may transition as it  
is a valid possibility to satisfy the specification $S_1$.
Of course, in our concrete example, this means that the light is constantly
\green and it is clearly an undesirable behaviour that cannot be, however,
easily avoided. 
The second implementation $I_2$ on the %\ref{fig:TL-both} 
other hand implements all may transitions, again a legal implementation
in the MTS methodology but not a desirable implementation of a traffic
light as the next action is not always deterministically given. Finally, the implementation $I_3$ of $S_1$ 
%Figure~\ref{fig:TL-alternation} 
illustrates the third problem with the MTS specifications, namely
that the choices made in each turn are not persistent and the
implementation alternates between entering \yellow or not.
None of these problems can be avoided when using the MTS formalism.

\begin{figure}[t!]
  \centering

    \subfloat[Implementation $I_1$.]{\label{fig:TL-none}  
        \begin{tikzpicture}[->,>=stealth',initial text=,xscale=0.8,yscale=0.8,transform shape]
            \tikzstyle{every node}=[font=\large] \tikzstyle{every
              state}=[fill=black,shape=rectangle,inner sep=.5mm,minimum height=12mm, minimum width=6mm]
            % background, as picture bound
            \draw [fill=white,draw=white] (-1.5,-2.5) rectangle (1.5,2);


            % nodes with traffic lights
            \node[initial,state,draw=white,fill=white] (green) at (0.5,0) {};
            \trafficlight{white}{white}{green}{0.5}{0}
            \node[draw=white,fill=white] (empty) at (4,0) {};
        \end{tikzpicture}  
    }
    \quad
    \subfloat[Implementation $I_2$.]{\label{fig:TL-both} 
        \begin{tikzpicture}[->,>=stealth',initial text=,xscale=0.8,yscale=0.8,transform shape]
            \tikzstyle{every node}=[font=\large] \tikzstyle{every
              state}=[fill=black,shape=rectangle,inner sep=.5mm,minimum height=12mm, minimum width=6mm]
            % background, as picture bound
            % \draw [fill=white,draw=white] (-1,-2) rectangle (3,2);

            % nodes with traffic lights
            \node[initial,state,draw=white,fill=white] (green) at (0,0) {};
            \trafficlight{white}{white}{green}{0}{0}

            \node[state,draw=white,fill=white] (red) at (4,0) {};
            \trafficlight{red}{white}{white}{4}{0}

            \node[state,draw=white,fill=white] (yellow) at (2,-2) {};
            \trafficlight{white}{yellow}{white}{2}{-2}

            \node[state,draw=white,fill=white] (yellowred) at (2,2) {};
            \trafficlight{red}{yellow}{white}{2}{2}

            \path (red) edge [bend right=10] node [above] {\go} (green);
            \path (green) edge [bend right=10] node [below] {\stop} (red);
            \path (red.north) edge [bend right] node [above,sloped] {\getReady} (yellowred.east);
            \path (yellowred.west) edge [bend right] node [above,sloped] {\go} (green.north);
            \path (green.south) edge [bend right] node [below,sloped] {\prepareToStop} (yellow.west);
            \path (yellow.east) edge [bend right] node [below,sloped] {\stop} (red.south);
        \end{tikzpicture}  
    }
    \quad
    \subfloat[Implementation $I_3$.]{\label{fig:TL-alternation}  
        \begin{tikzpicture}[->,>=stealth',initial text=,xscale=0.8,yscale=0.8,transform shape]
            \tikzstyle{every node}=[font=\large] \tikzstyle{every
            state}=[fill=black,shape=rectangle,inner sep=.5mm,minimum height=12mm, minimum width=6mm]
            % background, as picture bound
            \draw [fill=white,draw=white] (-1,-2) rectangle (3,2);

            % nodes with traffic lights

            \node[initial,state,draw=white,fill=white] (green1) at (-1.902,.618) {};
            \trafficlight{white}{white}{green}{-1.902}{.608}

            \node[state,draw=white,fill=white] (red1) at (-1.176,-1.618) {};
            \trafficlight{red}{white}{white}{-1.176}{-1.618}

            \node[state,draw=white,fill=white] (green2) at (1.176,-1.618){};
            \trafficlight{white}{white}{green}{1.176}{-1.618}

            \node[state,draw=white,fill=white] (yellow) at (1.902,.618) {};
            \trafficlight{white}{yellow}{white}{1.902}{.618}

            \node[state,draw=white,fill=white] (red2) at (0,2) {};
            \trafficlight{red}{white}{white}{0}{2}

            \path (green1) edge [bend right=10] node [below,sloped] {\stop} (red1);
            \path (red1) edge [bend right=10] node [below, sloped] {\go} (green2);
            \path (green2) edge [bend right=10] node [above,sloped] {\getReady} (yellow);
            \path (yellow) edge [bend right=10] node [above, sloped] {\stop} (red2);
            \path (red2) edge [bend right=10] node [above, sloped] {\go} (green1);
        \end{tikzpicture}  
    }
\caption{MTS implementations of a traffic light controller.}
\end{figure}



%\paragraph{DMTS}
A more expressive formalism of disjunctive modal transition systems
(DMTS) can overcome some of the above mentioned problems. 
A possible DMTS specification $S_2$ is depicted in Figure~\ref{fig:TL-DMTS}. 
Here the \getReady and \stop  transitions, as well as
\prepareToStop and \go ones, are disjunctive, meaning that
it is still optional which one is implemented but at least one of them 
must be present. Now the system $I_1$ in Figure~\ref{fig:TL-none} is
not a valid implementation of $S_2$ any more. Nevertheless, 
the undesirable implementations $I_2$ and $I_3$ 
%in Figure~\ref{fig:TL-both} and Figure~\ref{fig:TL-alternation} 
are still possible and the modelling power of DMTS 
is insufficient to eliminate them.

%%% OTS Figure
\begin{figure}[ht]
    \centering
    \begin{tikzpicture}[->,>=stealth',initial text=,xscale=0.8,yscale=0.8,transform shape]
        \tikzstyle{every node}=[font=\large] \tikzstyle{every
          state}=[fill=black,shape=rectangle,inner sep=.5mm,minimum height=12mm, minimum width=6mm]
        % background, as picture bound
        %\node[state,draw=white,fill=white] (empty) at (7.2,0) {};

        %\draw [fill=white,draw=white] (-1,-2) rectangle (3,2);
        
        % nodes with traffic lights
        \node[initial,state,draw=white,fill=white] (green) at (0,0) {};
        \trafficlight{white}{white}{green}{0}{0}
        
        \node[state,draw=white,fill=white] (red) at (4,0) {};
        \trafficlight{red}{white}{white}{4}{0}
        
        \node[state,draw=white,fill=white] (yellow) at (2,-2) {};
        \trafficlight{white}{yellow}{white}{2}{-2}
        
        \node[state,draw=white,fill=white] (yellowred) at (2,2) {};
        \trafficlight{red}{yellow}{white}{2}{2}
        
        \path (red) edge [bend right=10] node [above] {\go} (green);
        \path (green) edge [bend right=10] node [below] {\stop} (red);
        \path (red.north) edge [bend right] node [above,sloped] {\getReady} (yellowred.east);
        \path (yellowred.west) edge [bend right] node [above,sloped] {\go} (green.north);
        \path (green.south) edge [bend right] node [below,sloped] {\prepareToStop} (yellow.west);
        \path (yellow.east) edge [bend right] node [below,sloped] {\stop} (red.south);

        \node[right] at (5,.5) {\bfseries Obligation function:};
        \node[right] at (5,0) {$\UPhi(\green) = (\stop,\red) \oplus (\prepareToStop,\yellow)$};
        \node[right] at (5,-.5) {$\UPhi(\red) = (\go,\green) \oplus (\getReady,\yellowRed)$};
        % \node[right] at (5,-.5) {$\UPhi(\yellow) = (\stop,\red)$};
        % \node[right] at (5,-1) {$\UPhi(\yellowRed) = (\go,\green)$};
        %eqaual size
        % \node at (-1,-5.70) {};
    %   \node at (5,-5.70) {};
    \end{tikzpicture}
    \caption{OTS Specification $S_3$. \label{fig:TL-BMTS}}
\end{figure}

%\paragraph{MTS with boolean conditions}
Inspired by the recent notion of transition systems with 
obligations (OTS)~\cite{benes_et_al:OASIcs:2011:3070}, 
we can model the traffic light using specification
as a transition system with arbitrary\footnote{In the transition
systems with obligations only positive Boolean formulae are allowed.} 
obligation formulae. These formulae are Boolean
propositions over the outgoing transitions from each state,
whose satisfying assignments yield the allowed combinations
of outgoing transitions.
A possible specification called $S_3$ is given 
in Figure~\ref{fig:TL-BMTS} and it uses the operation of exclusive-or. 
We will follow an agreement that whenever the obligation function for some node 
is not listed in the system description then it is implicitly understood as 
requiring all the available outgoing transitions to be present.
Due to the use of exclusive-or in the obligation function, 
the transition systems $I_1$ and $I_2$ are not valid implementation any more. 
Nevertheless, the implementation $I_3$ 
in Figure~\ref{fig:TL-alternation} cannot be avoided in this
formalism either. 

%%% PMTS Figure
\begin{figure}[ht]
  \centering 
      \begin{tikzpicture}[->,>=stealth',initial text=,xscale=0.8,yscale=0.8,transform shape]
        \tikzstyle{every node}=[font=\large] \tikzstyle{every
          state}=[fill=black,shape=rectangle,inner sep=.5mm,minimum height=12mm, minimum width=6mm]
        % background, as picture bound
       % \draw [fill=white,draw=white] (-1,-2) rectangle (3,2);
        
        % nodes with traffic lights
        \node[initial,state,draw=white,fill=white] (green) at (0,0) {};
        \trafficlight{white}{white}{green}{0}{0}
        
        \node[state,draw=white,fill=white] (red) at (4,0) {};
        \trafficlight{red}{white}{white}{4}{0}
        
        \node[state,draw=white,fill=white] (yellow) at (2,-2) {};
        \trafficlight{white}{yellow}{white}{2}{-2}
        
        \node[state,draw=white,fill=white] (yellowred) at (2,2) {};
        \trafficlight{red}{yellow}{white}{2}{2}
        
        \path (red) edge [bend right=10] node [above] {\go} (green);
        \path (green) edge [bend right=10] node [below] {\stop} (red);
        \path (red.north) edge [bend right] node [above,sloped] {\getReady} (yellowred.east);
        \path (yellowred.west) edge [bend right] node [above,sloped] {\go} (green.north);
        \path (green.south) edge [bend right] node [below,sloped] {\prepareToStop} (yellow.west);
        \path (yellow.east) edge [bend right] node [below,sloped] {\stop} (red.south);
        
        \node [right] at (5,2.5) {{\bfseries Parameters:}};
        \node [right] at (5,1.9) {$\{\mustGoYellowRed,\mustGoYellow\}$};
        \node [right] at (5,1.2) {\bfseries Obligation function:};
        \node [right] at (5,.6) {$\UPhi(\green) = ((\stop,\red) \oplus (\prepareToStop,\yellow))$};
        \node [right] at (10,0.05) {$\wedge ( \mustGoYellow \Leftrightarrow (\prepareToStop,\yellow)) $};
        \node [right] at (5,-.6) {$ \UPhi(\red) = ((\go,\green) \oplus (\getReady,\yellowRed))$};
        \node [right] at (10,-1.15) {$ \wedge ( \mustGoYellowRed \Leftrightarrow (\getReady,\yellowRed))$};
        % \node [right] at (5,-1.8) {$\UPhi(\yellow) = (\stop,\red)$};
        % \node [right] at (5,-2.4) {$\UPhi(\yellowRed) = (\go,\green)$};
    \end{tikzpicture}
 
    \caption{PMTS specification $S_4$.\label{fig:TL-PMTS} }
\end{figure}
%\paragraph{PMTS}
Finally, the problem with the alternating implementation $I_3$ is 
that we cannot enforce in any of the above mentioned formalisms 
a uniform (persistent) 
implementation of the same transitions in all its states.
In order to overcome this problem, we propose the so-called parametric 
MTS where we can, moreover, choose persistently whether the transition 
to \yellow is present or not via the use of parameters. 
The PMTS specification with two parameters
\mustGoYellowRed and \mustGoYellow is shown in Figure~\ref{fig:TL-PMTS}. 
Fixing a priori the (Boolean) values of the parameters
 %\mustGoYellow and \mustGoYellowRed 
makes the choices permanent in the whole implementation, hence
we eliminate also the last problematic implementation $I_3$.

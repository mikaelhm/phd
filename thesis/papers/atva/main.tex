Modal  Transition  Systems  (MTS) is a specification formalism
\cite{DBLP:conf/lics/LarsenT88,AHLNW:EATCS:08} that aims at providing
a flexible  and   easy-to-use  compositional
development methodology  for reactive systems.  The formalism
can be viewed as  a   fragment  of   a  temporal   
%\todo{also a recent follow-up work http://dblp.uni-trier.de/rec/bibtex/journals/corr/abs-1108-4464}
logic~\cite{Aceto:graphical,DBLP:journals/tcs/BoudolL92} that at the same time
offers  a   behavioural compositional semantics with an intuitive
notion of process refinement.
%Recent attention to this logical view of MTS has been given in~\cite{Aceto:graphical}.
The formalism of MTS is essentially a labelled  transition system
that distinguishes two types of labelled transitions:  
\emph{must}  transitions  which  are required in any  refinement
of the system, and  \emph{may}  transitions  that are allowed
to appear in a refined system but are not required.
The refinement  of an  MTS  now  essentially consists  of
iteratively resolving the presence or absence of may transitions in 
the refined process.

In a recent line of work \cite{JLS:weighted,MFCS11}, the MTS framework
has  been  extended  to  allow  for the  specification  of  additional
constraints on  quantitative aspects  (e.g.  time, power  or memory),
which are  highly relevant in the  area of embedded  systems.  In this
paper we  continue the  pursuit of quantitative  extensions of  MTS by
presenting a novel extension of MTS with time durations being modelled
as  controllable or  uncontrollable  intervals. We  further equip  the
model with two kinds of  quantitative aspects: each action has its own
running cost per  time unit, and actions may  require several hardware
components of  different costs.   Thus, we ask  the question,  given a
fixed budget  for the investment into the hardware components, what  is the implementation
with the cheapest long-run average reward.

\vspace{\smallskipamount}
Before we  give a formal  definition of modal transition  systems with
durations (MTSD) and the dual-price scheme, and provide algorithms for
computing  optimal  implementations,  we  present  a  small  motivating
example.

%!TEX root = ../../thesis.tex

\begin{figure}[ht]
\centering
	\begin{tikzpicture}[node distance=100pt]
		\node[state,minimum size=20pt,initial] (s0) {$s$};
		\node[state,minimum size=20pt,right=of s0] (s1) {};
		\node[state,minimum size=20pt,right=of s1] (s2) {};
		\node[state,minimum size=20pt,below=of s2] (s3) {};
		\node[state,minimum size=20pt,below=of s1] (s4) {};
		\node[state,minimum size=20pt,below=of s0] (s5) {$t$};
		\node[below=0.4 of s4] {
			$ \Phi(t)= (\BigClean,s)\ \vee (\SkipClean,s)\ \vee (\SmallClean,s)$};
		
		\draw[arc] 
			(s0) edge node[above] {\Wait} node[below] {$\ci{1,5}$} (s1) 
			(s1) edge node[above] {\DriveBus} node[below] {$\ui{6,10}$} (s2) 
			(s2) edge node[sloped,above] {\SmallClean} node[sloped,below] {$\ci{4,6}$} (s3) 
			(s3) edge node[above] {\Wait} node[below] {$\ci{1,5}$} (s4) 
			(s4) edge node[above] {\DriveBus} node[below] {$\ui{6,10}$} (s5) 
			(s5) edge[dashed,bend left=80] node[sloped,above,yshift=-3] {\SmallClean} node[sloped,below,yshift=2] {$\ci{4,6}$} (s0) 
			(s5) edge[dashed,bend right=70] node[sloped,above,yshift=-1] {\BigClean} node[sloped,below,yshift=2] {$\ci{20,30}$} (s0) 
			(s5) edge[dashed] node[sloped,above,yshift=-3] {\SkipClean} node[sloped,below,yshift=2] {0} (s0) 
		{};
	\end{tikzpicture}
	\caption{Example of Dual-Priced Modal Transition Systems with Time Durations, specification $\Spec$.\label{fig:spec1}}
\end{figure}


Consider the specification $\Spec$ in Figure~\ref{fig:spec1} 
describing the work of a shuttle bus driver.
He drives a bus between a hotel and the
airport. First, the driver 
has to \Wait for the passengers at the hotel. This can take one
to five minutes. Since this behaviour is required to be present in all the
implementations of this specification, it is drawn as a solid arrow and 
called a \emph{must} transition. Then the driver has to \DriveBus the bus to
the airport (this takes six to ten minutes)
where he has to do a \SmallClean, then
\Wait before he can \DriveBus the bus back to the hotel. When he returns he can
do either a \SmallClean, \BigClean or \SkipClean of the bus before he
continues. Here we do not require a particular option to be realised in the
implementations, hence we only draw the transitions as dashed arrows. As these
transitions may or may not be present in the implementations, they are called
\emph{may} transitions. However, here the intention is to require at least one
option be realised. Hence, we specify this using a propositional formula $\Phi$
assigned to the state $t$ over its outgoing transitions as described
in~\cite{benes_et_al:OASIcs:2011:3070,BKLMS:ATVA:11}. After performing one of the
actions, the driver starts over again. Note that next time 
the choice in $t$ may differ.

Observe that there are three types of durations on the transitions. First,
there are \emph{controllable} intervals, written in angle brackets. The meaning
of e.g.~$\langle1,5\rangle$ is that in the implementation we can instruct the
driver to wait for a fixed number of minutes in the range. Second, there are
\emph{uncontrollable} intervals, written in square brackets. 
The interval~$[6,10]$ on
the \DriveBus transition means that in the implementation we cannot fix any
particular time and the time can vary, say, depending on the traffic 
and it is chosen
nondeterministically by the environment. Third, the degenerated case of a
single number, e.g.~$0$, denotes that the time taken is always constant and
given by this number. In particular, a~zero duration means that the transition
happens instantaneously.
\begin{figure}[ht]
	\centering
	\subfloat[Specification $\Spec_1$.\label{fig:spec2}]{
	\begin{tikzpicture}[node distance=80pt]
			\node[state,minimum size=20pt,initial] (s0) {$s$};
			\node[state,minimum size=20pt,right=of s0] (s1) {};
			\node[state,minimum size=20pt,below=of s1,xshift=-50pt] (s2) {};
			
			\draw[arc] 
				(s0) edge node[above] {\Wait} node[below] {$\ci{3,5}$} (s1) 
				(s1) edge node[sloped,below] {\DriveBus} node[sloped,above] {$\ui{6,10}$} (s2) 
				(s2) edge node[sloped,below] {\SmallClean} node[sloped,above] {$\ci{4,6}$}(s0)		{};
		\end{tikzpicture}
	}
	\qquad
	\subfloat[Implementation $\Impl_1$.\label{fig:impl1}]{
		\begin{tikzpicture}[node distance=80pt]
			\node[state,minimum size=20pt,initial] (s0) {$s$};
			\node[state,minimum size=20pt,right=of s0] (s1) {};
			\node[state,minimum size=20pt,below=of s1,xshift=-50pt] (s2) {};
			
			\draw[arc] 
				(s0) edge node[above] {\Wait} node[below] {$5$} (s1) 
				(s1) edge node[sloped,below] {\DriveBus} node[sloped,above] {$\ui{6,10}$} (s2) 
				(s2) edge node[sloped,below] {\SmallClean} node[sloped,above] {$6$}(s0)		{};
		\end{tikzpicture}
	}
	\caption{Examples of Dual-Priced Modal Transition Systems with Time Durations.}
\end{figure}

The system $\Spec_1$ is another specification, 
a \emph{refinement} of $\Spec$, where 
we additionally specify that the driver 
must do a \SmallClean after each \DriveBus.
Note that the \Wait interval has been narrowed. 
The system $\Impl_1$ is an implementation of
$\Spec_1$ (and actually also of $\Spec$) 
where all controllable time intervals have
already been fully resolved to their final single values: 
the driver must \Wait for 5
minutes and do the \SmallClean for 6 minutes. Note that uncontrollable
intervals remain unresolved in the implementations and the time is
chosen by the environment each time the action is
performed. This reflects  the inherent uncontrollable uncertainty of the
timing, e.g.~of a traffic.
\begin{figure}[ht]
\centering
	\begin{tikzpicture}[node distance=100pt]
		\node[state,minimum size=20pt,initial] (s0) {$s$};
		\node[state,minimum size=20pt,right=of s0] (s1) {};
		\node[state,minimum size=20pt,right=of s1] (s2) {};
		\node[state,minimum size=20pt,below=of s2] (s3) {};
		\node[state,minimum size=20pt,below=of s1] (s4) {};
		\node[state,minimum size=20pt,below=of s0] (s5) {$t$};
		% \node[below=0.4 of s4] {$ \Phi(t)= (\BigClean,s)\ \vee (\SkipClean,s)\ \vee (\SmallClean,s)$};
		
		\draw[arc] 
			(s0) edge node[above] {\Wait} node[below] {$\ci{1,5}$} (s1) 
			(s1) edge node[above] {\DriveBus} node[below] {$\ui{6,10}$} (s2) 
			(s2) edge node[sloped,above] {\SmallClean} node[sloped,below] {$6$} (s3) 
			(s3) edge node[above] {\Wait} node[below] {$\ci{1,5}$} (s4) 
			(s4) edge node[above] {\DriveBus} node[below] {$\ui{6,10}$} (s5) 
			(s5) edge[dashed,bend left=80] node[sloped,above,yshift=-3] {\SmallClean} node[sloped,below,yshift=2] {$\ui{4,5}$} (s0) 
			(s5) edge[bend right=70] node[sloped,above,yshift=-1] {\BigClean} node[sloped,below,yshift=2] {$\ci{20,30}$} (s0) 
			% (s5) edge[dashed] node[sloped,above,yshift=-3] {\SkipClean} node[sloped,below,yshift=2] {0} (s0) 
			{};
	\end{tikzpicture}
	\caption{Specification $\Spec_2$.\label{fig:spec3}}
\end{figure}
\begin{figure}[ht]
\centering
	\begin{tikzpicture}[node distance=100pt]
		\node[state,minimum size=20pt,initial] (s0) {$s$};
		\node[state,minimum size=20pt,right=of s0] (s1) {};
		\node[state,minimum size=20pt,right=of s1] (s2) {};
		\node[state,minimum size=20pt,below=of s2] (s3) {};
		\node[state,minimum size=20pt,below=of s1] (s4) {};
		\node[state,minimum size=20pt,below=of s0] (s5) {$t$};
		% \node[below=0.4 of s4] {$ \Phi(t)= (\BigClean,s)\ \vee (\SkipClean,s)\ \vee (\SmallClean,s)$};
		
		\draw[arc] 
			(s0) edge node[above] {\Wait} node[below] {$1$} (s1) 
			(s1) edge node[above] {\DriveBus} node[below] {$\ui{6,10}$} (s2) 
			(s2) edge node[sloped,above] {\SmallClean} node[sloped,below] {$6$} (s3) 
			(s3) edge node[above] {\Wait} node[below] {$1$} (s4) 
			(s4) edge node[above] {\DriveBus} node[below] {$\ui{6,10}$} (s5) 
			% (s5) edge[dashed,bend left=80] node[sloped,above,yshift=-3] {\SmallClean} node[sloped,below,yshift=2] {$\ui{4,5}$} (s0) 
			(s5) edge node[sloped,above,yshift=-1] {\BigClean} node[sloped,below,yshift=2] {$30$} (s0) 
			% (s5) edge[dashed] node[sloped,above,yshift=-3] {\SkipClean} node[sloped,below,yshift=2] {0} (s0) 
			{};
	\end{tikzpicture}
	\caption{Implementation $\Impl_2$.\label{fig:impl2}}
\end{figure}

The system $\Spec_2$ is yet another specification and again a 
refinement of $\Spec$, where the
driver can always do a \BigClean in $t$ and 
possibly there is also an~alternative
allowed here of a \SmallClean. Notice that both \SmallClean intervals have been
restricted and changed to uncontrollable. This means that we give up the
control over the duration of this action and if this transition is implemented,
its duration will be every time chosen nondeterministically in that range.
Finally, $\Impl_2$ is then an implementation of $\Spec_2$ and $\Spec$.





\begin{figure}[!t]
	\centering
	\begin{tikzpicture}[node distance=.2]
		\node[anchor=north west,fill=lightlightblue] at (0,0) (run-cost){ 
			\begin{tabular}{l|r}
				$a\in\Sigma$  	&\; $r(a)$	\\ \hline
				\Wait 			& 8\\
				\DriveBus 		& 10\\
				\SmallClean 	& 6\\
				\BigClean 		& 7\\
				\SkipClean 		& 0
			\end{tabular} 
		};
		\node[right=of run-cost.north east,anchor=north west, fill=lightlightblue] (invest-cost){
			\begin{tabular}{l|r}
				$h\in H$  		&\; $\hc(h)$\\ \hline
				\VacuumCleaner  & 100\\
				\Sponge 		& 5
			\end{tabular}
		};

		\node[above=of run-cost.north west,anchor=south west, fill=lightlightblue] (H-set) {$H=\{\VacuumCleaner,\Sponge\}$};
		\node[below=of run-cost.south west,anchor=north west, fill=lightlightblue] (H-req) {
				$ 
					\Psi(a) = \begin{cases}
									\VacuumCleaner & \text{if } a = \BigClean \\
									\Sponge \vee \VacuumCleaner & \text{if } a= \SmallClean \\
									true & \text{otherwise}
								\end{cases}
				$
		};
	\end{tikzpicture}
\caption{Price Scheme.\label{fig:cost}}
\end{figure}	

Furthermore, we develop a way to model cost of resources. Each action is
assigned a \emph{running price} it costs per time unit, e.g.~\DriveBus costs 10
each time unit it is being performed as it can be  seen in the left table of
Figure~\ref{fig:cost}. In addition, in order to perform an action, some
hardware may be needed, e.g.~a~\VacuumCleaner for the \BigClean and its price
is 100 as can be seen on the right. This \emph{investment price} is paid once
only.

Let us now consider the problem of finding an optimal implementation, so that we
spend the least possible amount of money (e.g.~the pay to the driver) per time
unit while conforming to the specification $\Spec$. 
We call this problem \emph{the
cheapest implementation problem}.
The optimal implementation is to buy a vacuum cleaner if one can afford an
investment of 100 and do the \BigClean every time as long as possible and \Wait
as shortly as possible. (Note that \BigClean is more costly per time unit than
\SmallClean but lasts longer.) This is precisely implemented in $\Impl_2$ and
the (worst-case) average cost per time unit is $\approx7.97$. 
If one cannot afford the
vacuum cleaner but only a sponge, the optimal worst case long run average is
then a bit worse and is implemented by doing the \SmallClean as long as
possible and \Wait now as \emph{long} as possible. This is depicted in $\Impl_1$
and the respective average cost per time unit is $\approx8.10$. 
%For further details, see Example~\ref{ex:prices}.


\vspace{\smallskipamount}

The most related work is~\cite{DBLP:conf/emsoft/ChakrabartiAHS03} where prices are introduced into a class of interface theories and long-run average objectives are discussed. Our work omits the issue of distinguishing input and output actions. Nevertheless, compared to~\cite{DBLP:conf/emsoft/ChakrabartiAHS03}, this paper deals with the time durations, the one-shot hardware investment and, most importantly, refinement of specifications. Further, timed automata have also been extended with prices~\cite{DBLP:conf/fmco/BehrmannLR04} and the long-run average reward has been computed in~\cite{DBLP:journals/fmsd/BouyerBL08}. However, priced timed automata lack the hardware and any notion of refinement, too.

The paper is organized as follows. We introduce the MTS with the 
time durations in Section~\ref{bmts} and 
the dual-price scheme 
%together with the problem of the cheapest implementation 
in Section~\ref{prices}. Section~\ref{complexity} presents the main results on the complexity of the cheapest implementation problem. 
First, we state the complexity of this problem in general and in an important special case and prove the hardness part. The algorithms proving the complexity upper bounds are presented only after introducing an extension of {mean payoff games with time durations}. These are needed to establish the results but are also interesting on their own as discussed in~Section~\ref{mpgd}.
 We conclude and give some more account on related and future 
 work in Section~\ref{concl}.



\section{Parametric Modal Transition Systems} 
In this section we present the formalism of parametric modal
transition systems (PMTS), starting with a motivating example and continuing
with the formal definitions, followed by the general notion of
modal refinement. 

%!TEX root = ../thesis.tex
\chapter{Motivation}
During the last decade, software has become a regular part in all types of products, both consumer- and industrial scale products. Twenty years ago software where generally regarded as programs running on personal computers. Today however, many products rely on software running inside it. For instance, all new cars contain a software system that triggers emergency functionality on the basis of inputs from sensors in the car. Furthermore, coffee machines, dishwashers, elevators, heating/cooling systems and medical monitors are all examples of products that rely on software-based control systems. Such applications are grouped under the common term \emph{embedded software}.

\section{Safety-critical Systems}
As with any other software application, embedded software can contain errors. Most people have experienced a computer program crashing, become unresponsive or doing something unexpected. Embedded software can have the same problems. As embedded software is not always directly visible, it can be hard to determine whether the system is running correctly. This can be extremely dangerous if the embedded software is responsible for triggering safety mechanisms in an emergency situation. We say that a system is \emph{safety critical} if an error in the system can result in a costly outcome, cause injury or even lead to death. Software errors can occur due to various reasons. One of these reasons is that the size of software systems can make them so complex that it is hard to avoid errors.

Guaranteeing that a piece of software does not contain any errors and only behaves as intended is a non-trivial task. It is crucial to describe exactly how the system should function, but also specify the potential errors of the system. By formally defining the behaviour of the system, one can examine the system specification using mathematically-based techniques. Techniques building upon these concepts are known in software engineering as \emph{formal methods}.

Formal methods rely on a precise system specification expressed in a well-defined mathematical formalism. Mathematical reasoning can then prove the absence of the specified errors. Formal method techniques can be applied in several stages of a design process in software development. A software design process is a tool that helps developers make better software. The design processes define a number steps that the development must go through, for instance requirement specification, system design, implementation etc. Model-based development is a design process centred around the model of a system. This central model can then be altered and refined throughout the first steps of the design process and used for comparison when testing the implemented software.

By expressing this central model in a mathematical formalism, we can apply formal methods in the development process and avoid errors early in the design phase. In order to do this, there must be a formalism capable of expressing the behaviour and concepts of the system. If a system must have deadlines and timing constraints in order to meet its requirements, then the modelling framework must also be able to express such timing features. Further, if a system manages a resource of some kind, the models should also include such mechanisms. However, having complex modelling formalisms will also make the mathematical reasoning harder and often impossible.

Complexity is a significant problem in the field of formal methods, but there are ``divide and conquer'' techniques that attempt to alleviate the problem. The idea is to split the system model into simpler components and then analyse them individually. When splitting a model into components, a new problem arises: how do one ensure that the composition of these components will meet the overall requirements? We need to make sure that they cooperate in a correct manner.

Some systems are constructed by individual components communicating together. Such systems are known as distributed systems. For instance, an intelligent heating control for a house is a distributed system. It consists of heating regulators that open and close the radiators, thermometers that report the room temperature, window sensors telling if the window is open and a control unit. The components communicate on a wireless connection, so it is essential that the communication is reliable.

The main subject of this thesis is to enhance the ability to use formal methods in system design of embedded systems. This is done by extending and improving existing techniques for system specification, in order make them more applicable in a development process of a real system. 









\subsection{Definition of Parametric Modal Transition System}

We shall now formally capture the intuition behind parametric MTS 
introduced above.
%in the previous section. 
First, we recall the standard
propositional logic.
 
%We start with a description of the formulae used for the specification of 
%the required behaviour.

%\begin{definition}
A~Boolean formula over a set $X$ of atomic propositions is given by
the following abstract syntax 
\[ 
\varphi ::= \true \mid x \mid \neg \varphi \mid \varphi \wedge \psi 
\mid \varphi \vee \psi \]
where $x$ ranges over $X$. 
The set of all Boolean formulae over the set $X$
is denoted by $\B(X)$. 
%The satisfaction relation is given as follows.
Let $\Valid \subseteq X$ be a truth assignment, 
i.e.~a~set of variables with value true, then the satisfaction relation
$\Valid \models \varphi$ is given by
%\begin{align*}
$\Valid \models \true$, $\Valid \models x$ iff $x \in \Valid $,
and the satisfaction of the remaining Boolean connectives is defined in the standard way.
%$\Valid X \models \neg \varphi$ if $\Valid X \not\models \varphi$,
%$\Valid X \models \varphi \wedge \psi$ if $\Valid X \models \varphi$
%and
%$\Valid X \models \psi$, and 
%$\Valid X \models \varphi \vee \psi$ if $\Valid X \models \varphi$
%or $\Valid X \models \psi$.
%\end{align*}
We also use the standard derived operators like
exclusive-or  
$\varphi \oplus \psi = (\varphi \wedge \neg \psi) \vee (\neg \varphi 
\wedge \psi)$, implication
$\varphi \Rightarrow \psi = \neg \varphi \vee \psi$ and equivalence $\varphi \Leftrightarrow \psi = (\neg \varphi \vee \psi)\wedge(\varphi \vee \neg \psi)$.
%\end{definition}

We can now proceed with the definition of parametric MTS.

\begin{definition}
A~\emph{parametric MTS} (PMTS) over an action alphabet $\USigma$ is 
a~tuple $(S,T,P,\UPhi)$ such that
	\begin{itemize}
	\item 
$S$ is a~set of \emph{states},
\item 
$T \subseteq S \times \USigma \times S$ is 
a~\emph{transition relation}, 
\item 
$P$ is a~finite set of \emph{parameters}, and
\item 
$\UPhi : S \to \mathcal B((\USigma \times S) \cup P)$ is a satisfiable
\emph{obligation function} over the atomic propositions containing 
outgoing transitions and parameters. 
\end{itemize}
We implicitly assume that whenever $(a,t) \in \UPhi(s)$ then 
$(s,a,t) \in T$.
By $T(s) = \{(a,t) \mid (s,a,t) \in T\}$ we denote the set of 
all outgoing transitions of $s$.
\end{definition}

% In what follows, 
We recall the agreement that whenever the obligation function for some node 
is not listed in the system description then it is implicitly understood as 
$\UPhi(s) = \bigwedge T(s)$, with the empty conjunction being $\true$.

We call a~PMTS \emph{positive} if, for all $s \in S$, any negation 
occurring in $\UPhi(s)$ is applied only to a parameter. 
A~PMTS is called \emph{parameter-free} if $P = \emptyset$.
We can now instantiate the previously studied specification formalisms
as subclasses of PMTS.

\begin{definition}
A PMTS is called
\begin{itemize}
\item \emph{transition system with obligation} (OTS)
if it is parameter-free and positive,
\item
\emph{disjunctive modal transition system} (DMTS)
if it is an OTS and $\UPhi(s)$ is in the 
conjunctive normal form for all $s \in S$,
% (conjunction of clauses, each of them being a disjunction of literals),
\item 
\emph{modal transition system} (MTS)
if it is a~DMTS and $\UPhi(s)$ is a~conjunction of positive literals 
(transitions) for all $s \in S$, and
\item \emph{implementation} (or simply a labelled
transition system) if it is an MTS and 
$\UPhi(s) = \bigwedge T(s)$ for all $s \in S$. 
\end{itemize}
\end{definition}

%In the case of DMTS and OTS, this definition yields the so 
%called \emph{syntactically consistent} variants.
%Further, 

Note that positive PMTS, despite the absence of a general negation and the
impossibility to define for example exclusive-or, can still express 
useful requirements like 
$\UPhi(s)=p\Rightarrow (a,t) \wedge \neg p \Rightarrow (b,u)$
requiring in a state $s$ a conditional presence of certain transitions.
Even more interestingly, %referring to our traffic light example, the
%positive formulae $\UPhi(\red)=p\Rightarrow (\prepareToStop,\yellowRed)$ 
%together with
%$\UPhi(\green)=p\Rightarrow (\getReady,\yellow)$ guarantee 
we can enforce binding of actions in different states, 
thus ensuring certain functionality. Take a~simple two state-example:
$\UPhi(s) = p \Rightarrow (\mathit{request},t)$ and
$\UPhi(t) = p \Rightarrow (\mathit{response},s)$.
%Setting $p$ to true 
%enforces 
%e.g.~the temporal property ``whenever in \yellowRed, \yellow is reachable''.
%\todo{JS: I still dont understand this temporal property. Why it looks assymetric
%while the definition of the formulae is symmetric ...???}
We shall further study OTS with formulae in the disjunctive normal form
that are dual to DMTS and whose complexity of parallel composition 
is lower~\cite{benes_et_al:OASIcs:2011:3070} while still being 
as expressive as DMTS.
%able to specify interesting
%properties including those given by MTS.

\subsection{Modal Refinement}\label{ss:ref}

A fundamental advantage of MTS-based formalisms is the presence of 
\emph{modal refinement} 
that allows for a step-wise system design (see e.g.~\cite{AHLNW:EATCS:08}).
We shall now provide such a refinement notion for
our general PMTS model so that it will specialize to the well-studied
refinement notions on its subclasses. In the definition,
the parameters are fixed first (persistence) followed by 
all valid choices modulo the fixed parameters that now behave as
constants. % are analysed mimicking the ideas of the refinement process.

First we set the following notation.
%\begin{definition}\label{def:pers-val}
Let $(S,T,P,\UPhi)$ be a~PMTS and $\Valid \subseteq P$ be 
a truth assignment. For $s \in S$, we denote 
by $\val_{\Valid}(s) = \{ E \subseteq T(s) \mid E \cup \Valid 
\models \UPhi(s)\}$ the set of all admissible sets of transitions from $s$
under the fixed truth values of the parameters.
%\end{definition}

We can now define the notion of modal refinement between PMTS.
%As the semantics is two-stage, this is also reflected in the refinement using two alternations.

\begin{definition}[Modal Refinement]\label{def:pers-mr}
Let $(S_1,T_1,P_1,\UPhi_1)$ and $(S_2,T_2,P_2,\UPhi_2)$ be two PMTSs. 
A binary relation
$R \subseteq S_1 \times S_2$ is a~\emph{modal refinement} if 
for each $\mu \subseteq P_1$ there exists $\Valid \subseteq P_2$
such that for every $(s,t) \in R$ holds %the following holds
\begin{align*}  
 \forall M \in \val_{\mu}(s) \exists N \in \val_{\Valid}(t) : 
        & \ \forall (a,s') \in M : \exists (a,t') \in N : (s',t') \in R \ \ \wedge \\
                &\ \forall (a,t') \in N : \exists (a,s') \in M : (s',t') \in R\;.
\end{align*}
We say that $s$ modally refines $t$, denoted by $s \mr t$, if there
exists a~modal refinement $R$ such that $(s,t) \in R$.
\end{definition}

\begin{figure}[b!]
\centering
\begin{tikzpicture}[->,>=stealth',initial text=,xscale=0.8,yscale=0.8,transform shape]
    \tikzstyle{every node}=[font=\large] \tikzstyle{every
      state}=[fill=black,shape=rectangle,inner sep=.5mm,minimum height=12mm, minimum width=6mm]
    
%%%%  Most general specification %%%%%
   % nodes with traffic lights
    \node[initial,state,draw=white,fill=white] (green3) at (8,0) {};
    \trafficlight{white}{white}{green}{8}{0}
    
    \node[state,draw=white,fill=white] (red3) at (12,0) {};
    \trafficlight{red}{white}{white}{12}{0}
    
    \node[state,draw=white,fill=white] (yellow3) at (10,-2) {};
    \trafficlight{white}{yellow}{white}{10}{-2}
    
    \node[state,draw=white,fill=white] (yellowred3) at (10,2) {};
    \trafficlight{red}{yellow}{white}{10}{2}
    
    \path (red3) edge [bend right=10] node [above] {\go} (green3);
    \path (green3) edge [bend right=10] node [below] {\stop} (red3);
    \path (red3.north) edge [bend right] node [above,sloped] {\getReady} (yellowred3.east);
    \path (yellowred3.west) edge [bend right] node [above,sloped] {\go} (green3.north);
    \path (green3.south) edge [bend right] node [below,sloped] {\prepareToStop} (yellow3.west);
    \path (yellow3.east) edge [bend right] node [below,sloped] {\stop} (red3.south);
    
    \node [right,font=\footnotesize] at (6.6,3.7) {\bfseries Parameters:};
    \node [right,font=\footnotesize] at (6.6,3.2) {$\{\mustGoYellowRed,\mustGoYellow\}$};
    \node [right,font=\footnotesize] at (6.6,-3) {\bfseries Obligation function:};
    \node [right,font=\footnotesize] at (6.6,-3.5) {$\UPhi(\green) = ((\stop,\red) \oplus (\prepareToStop,\yellow))$};
    \node [right,font=\footnotesize] at (7.2,-4) {$\wedge ( \mustGoYellow \Leftrightarrow (\prepareToStop,\yellow)) $};
    \node [right,font=\footnotesize] at (6.6,-4.5) {$ \UPhi(\red) = ((\go,\green) \oplus (\getReady,\yellowRed))$};
    \node [right,font=\footnotesize] at (7.2,-5) {$ \wedge ( \mustGoYellowRed \Leftrightarrow (\getReady,\yellowRed))$};


%%%% Refined Specification %%%%    
    % nodes with traffic lights
    \node[initial,state,draw=white,fill=white] (green2) at (1,0) {};
    \trafficlight{white}{white}{green}{1}{0}
    
    \node[state,draw=white,fill=white] (red2) at (5,0) {};
    \trafficlight{red}{white}{white}{5}{0}
    
    \node[state,draw=white,fill=white] (yellow2) at (3,-2) {};
    \trafficlight{white}{yellow}{white}{3}{-2}
    
    \node[state,draw=white,fill=white] (yellowred2) at (3,2) {};
    \trafficlight{red}{yellow}{white}{3}{2}
    
    \path (red2) edge [bend right=10] node [above] {\go} (green2);
    \path (green2) edge [bend right=10] node [below] {\stop} (red2);
    \path (red2.north) edge [bend right] node [above,sloped] {\getReady} (yellowred2.east);
    \path (yellowred2.west) edge [bend right] node [above,sloped] {\go} (green2.north);
    \path (green2.south) edge [bend right] node [below,sloped] {\prepareToStop} (yellow2.west);
    \path (yellow2.east) edge [bend right] node [below,sloped] {\stop} (red2.south);
    
    \node [right,font=\footnotesize] at (1.0,3.7) {\bfseries Parameters:};
    \node [right,font=\footnotesize] at (1.0,3.2) {$\{\mustGoYellowBoth\}$};
    \node [right,font=\footnotesize] at (-1.0,-3) {\bfseries Obligation function:};
    \node [right,font=\footnotesize] at (-1.0,-3.5) {$\UPhi(\green) = ((\stop,\red) \oplus (\prepareToStop,\yellow))$};
    \node [right,font=\footnotesize] at (-.4,-4) {$\wedge ( \mustGoYellowBoth \Leftrightarrow (\prepareToStop,\yellow)) $};
    \node [right,font=\footnotesize] at (-1.0,-4.5) {$ \UPhi(\red) = ((\go,\green) \oplus (\getReady,\yellowRed))$};
    \node [right,font=\footnotesize] at (-.4,-5) {$ \wedge ( \mustGoYellowBoth \Leftrightarrow (\getReady,\yellowRed))$};

%%%% Small implementation with no yellow light %%%%%
    % nodes with traffic lights
    \node[initial,state,draw=white,fill=white] (green0) at (-5,3) {};
    \trafficlight{white}{white}{green}{-5}{3}
    
    \node[state,draw=white,fill=white] (red0) at (-2,3) {};
    \trafficlight{red}{white}{white}{-2}{3}
    
       
    \path (red0) edge [bend right=10] node [above] {\go} (green0);
    \path (green0) edge [bend right=10] node [below] {\stop} (red0);

%%%% Implementation with yellow light. %%%%-
 % nodes with traffic lights
    \node[initial,state,draw=white,fill=white] (green) at (-5,-2) {};
    \trafficlight{white}{white}{green}{-5}{-2}
    
    \node[state,draw=white,fill=white] (red) at (-2,-2) {};
    \trafficlight{red}{white}{white}{-2}{-2}
    
    \node[state,draw=white,fill=white] (yellow) at (-3.5,-3.5) {};
    \trafficlight{white}{yellow}{white}{-3.5}{-3.5}
    
    \node[state,draw=white,fill=white] (yellowred) at (-3.5,-.5) {};
    \trafficlight{red}{yellow}{white}{-3.5}{-.5}
    
   
    \path (red.north) edge [bend right] node [above,sloped] {\getReady} (yellowred.east);
    \path (yellowred.west) edge [bend right] node [above,sloped] {\go} (green.north);
    \path (green.south) edge [bend right] node [below,sloped] {\prepareToStop} (yellow.west);
    \path (yellow.east) edge [bend right] node [below,sloped] {\stop} (red.south);

%%%% Modal Refinements %%%%
\node[font=\LARGE] at (6.4,0) {$\mr$} ;
\node[font=\LARGE, rotate=-20] at (-0.6,1.9) {$\mr$} ;    
\node[font=\LARGE, rotate=20] at (-0.6,-1.5) {$\mr$} ;
\end{tikzpicture}
\caption{Example of modal refinement.}\label{fig:mr-ex}
\end{figure}



\begin{example}
Consider the rightmost PMTS in Figure~\ref{fig:mr-ex}.
% (the one already presented in Figure~\ref{fig:TL-PMTS}). 
It has two parameters
\mustGoYellow and \mustGoYellowRed whose values can be set independently
and it can be refined by the system in the middle of the figure having
only one parameter \mustGoYellowBoth. This single parameter simply binds 
the two original parameters to the same value. The PMTS in the middle
can be further refined into the implementations where either 
\yellow is always used in both cases, or never at all.
%Recall the agreement that we omit the obligation function when describing 
%implementations because it is uniquely determined as a conjunction 
%of all outgoing transitions.  
Notice that there are in principle infinitely many implementations
of the system in the middle, however, they are all %strongly 
bisimilar to
either of the two implementations depicted in the left of Figure~\ref{fig:mr-ex}.
\end{example}

In the next section, we shall investigate the complexity of positive subclasses of PMTS.
For this reason we prove the following lemma showing how the definition of
modal refinement can be simplified in this particular case.

We shall first realize that in positive PMTS and for any truth assignment $\Valid$, 
$\val_{\Valid}(s)$ is \emph{upward closed},
meaning that if $M \in \val_{\Valid}(s)$ and
$M \subseteq  M' \subseteq T(s)$ then $M' \in \val_{\Valid}(s)$. 


\begin{lemma} \label{lem:ref-positive}
Consider Definition~\ref{def:pers-mr} where the right-hand side PMTS
is positive.
Now the condition in Definition~\ref{def:pers-mr} can be equivalently 
rewritten %as two conditions
as a \nopagebreak conjunction of conditions~\eqref{eqn:three} and \eqref{eqn:four}
\begin{align}
       &\forall M \in \val_{\mu}(s) : \forall (a,s') \in M : \exists
        (a,t') \in T(t) : (s',t') \in R \label{eqn:three} \\
     &\forall M \in \val_{\mu}(s) : \match_t(M) \in \val_{\nu}(t) \label{eqn:four}
\end{align}
where $\match_t(M)$ denotes the set $\{ (a,t') \in T(t) \mid \exists (a,s') \in M :
(s',t') \in R\}$.
If the left-hand side PMTS is moreover positive too, 
Condition~\eqref{eqn:three} is equivalent to
\begin{align}
\forall (a,s') \in T(s) : \exists (a,t') \in T(t) : (s',t') \in R \ .\label{eqn:five}
\end{align}
\end{lemma}
\begin{proof}
We shall first argue that the condition of modal refinement is equivalent to
the conjunction of Conditions 
\eqref{eqn:one} and \eqref{eqn:two}.
\begin{align}
        \forall M \in \val_{\mu}(s) : \exists N \in \val_{\Valid}(t) :
          \forall (a,s') \in M : \exists (a,t') \in N : (s',t') \in R \label{eqn:one}\\
     \forall M \in \val_{\mu}(s) : \exists N \in \val_{\Valid}(t) :
       \forall (a,t') \in N : \exists (a,s') \in M : (s',t') \in R \label{eqn:two}
\end{align}

Let $\mu$, $\nu$, $R$, $s$ and $t$ be fixed. 
Definition~\ref{def:pers-mr} 
trivially implies both Conditions~\eqref{eqn:one} and \eqref{eqn:two}.
We now prove that \eqref{eqn:one} and \eqref{eqn:two} imply  the condition in
Definition~\ref{def:pers-mr}.

Let $M \in \val_{\mu}(s)$ be arbitrary.
There is some $N_1 \in \val_{\Valid}(t)$ satisfying \eqref{eqn:one} and
some $N_2 \in \val_{\Valid}(t)$ satisfying \eqref{eqn:two}.
Let now $N'_1 = \{(a,t') \in N_1 \mid \exists (a,s') \in M : (s',t') \in R \}$.
Consider $N = N'_1 \cup N_2$. Clearly, as $\val_{\Valid}(t)$ is upward closed, 
$N \in \val_{\Valid}(t)$.
%Let now $(a,m) \in M$ be arbitrary. 
Moreover, due to Condition \eqref{eqn:one} we have some $(a,t') \in N_1$
such that $(s',t') \in R$. Clearly, $(a,t') \in N'_1$ and thus also in $N$.

Now let $(a,t') \in N$ be arbitrary. If $(a,t') \in N_2$, due to Condition~\eqref{eqn:two}
we have some $(a,s') \in M$ such that $(s',t') \in R$.
  If $(a,t') \not\in N_2$ then $(a,t') \in N'_1$.
  The existence of $(a,s') \in M$ such that $(s',t') \in R$ is then
  guaranteed by the definition of~$N'_1$.

Let us now proceed with proving the claims of the lemma.
Condition \eqref{eqn:one} is trivially equivalent 
to \eqref{eqn:three} since $\val_{\Valid}(t)$ is upward closed.
Condition \eqref{eqn:two} is equivalent to \eqref{eqn:four}. 
Indeed, \eqref{eqn:four} clearly implies \eqref{eqn:two} and
we show that also \eqref{eqn:two} implies \eqref{eqn:four}. 
Let $M$ be arbitrary. We then have some $N$ satisfying \eqref{eqn:two}.
Clearly, $N \subseteq \match_t(M)$. Since $\val_{\Valid}(t)$ is upward closed,
$N \in \val_{\Valid}(t)$ implies $\match_t(M) \in \val_\nu(t)$.
Due to the upward closeness of both $\val_{\mu}(s)$ and $\val_{\Valid}(t)$
in the case of a~positive left-hand side, 
the equivalence of \eqref{eqn:three} and \eqref{eqn:five} follows.
\end{proof}

\begin{theorem}\label{thm:ref-coincide}
Modal refinement as defined on PMTS coincides with the standard
modal refinement notions on MTS, DMTS and OTS.
On implementations it coincides with %strong 
bisimulation.
\end{theorem}
\begin{proof}%[for MTS; for the rest, see Appendix.]
The fact that Definition~\ref{def:pers-mr} coincides with modal refinement
on OTS as defined in~\cite{benes_et_al:OASIcs:2011:3070} is a~straightforward
corollary of Lemma~\ref{lem:ref-positive} and its proof. 
Indeed, the two conditions given
in~\cite{benes_et_al:OASIcs:2011:3070} are exactly conditions~\eqref{eqn:five}
and~\eqref{eqn:two}. As the definition of modal refinement on OTS
coincides with modal refinement on DMTS (as shown in~\cite{benes_et_al:OASIcs:2011:3070}) and thus also on MTS, the proof 
is done.

However, for the reader's convenience, we present a~direct proof
that Definition~\ref{def:pers-mr} coincides with modal refinement
on MTS.
Assume a parameter-free PMTS $(S,T,P,\UPhi)$ where
$\UPhi(s)$ is a conjunction of transitions for all $s \in S$, in
other words it is a standard MTS where the must transitions
are listed in the conjunction and the may transitions are simply
present in the underlying transition system but not a part of
the conjunction.
Observe that %$P = \emptyset$, and 
every transition $(s,a,t) \in T$ is contained in some 
$M \in \val_\emptyset(s)$. 
Further, each must transition $(s,a,t) \in T$  is contained in all 
$M \in \val_\emptyset(s)$. Therefore, the first conjunct in 
Definition~\ref{def:pers-mr} requires that for all may transition from $s$ 
there be a corresponding one from $t$ with the successors in the refinement 
relation. Similarly, the second conjunct now requires that for all must 
transitions from $t$ there be a corresponding must transition from $s$.
This is exactly the standard notion of modal refinement as introduced
in~\cite{DBLP:conf/lics/LarsenT88}.  
\end{proof}
 


\section{Complexity of Modal Refinement Checking} \label{ss:complexity}

We shall now investigate the complexity of refinement checking
on PMTS and its relevant subclasses. Without explicitly mentioning it,
we assume that all considered PMTS 
are now finite and the decision problems are hence well defined.
The complexity bounds include classes from the polynomial hierarchy
(see e.g.~\cite{papadimitriou1994cc}) where for example 
$\SigmaP_0 = \PiP_0 = $ P, $\PiP_1 = \coNP$ and $\SigmaP_1 = \NP$. 

\begin{table}[bh]
\footnotesize
\centering

    \begin{tabular}{|>{\columncolor{lightlightblue}}c|>{\columncolor{lightlightblue}}c|>{\columncolor{lightlightblue}}c|>{\columncolor{lightlightblue}}c|>{\columncolor{lightlightblue}}c|>{\columncolor{lightlightblue}}c|}\hline
	{\cellcolor{lightblue}}&{\cellcolor{lightblue}}  Boolean  	& {\cellcolor{lightblue}}Positive 	& {\cellcolor{lightblue}}pCNF 	& {\cellcolor{lightblue}}pDNF &{\cellcolor{lightblue}} MTS\\\hline
    {\cellcolor{lightblue}}&&& $\in$ coNP && $\in$ coNP \\
    \multirow{-2}{*}{{\cellcolor{lightblue}}~Boolean~} & 
        \multirow{-2}{*}{$\PiP_2$-complete} & 
        \multirow{-2}{*}{\ coNP-complete\ } & P-hard & 
        \multirow{-2}{*}{\ coNP-complete\ } & P-hard \\
    
    \hline
    {\cellcolor{lightblue}}Positive& $\PiP_2$-complete	& coNP-complete	& ~P-complete~	& coNP-complete & ~P-complete~
    \\\hline
    {\cellcolor{lightblue}}pCNF	& $\PiP_2$-complete	& coNP-complete	& P-complete & coNP-complete & P-complete
    \\\hline
    {\cellcolor{lightblue}}pDNF	& $\PiP_2$-complete	& P-complete		& P-complete	& P-complete & P-complete
    \\\hline
    {\cellcolor{lightblue}}MTS	& $\PiP_2$-complete	& P-complete		& P-complete	& P-complete & P-complete
    \\\hline
    {\cellcolor{lightblue}}Impl	& NP-complete	& P-complete		& P-complete	& P-complete & P-complete
    \\\hline
    \end{tabular}

\caption{Complexity of modal refinement checking of parameter-free systems.}
\label{tbl:mr-compl}
\end{table}

\subsection{Parameter-Free Systems}


Since even the parameter-free systems have interesting expressive
power and the complexity of refinement on OTS has not been studied
before, 
we first focus on parameter-free systems.

Moreover, the results of this subsection are then applied
to parametric systems in the next subsection.

The results are summarized in Table~\ref{tbl:mr-compl}.
The rows in the table correspond 
to the restrictions on the left-hand side PMTS while the columns correspond 
to the restrictions on the right-hand side PMTS. Boolean denotes the general 
system with arbitrary negation. Positive denotes the positive systems, 
in this case exactly OTS. We use pCNF and pDNF to denote positive systems 
with formulae in conjunctive and disjunctive normal forms, respectively. 
In this case, pCNF coincides with DMTS. The special case of satisfaction 
relation, where the refining system is an implementation is denoted by Impl.
We do not include Impl to the columns as it makes sense that 
an implementation is refined only to an implementation and here modal
refinement corresponds to 
bisimilarity that is 
P-complete~\cite{balcazar1992dbp} (see also \cite{SawaJ05}). 
The P-hardness is hence the obvious lower bound for all the 
problems mentioned in the table. 





We start with the simplest NP-completeness result. 

\begin{proposition} 
Modal refinement between an implementation and a\linebreak parameter-free PMTS is NP-complete.
\end{proposition}
\begin{proof}
The containment part is straightforward. 
First we guess the relation $R$.
As $s$ is an implementation then 
the set $\val_{\emptyset}(s)$ is a singleton.
We thus only need to further guess $N \in \val_{\Valid}(t)$ and then in polynomial 
time verify the two conjuncts in Definition~\ref{def:pers-mr}.

The hardness part is by a simple reduction from SAT. 
Let $\varphi(x_1, \ldots, x_n)$ be an given Boolean formula (instance of SAT).
We construct two PMTSs $(S,T,P,\UPhi)$ and \linebreak  $(S',T',P',\UPhi')$ such that 
\begin{enumerate}
    \item $S = \{s,s'\}, T={(s,a,s')}, P= \emptyset$, $\UPhi(s) = (a,s')$ and $\UPhi(s') = \true$ and 
    \item $S'= \{t, t_1, \ldots\ , t_n\}$, $T = \{(t,a,t_i) \mid 1 \le i \le n.\}$, 
$P' = \emptyset$, $\UPhi(t) = \varphi[(a,t_i)/x_i]$ and $\UPhi(t_i) = \true$ 
for all $i$, $1\le i \le n$.
\end{enumerate}
Clearly, $\varphi$ is satisfiable if and only if $s \mr t$.
\end{proof}


Next we show that modal refinement is $\PiP_2$-complete. 
The following proposition introduces a gadget used also later on in other 
hardness results. We will refer to it as the \emph{$*$-construction}. 

\begin{proposition} 
\label{prop:BB} 
Modal refinement between two parameter-free PMTS is 
$\PiP_2$-hard even if the left-hand side is
an~MTS.

\end{proposition}
\begin{proof}
The proof is by polynomial time reduction from the 
validity of the quantified Boolean 
formula $\psi \equiv \forall x_1 \ldots \forall x_n 
\exists y_1 \ldots \exists y_m : 
\varphi(x_1, \ldots, x_n, y_1, \ldots, y_m)$ 
to the refinement checking problem $s\mr t$
where $s$ and $t$ are given as follows. 

\medskip


\begin{center}
\begin{tikzpicture}[->,>=stealth',initial text=,xscale=0.96]
    \node[initial,state] (s) at(-3.5,0) {$s$};
    \node[state] (s2) at(-3.5,-2) {$s'$};
    \node at (-3.5, -1) {$\cdots$};  
    
    \path (s) edge [bend right=80] node [left] {$x_1$} (s2);
    \path (s) edge [bend right=25] node [left] {$x_2$} (s2);
    \path (s) edge [bend left=25] node [right] {$x_n$} (s2);
    \path (s) edge [bend left=80] node [right] {$*$} (s2);

    \node at(-3.5,.65) {$\UPhi(s)=(*,s')$};



    \tikzstyle{every node}=[font=\small] 
    
    \node[initial,state] (t) at(0,0) {$t$};
    \node[state] (tp) at(0,-2) {$t'$};
    \node[state] (t1) at (2.2,-2) {$t_1$};
    \node[state] (t2) at (3.9,-2) {$t_2$};
    \node[state] (tm) at (6.3,-2) {$t_m$};
    \node at (0, -1) {$\cdots$};    
    \node at (5.1, -2) {$\cdots$};    
   
    
    \path (t) edge [bend right=80] node [left] {$x_1$} (tp);
    \path (t) edge [bend right=25] node [left] {$x_2$} (tp);
    \path (t) edge [bend left=25] node [right] {$x_n$} (tp);
    \path (t) edge [bend left=80] node [right] {$*$} (tp);   
    \path (t.east) edge [bend left=40] node [right,pos=.745] {$*$} (t1.north);  
    \path (t.east) edge [bend left=30] node [right,pos=.815] {$*$} (t2.north);   
    \path (t.east) edge [bend left=17] node [right,pos=.825] {$*$} (tm.north);   

    \node [right] at(-.75,.65) {$\UPhi(t)=\varphi[(x_i,t')/x_i,(*,t_i)/y_i]$};




\end{tikzpicture}
\end{center}


















\smallskip

Assume that $\psi$ is true. Let 
$M \in \val_{\emptyset}(s)$ (clearly $(*,s') \in M$) and we want to argue
that there is $N \in \val_{\emptyset}(t)$ with $(*,t') \in N$ such that
for all $(x_i,s') \in M$  there is $(x_i,t') \in N$ (clearly the 
states $s'$, $t'$ and $t_i$ are in modal refinement)
and for all $(x_i,t') \in N$ there is $(x_i,s') \in M$.
Such an $N$ can be found by simply including 
$(x_i,t')$ whenever $(x_i,s') \in M$ and by adding
also $(*,t')$ into $N$. As $\psi$ is true, we include into $N$
also all $(*,t_i)$ whenever $y_i$ is set to true in $\psi$.
Hence we get $s \mr t$.

On the other hand if $\psi$ is false then we pick
$M \in \val_{\emptyset}(s)$ such that $M$ corresponds to the
values of $x_i$'s such that there are no values of 
$y_1, \ldots, y_m$ that make $\psi$ true. This means that from 
$t$ there will be no transitions as $\val_{\emptyset}(t)= \emptyset$
assuming that $(x_i,t')$ have to be set to true whenever
$(x_i,s') \in M$, otherwise the refinement between $s$ and $t$ will fail.
However, now $(*,s') \in M$ cannot be matched from 
$t$ and hence $s \not\mr t$.





\end{proof}

\begin{proposition} \label{prop:coinduction}
Modal refinement between two parameter-free PMTS is in $\PiP_2$.
\end{proposition}
\begin{proof}
The containment follows directly from Definition~\ref{def:pers-mr} 
(note that the parameters are empty) and 
the fact that the last conjunction 
in Definition~\ref{def:pers-mr} is polynomially verifiable 
once the sets $M$ and $N$ were fixed.
The relation $R$ could in principle be guessed before
it is verified, however, this would increase the complexity bound
to $\SigmaP_3$.  Instead, we will initially include all 
pairs (polynomially many) into $R$ and for each pair 
ask whether for every $M$ there is $N$ such that the two
conjuncts are satisfied. If it fails, we remove the pair and 
continue until we reach (after
polynomially many steps) the greatest fixed point. The complexity in
this way remains in $\PiP_2$. We shall use this standard method 
also in further proofs
and refer to it as a co-inductive computation of $R$.


\end{proof}


\subsubsection{Positive Right-Hand Side.}
We have now solved all the cases where the right-hand side is arbitrary.
We now look at the cases where the right-hand side is positive.
In the proofs that follow we shall use the alternative characterization
of refinement from Lemma~\ref{lem:ref-positive}.
The following proposition determines the subclasses on which modal refinement 
can be decided in polynomial time.

\begin{proposition}
Modal refinement on parameter-free PMTS is in P, provided that 
both sides are positive and 
either the left-hand side is in pDNF or the right-hand side is in pCNF.
\end{proposition}
\begin{proof}
Due to Lemma~\ref{lem:ref-positive}, the refinement is equivalent to the
conjunction of \eqref{eqn:five} and \eqref{eqn:four}. Clearly, \eqref{eqn:five} can be checked in P. 
We show that Condition \eqref{eqn:four} can be verified in P too.
Recall 
that \eqref{eqn:four}  says that
\begin{align*}
 \forall M \in \val_\mu(s) : \match_t(M) \in \val_\nu(t) 
\end{align*}
where $\match_t(M) = \{ (a,t') \in T(t) \mid \exists (a,s') \in M :
(s',t') \in R\}$.

First assume that the left-hand side is in pDNF.
If for some $M$ the Condition~\eqref{eqn:four}  
is satisfied then it is also
satisfied for all $M' \supseteq M$, as $\val_{\mu}(s)$ is upwards closed.
It it thus sufficient to verify the condition 
for all minimal elements (wrt. inclusion)
of $\val_{\mu}(s)$. In this case it correspond to the clauses of $\UPhi(s)$.
Thus we get a polynomial time algorithm as shown 
in Algorithm~\ref{alg:poly1}.

\begin{algorithm}[ht]
\SetKwInOut{Input}{Input}
\SetKwInOut{Output}{Output}
\Input{states $s$ and $t$ such that $\UPhi(s)$ is in positive DNF and
$\UPhi(t)$ is positive, relation $R$}
\Output{\emph{true} if $s$, $t$ satisfy the refinement condition, \emph{false} otherwise}

\ForEach{clause $(a_1,s_1) \wedge \cdots \wedge (a_k,s_k)$ in $\UPhi(s)$}{

	$N \gets \{(a,t') \in T(t) \mid \exists i : a_i = a \wedge (s_i,t') \in R\}$\;

	\lIf{$N \not\in \val_\nu(t)$}{\Return {\em false}\;}
}
\Return {\em true}\;
\caption{Test for Condition \eqref{eqn:four} of modal 
refinement (pDNF) \label{alg:poly1}} 
\end{algorithm}

Second, assume that the right-hand side is in pCNF. 
Note that Condition~\eqref{eqn:four} can be equivalently stated as
\begin{align}
\forall M : \match_t(M) \not\in\val_\nu(t) \Rightarrow M \not\in \val_\mu(s) \label{eqn:fourEq}
\end{align}
As $\UPhi(t)$ is in conjunctive normal form then 

$N \in \val_\nu(t)$ is equivalent to saying that

$N$ has nonempty
intersection with each clause of $\UPhi(t)$. 
We may thus enumerate all maximal $N \not\in \val_\nu(t)$.
Having a~maximal $N \not\in \val_\nu(t)$, we 
can easily construct $M$ such that $N = \match_t(M)$.
This leads to the polynomial time Algorithm~\ref{alg:poly2}.

\begin{algorithm}[h]
\SetKwInOut{Input}{Input}
\SetKwInOut{Output}{Output}
\Input{states $s$ and $t$ such that $\UPhi(s)$ is positive and
$\UPhi(t)$ is in positive CNF, relation $R$}
\Output{\emph{true} if $s$, $t$ satisfy the refinement condition, \emph{false} otherwise}

\ForEach{clause $(a_1,t_1) \vee \cdots \vee (a_k,t_k)$ 
	in $\UPhi(t)$}{

	$M \gets T(s) \setminus \{(a,s') \in T(s) \mid \exists i : a_i = a \wedge (s',t_i) \in R\}$\;

	\lIf{$M \in \val_\mu(s)$}{\Return {\em false}\;}
}
\Return {\em true}\;
\caption{Test for Condition~\eqref{eqn:four} of modal
refinement (pCNF)\label{alg:poly2}} 
\end{algorithm}
The statement of the proposition thus follows.

\end{proof}



























\begin{proposition}
Modal refinement on parameter-free PMTS is in coNP, if the 
right-hand side is positive.
\end{proposition}
\begin{proof}
Due to Lemma~\ref{lem:ref-positive} we can solve the two refinement conditions separately.
Furthermore, both Condition \eqref{eqn:three} an \eqref{eqn:four} of Lemma~\ref{lem:ref-positive} can be checked in coNP. The guessing of $R$ is 
done co-inductively as described in the proof of 
Proposition~\ref{prop:coinduction}.

\end{proof}

\begin{proposition}
Modal refinement on parameter-free systems is coNP-hard, 
even if the left-hand side is in positive CNF
and the right-hand side is in positive DNF.
\end{proposition}
\begin{proof}
We reduce SAT into non-refinement.
Let $\varphi(x_1, \ldots, x_n)$ be a~formula in CNF.
We modify $\varphi$ into an equivalent formula $\varphi'$ as follows:
add new variables $\tilde x_1$, \ldots, $\tilde x_n$
and for all $i$ change all occurrences of $\neg x_i$ into $\tilde x_i$
and add new clauses
$(x_i \vee \tilde x_i)$ and $(\neg x_i \vee \neg \tilde x_i)$.

Observe now that all clauses contain either all positive literals 
or all negative literals. 
Let $\psi^+$ denote a CNF formula that contains all positive clauses
of $\varphi'$ and $\psi^-$ denote a CNF formula that contains all 
negative clauses of $\varphi'$.
As $\varphi' = \psi^+ \wedge \psi^-$ it is clear that 
$\varphi'$ is satisfiable if and only if
$(\psi^+ \Rightarrow \neg \psi^-)$ is not valid.

Now we construct two 
PMTSs $(S,T,P,\UPhi)$ and $(S',T',P',\UPhi')$ 
over \linebreak $\USigma = \{ x_1, \ldots, x_n, \tilde x_1, \ldots, \tilde x_n \}$
 as follows: 
\begin{enumerate}
    \item  $S=\{s,s'\}$, 
    $T= \{(s,x_i,s'), (s,\tilde x_i, s') \mid 1 \leq i \leq n\}$, \\
    $P = \emptyset$,
    $\UPhi(s) = \psi^+[(x_i,s')/x_i,(\tilde x_i,s')/\tilde x_i]$ 
    and $\UPhi(s') = \true$, and
    \item $S' = \{t,t'\}$, 
    $T' = \{ (t,x_i,t'),(t,\tilde x_i,t) \mid 1 \leq i \leq n\}$, \\
    $P' = \emptyset$, 
    $\UPhi(t) = \neg \psi^-[(x_i,t')/x_i,(\tilde x_i,t')/\tilde x_i]$ and 
$\UPhi(t') = \true$.
\end{enumerate}
 Note that by pushing the negation of $\psi^-$ inside, 
 this formula can be written as pDNF.
It is easy to see that now $s \mr t$ if and only if 
$(\psi^+ \Rightarrow \neg \psi^-)$ is valid.
Therefore, $s \not\mr t$ if and only 
if $\varphi$ is satisfiable.

\end{proof}














Note that the exact complexity of modal refinement with the right-hand 
side being in positive CNF or MTS and the left-hand side Boolean remains open. 

\subsection{Systems with Parameters}

In the sequel we investigate the complexity of refinement checking in the 
general case of PMTS with parameters. The complexities are summarized 
in Table~\ref{tbl:pmr-compl}.
\begin{table}[b!]
\centering
\begin{tabular}{|>{\columncolor{lightlightblue}}c|>{\columncolor{lightlightblue}}c|>{\columncolor{lightlightblue}}c|>{\columncolor{lightlightblue}}c|>{\columncolor{lightlightblue}}c|}\hline
	{\cellcolor{lightblue}}&{\cellcolor{lightblue}} Boolean 	&{\cellcolor{lightblue}} positive 	&{\cellcolor{lightblue}} pCNF 	&{\cellcolor{lightblue}} pDNF \\\hline
    {\cellcolor{lightblue}}&&& $\in \PiP_3$ & \\
    \multirow{-2}{*}{{\cellcolor{lightblue}}~Boolean~} & 
	\multirow{-2}{*}{$\PiP_4$-complete} & 
	\multirow{-2}{*}{$\PiP_3$-complete} & $\PiP_2$-hard &
	\multirow{-2}{*}{$\PiP_3$-complete} \\
\hline
{\cellcolor{lightblue}}positive& $\PiP_4$-complete& $\PiP_3$-complete& $\PiP_2$-complete &$\PiP_3$-complete
\\\hline
{\cellcolor{lightblue}}pCNF	& $\PiP_4$-complete& $\PiP_3$-complete & $\PiP_2$-complete &$\PiP_3$-complete
\\\hline
{\cellcolor{lightblue}}pDNF	& $\PiP_4$-complete& $\PiP_2$-complete & $\PiP_2$-complete & $\PiP_2$-complete
\\\hline
{\cellcolor{lightblue}}MTS	& $\SigmaP_3$-complete &  NP-complete & NP-complete & NP-complete
\\\hline
{\cellcolor{lightblue}}Impl	& NP-complete & NP-complete & NP-complete & NP-complete
\\\hline
\end{tabular}
\caption{Complexity of modal refinement checking with parameters.}
\label{tbl:pmr-compl}
\end{table}
We start with an observation of how the results on parameter-free systems 
can be applied to the parametric case.


\begin{proposition}
The complexity upper bounds from Table~\ref{tbl:mr-compl} carry over to Table~\ref{tbl:pmr-compl},
as follows.
If the modal refinement in the parameter-free case is in \emph{NP}, 
\emph{coNP} or $\PiP_2$,
then the modal refinement with parameters is in $\PiP_2$, $\PiP_3$ and 
$\PiP_4$, respectively.
Moreover, if the left-hand side is an MTS, the complexity upper bounds shift
from \emph{NP} and $\PiP_2$ to \emph{NP} and $\SigmaP_3$, respectively.




\end{proposition}
\begin{proof}

In the first case, we first universally choose $\mu$, we then 
existentially choose $\nu$ and modify the formulae $\UPhi(s)$ and $\UPhi(t)$ 
by evaluating the parameters. This does not change the normal 
form/positiveness of the formulae.
We then perform the algorithm for the parameter-free refinement.

For the second case note that implementations and MTS have no parameters and
we may simply choose (existentially) $\nu$ and run the algorithm for the
parameter-free refinement. 

\end{proof}


We now focus on the respective lower bounds.






\begin{proposition}
Modal refinement between an implementation and a \linebreak right-hand side
in positive CNF or in DNF is NP-hard.
\end{proposition}
\begin{proof}
The proof is by reduction from SAT. 
Let $\varphi(x_1,\ldots,x_n)$ be a~formula in CNF
and let $\varphi_1$, $\varphi_2$, \ldots, $\varphi_k$ be the clauses of 
$\varphi$.
We construct two PMTSs $(S,T,P,\UPhi)$ and $(S',T',P',\UPhi')$ over the
action alphabet $\USigma = \{a_1, \ldots, a_k\}$
as follows: 
\begin{enumerate}
    \item $S = \{s,s'\}$, 
    $T = \{(s, a_i, s') \mid 1 \le i \le k\}$,\\
    $P = \emptyset$, 
    $\UPhi(s) = \bigwedge_{1 \le i \le k} (a_i,s')$ and
    $\UPhi(s') = \true$ and
    \item $S' = \{t\} \cup \{t_i \mid 1 \le i \le k\}$,
    $T' = \{(t,a_i,t_i) \mid 1 \le i \le k\}$, \\
    $P' = \{ x_1, \ldots, x_n \}$,
    $\UPhi'(t) = \bigwedge_{1 \le i \le k} (a_i,t_i)$
    and $\UPhi'(t_i) = \varphi_i$ for all $1 \le i \le k$.
\end{enumerate}
Notice that each $\varphi_i$ in $\UPhi'(t_i)$ is in positive
form as we negate only the parameters $x_i$ and every clause
$\varphi_i$ is trivially in DNF. 
Now we easily get that 
$s \mr t$ if and only if $\varphi$ is satisfiable.

\end{proof}



\begin{proposition}\label{p:lm}
Modal refinement is $\SigmaP_3$-hard even if the left-hand side is MTS.
\end{proposition}
\begin{proof}
The proof is done using the~construction of the proof
of Proposition~\ref{prop:BB} with parameters added on the right-hand side.

We will make a reduction from the validity of the quantified Boolean
formula $\psi \equiv \exists z_1, \ldots, z_k : \forall x_1 \ldots \forall x_n 
\exists y_1 \ldots \exists y_m :$
$\varphi(z_1, \ldots, z_k, x_1, \ldots, x_n, y_1, \ldots, y_m)$
to the refinement checking problem $s \mr t$
where $s$ and $t$ are given as follows.
Moreover, the right-hand side system has $\{z_1, \ldots, z_k\}$
as its set of parameters.

\medskip

\begin{tikzpicture}[->,>=stealth',initial text=]
\node[initial,state] (s) at(-3.5,0) {$s$};
    \node[state] (s2) at(-3.5,-2) {$s'$};
    \node at (-3.5, -1) {$\cdots$};  
    
    \path (s) edge [bend right=80] node [left] {$x_1$} (s2);
    \path (s) edge [bend right=25] node [left] {$x_2$} (s2);
    \path (s) edge [bend left=25] node [right] {$x_n$} (s2);
    \path (s) edge [bend left=80] node [right] {$*$} (s2);

    \node at(-3.5,.65) {$\UPhi(s)=(*,s')$};


    \tikzstyle{every node}=[font=\small] 
    
    \node[initial,state] (t) at(0,0) {$t$};
    \node[state] (tp) at(0,-2) {$t'$};
    \node[state] (t1) at (2.2,-2) {$t_1$};
    \node[state] (t2) at (3.9,-2) {$t_2$};
    \node[state] (tm) at (6.1,-2) {$t_m$};
    \node at (0, -1) {$\cdots$};    
    \node at (5.1, -2) {$\cdots$};    
    
    \path (t) edge [bend right=80] node [left] {$x_1$} (tp);
    \path (t) edge [bend right=25] node [left] {$x_2$} (tp);
    \path (t) edge [bend left=25] node [right] {$x_n$} (tp);
    \path (t) edge [bend left=80] node [right] {$*$} (tp);   
    \path (t.east) edge [bend left=40] node [right,pos=.745] {$*$} (t1.north);  
    \path (t.east) edge [bend left=30] node [right,pos=.815] {$*$} (t2.north);   
    \path (t.east) edge [bend left=17] node [right,pos=.825] {$*$} (tm.north);   

    \node [right] at(-.75,.65) {$\UPhi(t)=\varphi[(x_i,t')/x_i,(*,t_i)/y_i]$};
\end{tikzpicture}

\smallskip

Assume that $\psi$ is true. Then there exists a~valuation $\nu$ on 
$\{z_1, \ldots, z_k\}$ such that $\forall x : \exists y : \varphi(x,y)$
is true. Using now
the same argument as that in the proof of Proposition~\ref{prop:BB},
we get that $s \mr t$.

On the other hand let $\psi$ be false and let $\nu$ be an
arbitrary valuation on $\{z_1, \ldots, z_k\}$. We then may
again use the reasoning in the proof of Proposition~\ref{prop:BB}
to get that $s \not\mr t$.
\end{proof}


The following proof introduces a gadget used also later on in other hardness results. We refer to it as \emph{CNF-binding}. Further, we use
the $*$-construction here.

\begin{proposition}\label{p:lc}
Modal refinement is $\PiP_4$-hard even if the left-hand side is in positive CNF.
\end{proposition}
\begin{proof}[Sketch]
Consider a $\PiP_4$-hard QSAT instance, 
a formula $\psi=\forall x \exists y \forall z \exists w : \varphi(x,y,z,w)$ with $\varphi$ is in CNF and $x,y,z,w$ vectors of length $n$.
We construct two system $s$ and $t$ and use the variables 
$\{ x_1, \ldots, x_n \}$ as parameters for the 
left-hand side system $s$, and $\{ y_1, \ldots, y_n \}$ as parameters for the 
right-hand side system $t$. 
\begin{center}
\begin{tikzpicture}[->,>=stealth',initial text=,yscale=0.8]
\node[initial,state] (s) at(-3.5,0) {$s$};
    \node[state] (s2) at(-3.5,-2) {$s'$};
    

    \path (s) edge [bend right=80] node [left] {$t_i$} (s2);
    \path (s) edge [bend right=25] node [left] {$f_i$} (s2);
    \path (s) edge [bend left=25] node [right] {$z_i$} (s2);
    \path (s) edge [bend left=80] node [right] {$*$} (s2);

    \node at(-3.5,.8) {$\UPhi(s)=(*,s')\ \wedge\ $CNF-binding};



    \tikzstyle{every node}=[font=\small]

    \node[initial,state] (t) at(0,0) {$t$};
    \node[state] (tp) at(0,-2) {$t'$};
    \node[state] (t1) at (2.2,-2) {$u_i$};
    
    
    
    
   

    \path (t) edge [bend right=80] node [left] {$t_i$} (tp);
    \path (t) edge [bend right=25] node [left] {$f_i$} (tp);
    \path (t) edge [bend left=25] node [right] {$z_i$} (tp);
    \path (t) edge [bend left=80] node [right] {$*$} (tp);
    \path (t.east) edge [bend left=40] node [right,pos=.745] {$*$} (t1.north);

    \node [right] at(-.75,.8) {$\UPhi(t)=(*,s')\wedge \varphi[(t_i,t')/x_i,(f_i,t')/\neg x_i,$};
    \node [right] at(1.70,0.1) {$(z_i,t')/z_i,(*,u_i)/w_i]$};

\node at(4.6,-1.2) {for all $1\leq i\leq n$};
\end{tikzpicture}
\end{center}

On the left we require $\UPhi(s) = (*,s')\wedge \bigwedge_{1 \le i \le n} 
\Big((x_i \Rightarrow (t_i,s')) \wedge (\neg x_i \Rightarrow (f_i,s')\Big)$ and
call the latter conjunct \emph{CNF-binding}. 
Thus the value of each parameter $x_i$ is ``saved'' into transitions of the system. Note that although both $t_i$ and $f_i$ may be present, a ``minimal'' implementation contains exactly one of them. On the right-hand side 
the transitions look similar but we require
$\UPhi(t) = (*,t)\wedge\varphi'$ 
where $\varphi'$ is created from $\varphi$ by changing
every positive literal $x_i$ into $(t_i,t')$,
every negative literal $\neg x_i$ into $(f_i,t')$,
every $z_i$ into $(z_i,t')$, and
every $w_i$ into $(*,u_i)$.

We show that $\psi$ is true iff $s\mr t$. Assume first that $\psi$ is true. 
Therefore, for every choice of parameters $x_i$ there is a choice of parameters $y_i$ so that $\forall z \exists w : \varphi(x,y,z,w)$ is true and, moreover, $t_i$ or $f_i$ is present on the left whenever $x_i$ or $\neg x_i$ is true, respectively (and possibly even if it is false). We set exactly all these transitions $t_i$ and $f_i$ on the right, too. Further, for every choice of transitions $z_i$ on the left there are $w_i$'s so that $\varphi(x,y,z,w)$ holds. On the right, we implement a transition $(z_i,t')$ for each $z_i$ set to true and $(*,u_i)$ for each $w_i$ set to true. Now $\varphi'$ is satisfied 
as it has only positive occurrences of $(t_i,t')$
and $(f_i,t')$ and hence the extra $t_i$'s and $f_i$'s do not matter. 
Now for every implementation of $s$ we obtained an implementation of $t$. Moreover, their transitions match. Indeed, $t_i$'s and $f_i$'s were set the same as on the left, similarly for $z_i$'s. As for the $*$-transition, we use the same argumentation as in the original $*$-construction. On the left, there is always one. On the right, there can be more of them due to $w_i$'s but at least one is also guaranteed by $\UPhi(t)$.

Let now $s\mr t$. Then for every choice of $x_i$'s---and thus also for every choice of \emph{exactly} one transition of $t_i,f_i$ for each $i$---there are $y_i$'s so that every choice of transitions $z_i$ can be matched on the right so that $\varphi'$ is true with some transitions $(*,u_i)$. Since choices of $t_i/f_i$ correspond exactly to choices of $x_i$ it only remains to set $w_i$ true for each transition $(*,u_i)$ on the right, thus making $\varphi$ true.

\end{proof}


We now prove Proposition~\ref{p:lcrc}, followed by Proposition~\ref{p:lcrd}. The reason
for this ordering is that the setting of Proposition~\ref{p:lcrc} is simpler and the
proof involves just one method, namely that of the CNF-binding, 
no $*$-construction is used and no additional actions are needed.


\begin{proposition}\label{p:lcrc}
Modal refinement is $\PiP_2$-hard even if both sides are in positive CNF.
\end{proposition}
\begin{proof}
Recall that positive means that there may be negations, but only limited
to parameter literals.
The proof is done by reduction from the validity of \linebreak  
$\forall x_1, \ldots, x_n \exists y_1, \ldots, y_m : 
\varphi(x_1, \ldots, x_n, y_1, \ldots, y_m)$, where $\varphi$ is in CNF.
The idea is that the left-hand side has only $x_i$ as parameters
while the right-hand side has $y_i$ as parameters.
To ensure that the valuation of $x_i$ is the same
on both sides, we bind them through transitions.

Let 
$\USigma = \{t_1, \ldots, t_n, f_1, \ldots, f_n\}$
be the set of actions.
The systems $(S,T,P,\UPhi)$ and $(S',T',P',\UPhi')$ are built as follows: 
\begin{enumerate}
    \item $S = \{s, s'\}$,\\ 
    $T = \{(s,t_i,s'), (s,f_i,s') \mid
1 \le i \le n \}$,\\ 
    $P = \{x_1, \ldots, x_n\}$,\\
    $\UPhi(s) = \bigwedge_{1 \le i \le n} ((x_i \Rightarrow (t_i,s')) \wedge (\neg x_i \Rightarrow (f_i,s'))$ (note that this may be written in positive CNF),\\
    $\UPhi(s') = \true$;
    \item $S' = \{ t, t' \}$,\\
    $T' = \{ (t,t_i,t'), (t,f_i,t') \mid 1 \le i \le n \}$,\\
    $P' = \{y_1, \ldots, y_m\}$,\\

    $\UPhi'(t) = \varphi[(t_i,t')/x_i,(f_i,t')/\neg x_i]$,\\

    $\UPhi'(t') = \true$.
\end{enumerate}



We now claim that $\forall x \exists y : \varphi$ holds
if and only if $s \mr t$.
We show the two implications separately. 

Let first $\forall x \exists y : \varphi$ hold.
Let $\mu \subseteq P_1$ be arbitrary. As this is a truth valuation
on the $x_i$ variables, we know that there exists a valuation
on the $y_i$ variables such that $\varphi$ holds. Let $\nu \subseteq P_2$
be such a valuation.
Let further $M \in \val_\mu(s)$ be arbitrary.
Clearly, if $x_i \in \mu$ then $(t_i,s') \in M$ and 
if $x_i \not\in \mu$ then $(f_i,s') \in M$.

We set \[N = \{ (x,t') \mid (x,s') \in M\}\]. Clearly, such $N$ satisfies
both conjuncts of the refinement definition. We need to show that 
$N \in \val_\nu(t)$. We thus need to show that $N$ satisfies
all the clauses in $\UPhi'(t)=\varphi[(t_i,t')/x_i,(f_i,t')/\neg x_i]$.

We use the fact that $\varphi$ holds, given the current valuation
$\mu$ on $x_i$ and $\nu$ on $y_i$.
Let $(\ell_1 \vee \ell_2 \vee \cdots \vee \ell_k)$ be an arbitrary clause
of $\varphi$. Clearly, at least one literal was satisfied.
If that literal was $y_i$ or $\neg y_i$ then the same literal appears
in the modified clause of $\UPhi'(t)$ and we are done.
If that literal was $x_i$ then it has been changed into
$(t_i,t')$, but as $x_i \in \mu$, we have that $(t_i,t') \in N$.
Similarly, if that literal was $\neg x_i$ then it has been 
changed into $(f_i,t')$, but as $x_i \not\in\mu$, we have
that $(f_i,t') \in N$. Thus $s \mr t$.

For the other implication let 
$\exists x \forall y : \neg \varphi$ hold.  We show that $s \not\mr t$.
Let $\mu$ be the valuation of $x_i$ such that 
$\exists x \forall y : \neg \varphi$ holds.
Let $\nu$ be arbitrary. This corresponds to a valuation on $y_i$.

We now set 
\[
M = \{ (t_i, s') \mid x_i \in \mu \} \cup \{ (f_i, s') \mid x_i \not\in \mu \}.
\] 
Clearly, $M\in\val_\mu(s)$. Let further $N \in \val_\nu(t)$. (If $\val_\nu(t) = \emptyset$, we are done.)

We know that given the current $x$ and $y$ valuation, $\varphi$ does not hold.
This means that there exists at least one clause of $\varphi$ that is false.
Let $(\ell_1 \vee \ell_2 \vee \cdots \vee \ell_k)$ be such clause. 
All $\ell_j$ are false, given current valuation $\mu$ and $\nu$.
However, the modified clause of $\UPhi'(t)$ corresponding to this one
is satisfied by $N$ (valuation of $(t_i,t')$ and $(f_i,f')$) as $N\in\val_\nu(t)$.

Therefore, for some $i$,
     either $(t_i,t') \in N$ while $x_i \not \in \mu$
    or $(f_i,t') \in N$ while $x_i \in \mu$.
In both cases $N$ does not satisfy 
the second conjunct part of the modal refinement
definition. Therefore $s \not\mr t$.
\end{proof}

The next proposition again reuses the idea of CNF-binding in the very same fashion as above. Moreover, it handles more actions, more precisely those that appear as $z_i$'s in Proposition~\ref{p:lc}. Thus, the proof is the same, omitting the $*$-construction. Therefore, we only provide the reduction without repeating the formal arguments that it indeed works.

\begin{proposition}\label{p:lcrd}
Modal refinement is $\PiP_3$-hard for the left-hand side 
in positive CNF and the right-hand side in positive DNF.
\end{proposition}
\begin{proof}
The proof is done by reduction from the validity of 
the quantified Boolean formula
$\forall x_1, \ldots, x_k
\exists y_1,\ldots, y_l \forall z_1,\ldots, z_m
: \varphi$ with $\varphi$ in DNF. 

Let the action alphabet be $\USigma = \{t_1, \ldots, t_k, f_1, \ldots, f_k, 
  z_1, \ldots, z_m \}$.

The two systems $(S,T,P,\UPhi)$ and $(S',T',P',\UPhi')$ are built as
follows: 
\begin{enumerate}
    \item $S = \{s, s'\}$, 
    $T = \{ (s,t_i,s'), (s,f_i,s') \mid 1 \le i \le k\} \cup \{ (s, z_j, s') \mid 1 \le j \le m \}$, \\
    $P = \{x_1, \ldots, x_k\}$,
    $\UPhi(s) = \bigwedge_{1 \le i \le n} ((x_i \Rightarrow (t_i,s')) \wedge (\neg x_i \Rightarrow (f_i,s'))$, and
    $\UPhi(s') = \true$;
    \item $S' = \{t, t'\}$, 
    $T' = \{(t,t_i,t'), (t,f_i,t') \mid 1 \le i \le k\} \cup \{(t, z_j, t')\mid 1 \le j \le m\}$, \\
    $P' = \{y_1, \ldots, y_k\}$,
    $\UPhi'(t) = \varphi[(t_i,t')/x_i,(f_i,t')/\neg x_i,(z_i,t')/z_i]$,
    $\UPhi'(t') = \true$.
\end{enumerate}

Now $\forall x \exists y \forall z : \varphi(x,y,z)$ holds if and only if
$s \mr t$.
\end{proof}

We now modify the idea of CNF-binding to \textbf{DNF-binding} where instead of $(x_i \Rightarrow (t_i,s')) \wedge (\neg x_i \Rightarrow (f_i,s'))$ we use $(x_i \wedge (t_i,s')) \vee (\neg x_i \wedge (f_i,s'))$ to bind parameters of 
left-hand side with transitions of right-hand side. 

The binding works slightly differently, as with DNF we are unable to make
a~conjuction of such formulae for all $i$. We thus employ 
a~new special action $\bullet$. The left-hand side then first requires
a~$\bullet$-transition into $n$ different states $s_i$, each requiring
the above formula for its respective $i$.


\begin{proposition}\label{p:ldrc}
Modal refinement is $\PiP_2$-hard even if left-hand side is in positive DNF and
right-hand side is in positive CNF.
\end{proposition}
\begin{proof}
The proof is done by reduction from the validity of the
quantified Boolean formula
 $\forall x_1, \ldots, x_n \exists y_1, \ldots, y_m : 
 \varphi(x_1, \ldots, x_n, y_1, \ldots, y_m)$, where $\varphi$ is in CNF.

 Let the action alphabet be $\USigma = \{ t_1, \ldots, t_n, f_1, \ldots, f_n, \bullet \}$.
The two systems $(S,T,P,\UPhi)$ and $(S',T',P',\UPhi')$ are built as
follows: 
\begin{enumerate}
    \item $S = \{s,s' \} \cup \{s_i \mid 1 \le i \le n\}$,
$T=\{(s,\bullet,s_i), (s_i, t_i, s'), (s_i, f_i, s') \mid 1 \le i \le n\}$, \\
$P =  \{ x_1, \ldots, x_n\}$,
 $\UPhi(s) = \bigwedge_i (\bullet,s_i)$,
 $\UPhi(s_i) = (x_i \wedge (t_i,s')) \vee (\neg x_i \wedge (f_i,s'))$,
 $\UPhi(s') = \true$;

\item 
$S' = \{t,t' \} \cup \{u_i,v_i \mid 1 \le i \le n \}$,\\
$T' = \{(t,\bullet,u_i), (t,\bullet,v_i), (u_i,t_i,t'), (u_i,f_i,t'),\allowbreak (v_i,f_i,t'), (v_i,t_i,t') \mid 1 \le i \le n\}$,\\
$P' = \{y_1, \ldots, y_n \}$,
$\UPhi'(t) = \varphi[(\bullet,u_i)/x_i,(\bullet,v_i)/\neg x_i]$,\\
 $\UPhi'(u_i) = (t_i,t')$, $\UPhi'(v_i) = (f_i,t')$, $\UPhi'(t') = \true$.
\end{enumerate}

 Now $\forall x \exists y : \varphi(x,y)$ holds if
 and only if $s \mr t$.
 The reasoning behind this fact is similar to the proof of
 Proposition~\ref{p:lcrc}.

\end{proof}

The proof of the next proposition is only a slight alteration of previous proof where the $\bullet$-construction is performed in two steps.

\begin{proposition}\label{p:ldpd}
Modal refinement is $\PiP_2$-hard even if left-hand side is in positive DNF and
right-hand side is in positive DNF.
\end{proposition}
\begin{proof}

The proof is done by reduction from the validity of the quantified
Boolean formula
$\forall x_1, \ldots, x_n \exists y_1, \ldots, y_m : 
\varphi(x_1, \ldots, x_n, y_1, \ldots, y_m)$, where $\varphi$ is in CNF.
Let $\varphi_1$, \ldots, $\varphi_k$ denote the clauses of $\varphi$.


Let the action alphabet be $\USigma = \{ t_1, \ldots, t_n, f_1, \ldots, f_n, \bullet \}$.
The two systems $(S,T,P,\UPhi)$ and $(S',T',P',\UPhi')$ are built as
follows: 
\begin{enumerate}
    \item 
$S = \{s,s',s'' \} \cup \{s_i \mid 1 \le i \le n\}$,\\
$T = \{(s,\bullet,s')\} \cup \{ (s',\bullet,s_i), (s_i, t_i, s''), 
(s_i, f_i, s'') \mid 1 \le i \le n\}$,\\
$P = \{ x_1, \ldots, x_n\}$,
 
$\UPhi(s) = (\bullet,s')$,
  $\UPhi(s') = \bigwedge_i (\bullet,s_i)$,\\
  $\UPhi(s_i) = (x_i \wedge (t_i,s'')) \vee (\neg x_i \wedge (f_i,s''))$,
  $\UPhi(s'') = \true$;

\item 
$S' = \{t,t'\} \cup \{ u_i, v_i \mid 1 \le i \le n \}
\cup \{ w_j \mid 1 \le j \le k \}$,\\
$T' = \{ (t,\bullet,w_j) \mid  1 \le j \le k \}
\cup \{ (w_j,\bullet,u_i), (w_j,\bullet,v_i) \mid
1 \le i \le n,  1 \le j \le k \}
\\ ~~\qquad \cup \{ (u_i,t_i,t'), (u_i,f_i,t'), (v_i,f_i,t'), (v_i,t_i,t')
\mid 1 \le i \le n \}$, \\
$P' = \{y_1, \ldots, y_n \}$,

$\UPhi'(t) = \bigwedge_j w_j$,
$\UPhi'(w_j) = \varphi_j'$ where $\varphi_j'$ is created from
$\varphi_j$ by changing all positive literals $x_i$ into $(\bullet,u_i)$
and all negative literals $\neg x_i$ into $(\bullet,v_i)$.
$\UPhi'(u_i) = (t_i,t')$, $\UPhi'(v_i) = (f_i,t')$, $\UPhi'(t') = \true$.

\end{enumerate}

Now $\forall x \exists y : \varphi(x,y)$ holds if
and only if $s \mr t$.
\end{proof}

The proof of the last proposition is a~combination of DNF-binding (including
the $\bullet$-construction) with the previously used $*$-construction.

\begin{proposition}\label{p:ld}
Modal refinement is $\PiP_4$-hard even if the left-hand side is in positive DNF.
\end{proposition}

\begin{proof}
The proof is done by reduction from the validity of the quantified
Boolean formula
 $\forall x \exists y \forall z \exists w : \varphi(x,y,z,w)$ where
 $x,y,z,w$ are all $n$-dimensional binary vectors and $\varphi$ is in CNF.

We let
$\USigma = \{t_1, \ldots, t_n, f_1, \ldots, f_n, 
        z_1, \ldots, z_n, *, \bullet \}$
and we create the two systems 
$(S,T,P,\UPhi)$, $(S',T',P',\UPhi')$ over the
action alphabet $\Sigma$
as follows:
\begin{enumerate}
    \item $S = \{s,s'\} \cup \{s_i \mid 1 \le i \le n\}$,\\
$T = \{(s,\bullet,s_i), (s_i,t_i,s'), (s_i,f_i,s'),
(s,z_i,s') \mid 1 \le i \le n\} \cup \{(s,*,s')\}$,\\
$P = \{ x_1, \ldots, x_n \}$, 

$\UPhi(s) = (*,s') \wedge \bigwedge_i (\bullet,s_i)$,\\
$\UPhi(s_i) = (x_i \wedge (t_i,s')) \vee (\neg x_i \wedge (f_i,s'))$
for all $1 \le i \le n$,
$\UPhi(s') = \true$;
\item 
$S' = \{t, t'\} \cup \{u_i, v_i, w_i \mid 1 \le i \le n\}$,\\
$T' = \{(t,z_i,t'), (t,\bullet,u_i), (t,\bullet,v_i),\allowbreak
(t,*,w_i), \allowbreak(u_i,t_i,t'), (u_i,f_i,t'), (v_i,t_i,t'),
\allowbreak
(v_i,f_i,t') \mid 1 \le i \le n\} \cup \{(t,*,t')\}$,\\

$P' = \{ y_1, \ldots, y_n \}$,

$\UPhi'(t) = (*,t') \wedge \allowbreak \varphi[(\bullet,u_i)/x_i,
        (\bullet,v_i)/\neg x_i,
        (z_i,t')/z_i,
        (*,w_i)/w_i]$,

\\for all $1 \le i \le n$: $\UPhi'(u_i) = (t_i,t')$,
$\UPhi'(v_i) = (f_i,t')$,
$\UPhi'(w_i) = \UPhi'(t') = \true$.


\end{enumerate}

 It can be verified, using similar arguments as before, that 
$s \mr t$ if and only if \linebreak
$\forall x \exists y \forall z \exists w : \varphi(x,y,z,w)$.
\end{proof}


Although the complexity may seem discouraging in many cases, there is an 
important remark to make. The refinement checking may be exponential,
but only in the outdegree of each state and the number of
parameters, while it is polynomial in the number of states. 
As one may expect the outdegree and the number of parameters to be much smaller
than the number of states, this means that the refinement checking
may still be done in a~rather efficient way. This claim is furthermore
supported by the existence of efficient SAT solvers that may be employed 
to check the inner conditions in the modal refinement.





\section{Conclusion and Future Work} 

We have introduced an extension of modal transition systems
called PMTS for parametric systems. 
The formalism is general enough to capture several features missing in
the other extensions, while at the same time it offers an easy 
to understand semantics and a natural notion of modal refinement
that specializes to the well-known refinements 
already studied on the subclasses of PMTS. Finally, we provided a comprehensive
overview of complexity of refinement checking on PMTS and its subclasses.

We believe that our formalism is a step towards a more applicable 
notion of specification theories based on MTS.

In the future work
we will study logical characterizations of the refinement relation,
investigate compositional properties 
and focus on introducing quantitative aspects into the model in order
to further increase its applicability. 


\paragraph{Acknowledgments.} We would like to thank to
Sebastian Bauer for suggesting the traffic light example
and for allowing us to use his figure environments.
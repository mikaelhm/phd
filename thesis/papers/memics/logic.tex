\section{Characterizing Logic}
In this section we define a new logic for reasoning about BMTS specifications. The logic is based on Hennessy Milner Logic (HML) with recursion \cite{DBLP:conf/caap/Larsen88}. Originally \cite{DBLP:journals/jacm/HennessyM85} HML was defined on LTS and has been proven to characterize bisimulation. 

The syntax is identical to HML with recursion, we actually only extend the semantics to capture the modalities of BMTS. We follow the approach used in~\cite{DBLP:journals/iandc/SteffenI94}, for more see~\cite{AILS:reactive:07}. 
The main idea is to present a way to construct the characteristic formula for a BMTS, and then prove that the satisfaction of the logic coincide with modal refinement checking.

We build the characteristic formula as a mutually recursive equation system with an equation for each state of the BMTS.

Let $\mathcal{X} = \{X_1,\ldots,X_n\}$ be a set of variables, then the set $\mathcal{M_X}$ of formulas over a set of actions $\Sigma$ is given by the following abstract syntax.
\[
 F,G =:: 	X \mid
			\true \mid 
			\false \mid 
			F \wedge G \mid 
			F \vee G \mid 
			\ldia{a} F \mid
			\lbox{a} F\; ,
\]
where $a \in \Sigma$ and $X \in \mathcal{X}$.

A mutually recursive equation system is a set of equations like 
\begin{align*}
	X_1=F_{X_1} \\
	X_2=F_{X_2} \\
	\vdots \\
 	X_n=F_{X_n}\;,
\end{align*}	
where $X_1, \ldots, X_n$ are unique variables and $F_{X_1},\ldots, F_{X_n}$ are logic formulas that each may contain several of the variables. The equation system is formally given by a declaration, a function $D: \mathcal{X} \goes{} \mathcal{M_X}$, assigning a formula to each variable in $\mathcal{X}$.

Given a BMTS $(S,T,\UPhi,s_0)$ over a set of actions $\Sigma$ we define the
denotational semantics as a function $\Osat{F}: (2^S)^n \rightarrow 2^S$  
inductively for $F \in \mathcal{M_X}$ by
	\begin{align*}
		\Osat{X_i}(S_1,\ldots,S_n) &= S_i\\
		\Osat{\true}(S_1,\ldots,S_n) &= S\\		
		\Osat{\false}(S_1,\ldots,S_n) &= \emptyset \\
		\Osat{F \wedge G}(S_1,\ldots,S_n) &= \Osat{F}(S_1,\ldots,S_n) \cap \Osat{G}(S_1,\ldots,S_n) \\
		\Osat{F \vee G}(S_1,\ldots,S_n) &= \Osat{F}(S_1,\ldots,S_n) \cup \Osat{G}(S_1,\ldots,S_n) \\
		\Osat{\ldia{a} F}(S_1,\ldots,S_n) &= \diad{a}\Osat{F}(S_1,\ldots,S_n) \\
		\Osat{\lbox{a} F}(S_1,\ldots,S_n) &= \boxd{a}\Osat{F}(S_1,\ldots,S_n)\;,
	\end{align*}
where \diad{a} and \boxd{a} are defined as follows
\begin{align*}
	\diad{a}S' &= \{s \in S \mid \forall M \in \val(s) ,\; \exists (a,s'') \in M : s'' \in S'\}\\
	\boxd{a}S' &= \{s \in S \mid \forall M \in \val(s) ,\; \forall (a,s') \in M : s' \in S'\}.
\end{align*}

The modal logic presented here is basically HML with recursion, with $\boxd{a}$ and $\diad{a}$ defined to match must and may modalities in BMTSs.

We define $\diad{a}$ so it corresponds to a must transition, recall that must means that there has to be an $a$ transition no matter how one satisfies the obligation formula. Note that it is enough to quantify over $\forall M \in \val(s)$, as $\UPhi(s)$ is satisfiable, so there exists at least one $M \in \val(s)$.

We define $\boxd{a}$ so it corresponds to may transitions, i.e. a transition that can be there in any of the possible configurations of a state. Also note that the definition of $\boxd{a}$ and $\diad{a}$ are closely related to the two requirements in modal refinement in Definition~\ref{def:bool-mr}.

The return value of $\Osat{F}(S_1,\ldots,S_n)$ is the set of states satisfying $F$ assuming that $S_i$ is the set of states satisfying $X_i$ for $1 \leq i \leq n$.  

As we are dealing with a mutually recursive equation system declared by a function $D$, the solution of such can be a maximal fix-point. Let $\setsat{D}$ be the \emph{semantic function} $\setsat{D} : \mathcal{D} \rightarrow \mathcal{D}$ defined by
	\[
		\setsat{D}(S_1,\ldots,S_n) = (\Osat{F_{X_1}}(S_1,\ldots,S_n),\ldots,\Osat{F_{X_n}}(S_1,\ldots,S_n)),
	\]
	where $S_i \subseteq S$ for $1 \leq i \leq n$.

As $\setsat{D}$ is monotonic, Tarski's fixed-point theorem gives us that there exists a unique maximal fix-point. 
We define the satisfaction relation for every $i$ as

\[
	s \models_{max} D(X_i) \iff s \in S_i, \text{ where } (S_1,\ldots,S_n) = \bigcap \{O \mid O \supseteq \setsat{D}(O)\}\;.
\]

We only consider the maximal fix-points, so from now on we will omit the $max$ subscript. We shall sometimes abuse the notation a bit, by applying the satisfaction relation directly on a BTMS and a declaration, like $ \mathcal S \models D_\mathcal T$. This just means that $s_0 \models \DecT(X_{t_0})$, where $s_0$ and $t_0$ are the initial states of $\mathcal S$ and $\mathcal T$.

Now given a BTMS $\mathcal{T}$ we will define a declaration $\DecT$ such that for all other BMTSs $\mathcal{S}$ we know that $\mathcal{S}$ is in modal refinement of $\mathcal{T}$ if and only if $\mathcal{S}$ satisfies $\DecT$.

\begin{theorem}\label{thm:charataristic}
	Given two BMTSs $\mathcal{T}=(S_1,T_1,\UPhi_1,t_0)$, $\mathcal{S} = (S_2,T_2,\UPhi_2,s_0)$. Let $\DecT$ be constructed as follows for each state $t \in S_1$ 
	\[\label{eqn:logic-construction}
		\DecT(X_t) = 	%	\bigvee_{\mu \subseteq P} 
				 	\bigvee_{ N \in \val(t)}
					\left [ 
						\left ( 
							\bigwedge_{(a,t') \in N}\ldia{a}X_{t'}
						\right ) 
						\wedge 
						\bigwedge_{a \in \Sigma}\lbox{a}
							\left( 
								\bigvee_{(a,t') \in N}(X_{t'}) 
							\right)
					\right ],
	\]
	then 
	\[
		\mathcal{S} \models \DecT \iff S \mr T\,.
	\]
\end{theorem}
We note that \eqref{eqn:logic-construction} generalizes the characteristic formula construction in \cite{DBLP:journals/iandc/SteffenI94} and \linebreak \cite{AILS:reactive:07} with the disjunction in front. 

Before we prove Theorem~\ref{thm:charataristic}, let us again have a look at the example from last section.
\begin{example}
 The logical characterization of $\mathcal{S}_1$ looks like this,
	\begin{align*}
		D_{\mathcal{S}_1}(X_{s_1}) =& ( \ldia{\Coin} X_{s_2} ) \wedge ( \lbox{\Coin} X_{s_2} \wedge \lbox{\Choc} \false \wedge \lbox{\Coffee} \false \wedge \lbox{\Tea} \false ) \\
	    D_{\mathcal{S}_1}(X_{s_2}) =
	    & \Bigg{(} ( \ldia{\Choc} X_{s_1} \wedge \ldia{\Coffee} X_{s_1} \wedge \ldia{\Tea} X_{s_1}  ) \wedge \\ &\quad( \lbox{\Coin} \false \wedge \lbox{\Choc} X_{s_1} \wedge \lbox{\Coffee} X_{s_1} \wedge \lbox{\Tea} X_{s_1}  )
	\Bigg{)} \vee\\
		& \Bigg{(} ( \ldia{\Choc} X_{s_1} \wedge \ldia{\Coffee} X_{s_1}  ) \wedge \\ &\quad( \lbox{\Coin} \false \wedge \lbox{\Choc} X_{s_1} \wedge \lbox{\Coffee} X_{s_1} \wedge \lbox{\Tea} \false  )
	\Bigg{)} \vee\\
		& \Bigg{(}( \ldia{\Choc} X_{s_1} \wedge \ldia{\Tea} X_{s_1}  ) \wedge\\ &\quad  ( \lbox{\Coin} \false \wedge \lbox{\Choc} X_{s_1} \wedge \lbox{\Coffee} \false \wedge \lbox{\Tea} X_{s_1}  )
	\Bigg{)} \vee\\
		 & \Bigg{(}( \ldia{\Coffee} X_{s_1} \wedge \ldia{\Tea} X_{s_1}  ) \wedge \\ &\quad( \lbox{\Coin} \false \wedge \lbox{\Choc} \false \wedge \lbox{\Coffee} X_{s_1} \wedge \lbox{\Tea} X_{s_1}  )
		\Bigg{)}\; .
	\end{align*}
\end{example}


We will prove Theorem~\ref{thm:charataristic} by the following two lemmas.


\begin{lemma}\label{lem:sat-impl-mr}
		Let $\DecT(X_t)$ be defined as in \eqref{eqn:logic-construction}, then for each state $t \in S_1$ we have

	\[
		s \mr t \text{ implies } s \models \DecT(X_t)\, ,
	\]
	for all $s \in S_2$.
\end{lemma}
\begin{proof}
 	Assume that $s \mr t$. We will show that $s \in \satset{\DecT(X_t)}$, the set of all states that satisfy $\DecT(X_t)$. Further let $\mrset{t}= \{s \mid s \mr t\}$ be the set of all states in modal refinement of $t$.

As the satisfaction relation is defined as a maximal fix-point, we need to show that  
	\[
		s \in \bigcup_{ N \in \val(t)}
		\left [ 
			\left ( 
				\bigcap_{(a,t') \in N}\diad{a}\mrset{t'}
			\right ) 
			\cap 
			\bigcap_{a \in \Sigma}\boxd{a}
				\left( 
					\bigcup_{(a,t') \in M}\mrset{t'}
				\right)
		\right ].
	\]
	Assume $M \in \val(s)$. By definition of modal refinement we know that there exists $N \in \val(t)$, such that the two requirements for modal refinement are fulfilled. Let us now use this $N$, and prove the two following cases:
\begin{enumerate}
	\item $s \in 
		\left ( 
			\bigcap_{(a,t') \in N}\diad{a}\mrset{t'}
		\right ) 
	$
	\item $s \in
	\bigcap_{a \in \Sigma}\boxd{a}
		\left ( 
			\bigcup_{(a,t') \in O}\mrset{t'}
		\right ) $
		\end{enumerate}
		
	Proof of 1. Assume $(a,t') \in N$. By modal refinement we know that for all $(a,t') \in N$ there exists $(a,s') \in M$ such that $s' \mr t'$. Thus we have that
		$s \in 
			\left ( 
				\bigcap_{(a,t') \in N}\diad{a}\mrset{t'}
			\right ) 
		$.
		
		
	Proof of 2. Assume $(a,s') \in M$. By modal refinement we get that there exists $(a,t') \in N$ st. $s' \in \mrset{t'}$. This gives us $s \in
		\bigcap_{a \in \Sigma}\boxd{a}
			\left ( 
				\bigcup_{(a,t') \in N}\mrset{t'}
			\right )$. 
\end{proof}

\begin{lemma}\label{lem:mr-impl-sat}
	Let $\DecT(X_t)$ be defined as in \eqref{eqn:logic-construction}, then for each state $t \in S_1$ we have
	
\[
	s \models \DecT(X_t) \text{ implies }  s \mr t,
\]
for all $s \in S_2$.
\end{lemma}
This lemma can be proved in the same manner as Lemma~\ref{lem:sat-impl-mr}.
%, the full proof can be found in Appendix~\ref{app-logic}.
By these two lemmas we have proved Theorem~\ref{thm:charataristic}.

Unfortunately the size of the formula is exponential in the given BMTS. In the worst case there can be exponentially many $N \in \val(s)$, so it is exponential in the number of outgoing transitions. We could however not expect it to be smaller, as modal refinement checking of BMTS was shown to be $\Pi^P_2$-complete in \cite{BKLMS:ATVA:11}.